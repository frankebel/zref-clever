\documentclass{book}


% \usepackage[ngerman]{babel}

\usepackage{zref-clever}
\usepackage[user]{zref}
\usepackage{zref-titleref}

\usepackage{chappg}

\usepackage{hyperref}

% \labelformat{section}{section~#1}

\setcounter{secnumdepth}{4}

\usepackage{cleveref}

\newtheorem{lemma}{Lemma}

\makeatletter
\AtBeginEnvironment{lemma}{\show\@currentcounter}
\makeatother

\begin{document}

\frontmatter{}


\part{Part 1}
\label{part:part-1}
\zlabel{part:part-1}

\setcounter{chapter}{12}


\chapter{Chapter 1}
\label{cha:chapter-1}
\zlabel{cha:chapter-1}

\clearpage{}

\section{Section 1.1}
\label{sec:section-1-1}
\zlabel{sec:section-1-1}
\mainmatter


\chapter{Chapter 2}
\label{cha:chapter-2}
\zlabel{cha:chapter-2}

\clearpage{}

\section{Section 1}
\label{sec:section-2-1}
\zlabel{sec:section-2-1}

\clearpage{}

\subsection{Subsection 2.1.1}
\label{sec:subsection-2.1.1}
\zlabel{sec:subsection-2.1.1}

\clearpage{}

\subsubsection{Subsubsection 2.1.1.1}
\label{sec:subs-2.1.1.1}
\zlabel{sec:subs-2.1.1.1}

\zref{sec:subs-2.1.1.1}

\begin{enumerate}
\item \zlabel{item:1}
  \begin{enumerate}
  \item \zlabel{item:1-1}
  \end{enumerate}
\end{enumerate}

\chapter{Chapter 3}
\label{cha:chapter-3}
\zlabel{cha:chapter-3}



\begin{figure}
  \centering

  \caption{Figure 3.1}
  \label{fig:figure-3-1}
  \zlabel{fig:figure-3-1}
\end{figure}

\clearpage{}

\section{Section 3.1}
\label{sec:section-3-1}
\zlabel{sec:section-3-1}

\cref{sec:section-1-1,cha:chapter-1,cha:chapter-2,sec:section-2-1,sec:subsection-2.1.1,sec:subs-2.1.1.1,cha:chapter-3,sec:section-3-1}


\zcref{undefined,sec:section-1-1,{{cha:chapter-1}},cha:chapter-2,sec:section-2-1,sec:subsection-2.1.1,sec:subs-2.1.1.1,cha:chapter-3,sec:section-3-1}


% \cref{sec:subsection-2.1.1}
% \cref{sec:subs-2.1.1.1}


\begin{lemma}
  teste
\end{lemma}

\makeatletter
\show\@currentcounter
\makeatother

\figurename

\tablename


\end{document}
