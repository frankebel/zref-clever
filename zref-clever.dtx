% \iffalse meta-comment
%
% File: zref-clever.dtx
%
% This file is part of the LaTeX package "zref-clever".
%
% Copyright (C) 2021  Gustavo Barros
%
% It may be distributed and/or modified under the conditions of the
% LaTeX Project Public License (LPPL), either version 1.3c of this
% license or (at your option) any later version.  The latest version
% of this license is in the file:
%
%    https://www.latex-project.org/lppl.txt
%
% and version 1.3 or later is part of all distributions of LaTeX
% version 2005/12/01 or later.
%
%
% This work is "maintained" (as per LPPL maintenance status) by
% Gustavo Barros.
%
% This work consists of the files zref-clever.dtx,
%                                 zref-clever.ins,
%                                 zref-clever.tex,
%                                 zref-clever-code.tex,
% and the derived files listed in MANIFEST.md.
%
% The released version of this package is available from CTAN.
%
% -----------------------------------------------------------------------
%
% The development version of the package can be found at
%
%    https://github.com/gusbrs/zref-clever
%
% for those people who are interested.
%
% -----------------------------------------------------------------------
%
% \fi
%
% \iffalse
%<*driver>
\documentclass{l3doc}

% Have \GetFileInfo pick up date and version data and used in the
% documentation.
\usepackage[cap,nameinlink=false]{zref-clever}

% Used in the documentation.
\usepackage{zref-titleref}

\MakeShortVerb{\|}

\begin{document}

\DocInput{zref-clever.dtx}

\end{document}
%</driver>
% \fi
%
%
% \begin{documentation}
%
% \section{Introduction}
%
%
%
%
% \section{Loading the package}
% \zlabel{sec:loading-package}
%
% As usual:
%
% \begin{syntax}
%   \cs{usepackage}\oarg{options}\texttt{\{zref-clever\}}
% \end{syntax}
%
%
% \section{Dependencies}
%
% \pkg{zref-clever} requires \pkg{zref}, and \LaTeX{} kernel 2021-06-01, or
% newer.  It also needs \pkg{l3keys2e} and \pkg{ifdraft}.  Some packages are
% leveraged by \pkg{zref-clever} if they are present, but are not loaded by
% default or required by it, namely: \pkg{hyperref}, \pkg{zref-check}, and
% \pkg{zref}'s modules \pkg{zref-titleref} and \pkg{zref-xr}.
%
%
% \section{User interface}
% \zlabel{sec:user-interface}
%
% \begin{function}{\zcref}
%   \begin{syntax}
%     \cs{zcref}\meta{*}\oarg{options}\marg{labels}
%   \end{syntax}
%   Typesets references to \meta{labels}, given as a comma separated list.
%   When \pkg{hyperref} support is enabled, references will be hyperlinked to
%   their respective anchors, according to options.  The starred version of
%   the command does the same as the plain one, just does not form links.  The
%   \meta{options} are (mostly) the same as those of the package, and can be
%   given to local effect.  The \meta{labels} argument is protected by
%   \pkg{zref}'s \cs{zref@wrapper@babel}, so that it enjoys the same support
%   for \pkg{babel}'s active characters as \pkg{zref} itself does.
% \end{function}
%
% \begin{function}{\zcpageref}
%   \begin{syntax}
%     \cs{zcpageref}\meta{*}\oarg{options}\marg{labels}
%   \end{syntax}
%   Typesets page references to \meta{labels}, given as a comma separated
%   list.  It is equivalent to calling \cs{zcref} with the \opt{ref=page}
%   option: \cs{zcref}\texttt{\meta{*}[}\meta{options}\texttt{,
%   ref=page]}\marg{labels}.
% \end{function}
%
%
% \begin{function}{\zcsetup}
%   \begin{syntax}
%     \cs{zcsetup}\marg{options}
%   \end{syntax}
%   Sets \pkg{zref-clever}'s general options (see \zcref{sec:options,
%   sec:reference-format}).  The settings performed by \cs{zcsetup} are local,
%   within the current group.  But, of course, it can also be used to global
%   effects if ungrouped, e.g.\ in the preamble.
% \end{function}
%
% \begin{function}{\zcRefTypeSetup}
%   \begin{syntax}
%     \cs{zcRefTypeSetup} \marg{type} \marg{options}
%   \end{syntax}
%   Sets type-specific reference format options (see
%   \zcref{sec:reference-format}).  Just as for \cs{zcsetup}, the settings
%   performed by \cs{zcRefTypeSetup} are local, within the current group.
% \end{function}
%
% \bigskip{}
%
% Besides these, user facing commands related to \zcref*[ref=title,
% noname]{sec:internationalization} are presented in
% \zcref{sec:internationalization}.  Note still that all user commands are
% defined with \cs{NewDocumentCommand}, which translates into the usual
% handling of arguments by it and/or processing by \pkg{l3keys}, particularly
% with regard to brace-stripping and space-trimming.
%
% Furthermore, \pkg{zref-clever} loads \pkg{zref}'s \pkg{zref-user} module by
% default.  So you also have its user commands available out-of-the-box,
% notably \cs{zlabel}, but also \cs{zref} and \cs{zpageref} themselves.
%
%
%
% \section{Options}
% \zlabel{sec:options}
%
% \pkg{zref-clever} is highly configurable, offering a lot of flexibility in
% typeset results of the references, but it also tries to keep these
% ``handles'' as convenient and user friendly as possible.  To this end, most
% of what one can do with \pkg{zref-clever} (pretty much all of it), can be
% achieved directly through the standard and familiar ``comma separated list
% of \texttt{key=value} options''.
%
% There are two main groups of options in \pkg{zref-clever}: ``general
% options'', which affect the overall behavior of the package, or the
% reference as a whole; and ``reference format options'', which control the
% detail of reference formatting, including type-specific and
% language-specific settings.
%
% This section covers the first group (for the second one, see
% \zcref{sec:reference-format}).  General options can be set globally either
% as package options at load-time (see \zcref{sec:loading-package}) or by
% means of \cs{zcsetup} in the preamble (see \zcref{sec:user-interface}).
% They can also be set locally with \cs{zcsetup} along the document or through
% the optional argument of \cs{zcref} (see \zcref{sec:user-interface}).  Most
% general options can be used in any of these contexts, but that is not
% necessarily true for all cases, some restrictions may apply, as described in
% each option's documentation.
%
% \bigskip{}
%
% \DescribeOption{ref} %
% \DescribeOption{page} %
% The \opt{ref} option controls the label property to which \cs{zcref} refers
% to.  It can receive values \texttt{default}, \texttt{zc@thecnt} and
% \texttt{page}.  If \pkg{zref-titleref} is loaded, \opt{ref} also accepts the
% value \texttt{title}.  The package's default is, well, \texttt{default},
% which is our standard reference.  \texttt{zc@thecnt} is a property set by
% \pkg{zref-clever} and is similar to \pkg{zref}'s \texttt{default} property,
% except that it is not affected by the kernel's
% \cs{labelformat}.\footnote{Technical note: the \texttt{default} property
% stores \cs{@currentlabel}, while the \texttt{zc@thecnt} property stores
% \cs{the}\cs{@currentcounter}.  The later is exactly what \cs{refstepcounter}
% uses to build \cs{@currentlabel}, except for the \cs{labelformat} prefix
% and, hence, has the advantage of being unaffected by it.  But the former is
% \emph{more reliable}, since \cs{@currentlabel} is expected to be correct
% pretty much anywhere whereas, even though \cs{refstepcounter} does set
% \cs{@currentcounter}, it is not everywhere that uses \cs{refstepcounter} for
% the purpose.  In the cases where the references from these two do diverge,
% \pkg{zref-clever} will likely misbehave (reference type, sorting and
% compression inevitably depend on a correct \opt{currentcounter}), but using
% \texttt{default} at least ensures that the reference itself is correct.
% That said, if you do set \cs{labelformat} for some reason,
% \texttt{zc@thecnt} may be useful.}  By default, reference formatting,
% sorting, and compression are done according to information inferred from the
% \emph{current counter} (see \opt{currentcounter} option below).  Special
% treatment in these areas is provided for \texttt{page}, but not for
% \texttt{title}.  The \opt{page} option is a convenience alias for
% \texttt{ref=page}.
%
% \DescribeOption{typeset} %
% \DescribeOption{noname} %
% When \cs{zcref} typesets a set of references, each group of references of
% the same type can be, and by default are, preceded by the type's ``name'',
% and this is indeed an important feature of \pkg{zref-clever}.  This is
% optional however, and the \opt{typeset} option controls this behavior.  It
% can receive values \texttt{ref}, in which case it typesets only the
% reference(s), \texttt{name}, in which case it typesets only the name(s), or
% \texttt{both}, in which case it typesets, well, both of them.  Note that,
% when value \texttt{name} is used, the name is still typeset according to the
% set of references given to \cs{zcref}.  For example, for multiple
% references, the plural form is used, capitalization options are honored,
% etc.  Also hyperlinking behaves just \emph{as if} the references were
% present and, depending on the corresponding options, the name may be linked
% to the first reference of the type group.  The \opt{noname} option is a
% convenience alias for \texttt{typeset=ref}.
%
% \DescribeOption{sort} %
% \DescribeOption{nosort} %
% The \opt{sort} option controls whether the list of \meta{labels} received as
% argument by \cs{zcref} should be sorted or not.  It is a boolean option, and
% defaults to \texttt{true}.  The \opt{nosort} option is a convenience alias
% for \texttt{sort=false}.
%
% \DescribeOption{typesort} %
% \DescribeOption{notypesort} %
% Sorting references of the same type can be done with well defined logical
% criteria.  They either have the same counter or their counters share a clear
% hierarchical relation (in the resetting behavior), such that a definite
% sorting rule can be inferred from the label's data.  The same is not true
% for sorting of references of different types.  Should ``tables'' come before
% or after ``figures''?  The \pkg{typesort} option allows to specify the
% sorting priority of different reference types.  It receives as value a comma
% separated list of reference types, specifying that their sorting is to be
% done in the order of that list.  But \opt{typesort} does not need to receive
% \emph{all} possible reference types.  The special value
% \texttt{\{\{othertypes\}\}} (yes, double braced, one for \pkg{l3keys}, so
% that the second can make the list) can be placed anywhere along the list, to
% specify the sort priority of any type not included explicitly in the list.
% If \texttt{\{othertypes\}} is not present in the list, it is presumed to be
% at the end of it.  Any unspecified types (that is, those falling implicitly
% or explicitly into the \texttt{\{othertypes\}} category) get sorted between
% themselves in the order of their first appearance in the label list given as
% argument to \cs{zcref}.  I presume the common use cases will not need to
% specify \texttt{\{othertypes\}} at all but, for the sake of example, if you
% just really dislike equations, you could use
% \texttt{typesort=\{\{\{othertypes\}\}, equation\}}.  \opt{typesort}'s
% default value is \texttt{\{part, chapter, section, paragraph\}}, which
% places the sectioning reference types first in the list, in their
% hierarchical order, and leaves everything else to the order of appearance of
% the labels.  The \opt{notypesort} option behaves like
% \texttt{typesort=\{\{\{othertypes\}\}\}} would do, that is, it sorts all
% types in the order of the first appearance in the labels' list.
%
% \DescribeOption{comp} %
% \DescribeOption{nocomp} %
% \cs{zcref} can automatically compress a set of references of the same type
% into a range, when they occur in immediate sequence.  The \opt{comp}
% controls whether this compression should take place or not.  It is a boolean
% option, and defaults to \texttt{true}.  The \opt{nocomp} option is a
% convenience alias for \texttt{comp=false}.  Of course, for better
% compression results the \opt{sort} is recommended, but the two options are
% technically independent.
%
% \DescribeOption{range} %
% By default (that is, when the \opt{range} option is not given), \cs{zcref}
% typesets a complete list of references according to the \meta{labels} it
% received as argument, and only compresses some of them into ranges if the
% \opt{comp} option is enabled and if references of the same type occur in
% immediate sequence.  The \opt{range} option makes \cs{zcref} behave
% differently.  Sorting is implied by this option (the \opt{sort} option is
% disregarded) and, for each reference type group in \meta{labels}, \cs{zcref}
% builds a range from the first to the last reference in it, even if
% references in between do not occur in immediate sequence.  \cs{zcref} is
% smart enough, though, to recognize when the first and last references of a
% type do happen to be contiguous, in which case it typesets a ``pair'',
% instead of a ``range''.  It is a boolean option, and the package's default
% is \texttt{range=false}.  The option given without a value is equivalent to
% \texttt{range=true} (in the \pkg{l3keys}' jargon, the \emph{option}'s
% default is \texttt{true}).
%
% \DescribeOption{cap} %
% \DescribeOption{nocap} %
% \DescribeOption{capfirst} %
% The \opt{cap} option controls whether the reference type names should be
% capitalized or not.  It is a boolean option, and the package's default is
% \texttt{cap=false}.  The option given without a value is equivalent to
% \texttt{cap=true}.  The \opt{nocap} option is a convenience alias for
% \texttt{cap=false}.  The \opt{capfirst} option ensures that the reference
% type name of the \emph{first} type block is capitalized, even when \opt{cap}
% is set to \texttt{false}.
%
% \DescribeOption{abbrev} %
% \DescribeOption{noabbrev} %
% \DescribeOption{noabbrevfirst} %
% The \opt{abbrev} option controls whether to use abbreviated reference type
% names when they are available.  It is a boolean option, and the package's
% default is \texttt{abbrev=false}.  The option given without a value is
% equivalent to \texttt{abbrev=true}.  The \opt{noabbrev} option is a
% convenience alias for \texttt{abbrev=false}.  The \opt{noabbrevfirst}
% ensures that the reference type name of the \emph{first} type block is never
% abbreviated, even when \opt{abbrev} is set to \texttt{true}.
%
% \DescribeOption{S} %
% \opt{S} for ``Sentence''.  The \opt{S} option is a convenience alias for
% \texttt{capfirst=true, noabbrevfirst=true}, and is intended to be used in
% references made at the beginning of a sentence.  It is highly recommended
% that you make a habit of using the \opt{S} option for beginning of sentence
% references.  Even if you do happen to be currently using \texttt{cap=true,
% abbrev=false}, proper semantic markup will ensure you get expected results
% even if you change your mind in that regard later on.  For that reason, it
% was made short and mnemonic, it can't get any easier.
%
% \DescribeOption{hyperref} %
% The \opt{hyperref} option controls the use of \pkg{hyperref} by
% \pkg{zref-clever} and takes values \opt{auto}, \opt{true}, \opt{false}.  The
% default value, \opt{auto}, makes \pkg{zref-clever} use \pkg{hyperref} if it
% is loaded, meaning that references made with \cs{zcref} get hyperlinked to
% the anchors of their respective \meta{labels}.  \opt{true} does the same
% thing, but warns if \pkg{hyperref} is not loaded (\pkg{hyperref} is never
% loaded for you).  In either of these cases, if \pkg{hyperref} is loaded,
% module \pkg{zref-hyperref} is also loaded by \pkg{zref-clever}.  \opt{false}
% means not to use \pkg{hyperref} regardless of its availability.  This is a
% preamble only option, but \cs{zcref} provides granular control of
% hyperlinking by means of its starred version.
%
% \DescribeOption{nameinlink} %
% The \opt{nameinlink} option controls whether the type name should be
% included in the reference hyperlink or not (provided there is a link, of
% course).  Naturally, the name can only be included in the link of the
% \emph{first} reference of each type block.  \opt{nameinlink} can receive
% values \texttt{true}, \texttt{false}, \texttt{single}, and \texttt{tsingle}.
% When the value is \texttt{true} the type name is always included in the
% hyperlink.  When it is \texttt{false} the type name is never included in the
% link.  When the value is \texttt{single}, the type name is included in the
% link only if \cs{zcref} is typesetting a single reference (not necessarily
% having received a single label as argument, as they may have been
% compressed), otherwise, the name is left out of the link.  When the value is
% \texttt{tsingle}, the type name is included in the link for each type block
% with a single reference, otherwise, it isn't.  An example: suppose you make
% a couple of references to something like
% \cs{zcref}\texttt{\{chap:chapter1\}} and \cs{zcref}\texttt{\{chap:chapter1,
% sec:section1, fig:figure1, fig:figure2\}}.  The ``figure'' type name will
% only be included in the hyperlink if \opt{nameinlink} option is set to
% \texttt{true}.  If it is set to \texttt{tsingle}, the first reference will
% include the name in the link for ``chapter'', as expected, but also in the
% second reference the ``chapter'' and ``section'' names will be included in
% their respective links, while that of ``figure'' will not.  If the option is
% set to \texttt{single}, only the name for ``chapter'' in the first reference
% will be included in the link, while in the second reference none of them
% will.  The package's default is \texttt{nameinlink=tsingle}, and the option
% given without a value is equivalent to \texttt{nameinlink=true}.
%
% \DescribeOption{preposinlink} %
% The \opt{preposinlink} option controls whether the \opt{refpre} and
% \opt{refpos} reference format options (see \zcref{sec:reference-format}) are
% included in the reference hyperlink or not.  It is a boolean option.  The
% package's default is \texttt{preposinlink=false}, and the option given
% without a value is equivalent to \texttt{preposinlink=true}.
%
% \DescribeOption{lang} %
% The \opt{lang} option controls the language used by \cs{zcref} when looking
% for language-specific reference format options (see
% \zcref{sec:reference-format}).  The default value, \texttt{main}, uses the
% main document language, as defined by \pkg{babel} or \pkg{polyglossia} (or
% \texttt{english} if none of them is loaded).  Value \texttt{current} uses
% the current language, as defined by \pkg{babel} or \pkg{polyglossia} (or
% \texttt{english} if none of them is loaded).  The \opt{lang} option also
% accepts that the language be specified directly by its name, as long as it's
% a language known by \pkg{zref-clever}.  For more details on
% \zcref*[ref=title,noname]{sec:internationalization}, see
% \zcref{sec:internationalization}.
%
% \DescribeOption{d} %
% The \opt{d} option sets the declension case, and affects the type name used
% for typesetting the reference.  Whether this option is operative, and which
% values it accepts, depends on the declared setup for each language.  For
% details, see \zcref{sec:internationalization}.
%
% \DescribeOption{nudge} %
% \DescribeOption{nudgeif} %
% \DescribeOption{nonudge} %
% \DescribeOption{sg} %
% \DescribeOption{g} %
% This set of options revolving around \opt{nudge} aims to offer some guard
% against mischievous automation on the part of \pkg{zref-clever} by providing
% a number of ``nudges'' (compilation time messages) for cases in which you
% may wish to revise material \emph{surrounding} the reference -- an article,
% a preposition -- according to the reference typeset results.  Useful mainly
% for languages which inflect the preceding article to gender and/or number,
% but may be used generally to fine-tune the language and style around the
% cross-references made with \cs{zcref}.  The \opt{nudge} option is the main
% entrance to this feature and takes values \texttt{true}, \texttt{false},
% \texttt{obeydraft}, or \texttt{obeyfinal}.  The first two, respectively,
% enable or disable the ``nudging'' unconditionally.  With \texttt{obeydraft},
% \opt{nudge} keeps quiet when option \texttt{draft} is given to
% \cs{documentclass}, while with \texttt{obeyfinal}, nudging is only enabled
% when option \texttt{final} is (explicitly) passed to \cs{documentclass}.
% The option given without a value is equivalent to \texttt{nudge=true} and
% the package's default is \texttt{nudge=false}.  \opt{nonudge} is a
% convenience alias for \texttt{nudge=false}, and can be used to silence
% individual references.  The \opt{nudgeif} option controls the events which
% may trigger a nudge.  It takes a comma separated list of elements, and
% recognizes values \texttt{multitype}, \texttt{comptosing}, \texttt{gender},
% and \texttt{all}.  The \texttt{multitype} nudge warns when the reference is
% composed by multiple type blocks (see \zcref{sec:reference-format}).  The
% \texttt{comptosing} nudge let's you know when multiple labels of the same
% type have been compressed to a singular type name form.  It can be combined
% with the \opt{sg} option, which is the way to tell \cs{zcref} you know it's
% a singular and so not to nudge if a compression to singular occurs, but to
% nudge if the contrary occurs, that is, when a plural type name form is
% employed.  The \texttt{gender} nudge must be combined with option \opt{g},
% and depends on the language having support for it.  In essence language
% dictionaries can store the gender of each type name (this is done for
% built-in dictionaries, but can also be done with \cs{zcLanguageSetup} for
% languages declared to support it).  The \opt{g} option let's you specify the
% gender you expect for that particular reference and the nudge is triggered
% if there is a mismatch between \opt{g} and the gender for the type name in
% the dictionary.  Both the \texttt{comptosing} and the \texttt{gender} nudges
% have a type block as its scope.  See \zcref{sec:internationalization} for
% more details and intended use cases of the ``nudging'' feature.
%
% \DescribeOption{font} %
% The \opt{font} option can receive font styling commands to change the
% appearance of the whole reference list (see also the reference format
% options, \opt{namefont} and \opt{reffont} in \zcref{sec:reference-format}).
% It does not affect the content of the \opt{note}, however.  The option is
% intended exclusively for commands that only change font attributes: style,
% family, shape, weight, size, color, etc.  Anything else, particularly
% commands that may generate typeset output, is not supported.  Given how
% package options are handled by \LaTeX{}, the fact that this option receives
% commands as value means this option \emph{can't} be set at load time, as a
% package option.  If you want to set it globally, use \cs{zcsetup} instead.
%
% \DescribeOption{titleref} %
% The \opt{titleref} option receives no value and, when given, loads
% \pkg{zref}'s \pkg{zref-titleref} module.  This is a preamble only option.%
%
% \DescribeOption{note} %
% The \opt{note} option receives as value some text to be typeset at the end
% of the whole reference list.  It is separated from it by \opt{notesep} (see
% \zcref{sec:reference-format}).
%
% \DescribeOption{check} %
% Provides integration of \pkg{zref-clever} with the \pkg{zref-check} package.
% In the preamble, the \opt{check} option receives no value and, when given,
% loads \pkg{zref-check}.  In the document body, \opt{check} requires a value,
% which works exactly like the optional argument of \cs{zcheck}, and can
% receive both checks and \cs{zcheck}'s options.  And the checks are performed
% for each label in \marg{labels} received as argument by \cs{zcref}.  See the
% User manual of \pkg{zref-check} for details.  The checks done by the
% \opt{check} option in \cs{zcref} comprise the complete reference, including
% the \opt{note} (see \zcref{sec:reference-format}).  If \pkg{zref-check} was
% not loaded in the preamble, at begin document the option is made no-op and
% issues a warning.
%
% \DescribeOption{countertype} %
% The \opt{countertype} option allows to specify the ``reference type'' of
% each counter, which is stored as a label property when the label is set.
% This ``reference type'' is what determines how a reference to this label
% will eventually be typeset when it is referred to (see
% \zcref{sec:reference-types}).  A value like \texttt{countertype = \{foo =
% bar\}} sets the \texttt{foo} counter to use the reference type \texttt{bar}.
% There's only need to specify the \opt{countertype} for counters whose name
% differs from that of their type, since \pkg{zref-clever} presumes the type
% has the same name as the counter, unless otherwise specified.  Also, the
% default value of the option already sets appropriate types for basic
% \LaTeX{} counters, including those from the standard classes.  Setting a
% counter type to an empty value removes any (explicit) type association for
% that counter, in practice, this means it then uses a type equal to its name.
% Since this option only affects how labels are set, it is not available in
% \cs{zcref}.
%
% \DescribeOption{\raisebox{-.2em}{\dbend}\ counterresetters} %
% \DescribeOption{counterresetby} %
% The sorting and compression of references done by \cs{zcref} requires that
% we know the counter whose \cs{refstepcounter} is being stored by \cs{zlabel}
% but also information on any counter whose stepping may trigger its
% resetting, or its ``enclosing counters''.  This information is not easily
% retrievable from the counter itself but is (normally) stored with the
% counter that does the resetting.  The \opt{counterresetters} option adds
% counter names, received as a comma separated list, to the list of counters
% \pkg{zref-clever} uses to search for ``enclosing counters'' of the counter
% for which a label is being set.  Unfortunately, not every counter gets reset
% through the standard machinery for this, including some \LaTeX{} kernel ones
% (e.g. the \texttt{enumerate} environment counters).  For those, there is
% really no way to retrieve this information directly, so we have to just tell
% \pkg{zref-clever} about them.  And that's what the \opt{counterresetby}
% option is made for.  It receives a comma separated list of
% \texttt{key=value} pairs, in which \texttt{key} is the counter, and
% \texttt{value} is its ``enclosing counter'', that is, the counter whose
% stepping results in its resetting.  This is not really an ``option'' in the
% sense of ``user choice'', it is more of a way to inform \pkg{zref-clever} of
% something it cannot know or automatically find in general.  One cannot place
% arbitrary information there, or \pkg{zref-clever} can be thoroughly
% confused.  The setting must correspond to the actual resetting behavior of
% the involved counters.  \opt{counterresetby} has precedence over the search
% done in the \opt{counterresetters} list.  The default value of
% \opt{counterresetters} includes the counters for sectioning commands of the
% standard classes which, in most cases, should be the relevant ones for
% cross-referencing purposes.  The default value of \opt{counterresetby}
% includes the \texttt{enumerate} environment counters.  So, hopefully, you
% don't need to ever bother with either of these options.  But, if you do,
% they are here.  Use them with caution though.  Since these options only
% affect how labels are set, they are not available in \cs{zcref}.
%
% \DescribeOption{\raisebox{.4em}{\dbend}\ currentcounter} %
% \LaTeX{}'s \cs{refstepcounter} sets two variables which potentially affect
% the \cs{zlabel} set after it: the current label (\cs{@currentlabel}), and
% the current counter (\cs{@currentcounter}).  Actually, traditionally, only
% the current label was thus stored, the current counter was added to
% \cs{refstepcounter} somewhat recently (with the 2020-10-01 kernel release).
% But, since \pkg{zref-clever} relies heavily on the information of what the
% current counter is, it must set \pkg{zref} to store that information with
% the label, as it does.  As long as the document element we are trying to
% refer to uses the standard machinery of \cs{refstepcounter} we are on solid
% ground and can retrieve the correct information.  However, it is not always
% ensured that \cs{@currentcounter} is kept up to date.  For example, packages
% which handle labels specially, for one reason or another, may or may not set
% \cs{@currentcounter} as required.  Considering the addition of
% \cs{@currentcounter} to \cs{refstepcounter} itself is not that old, it is
% likely that in a good number of places a reliable \cs{@currentcounter} is
% not really in place.  Therefore, it may happen we need to tell
% \pkg{zref-clever} what the current counter is in certain circumstances, and
% that's what \opt{currentcounter} does.  The same as with the previous two
% options, this is not really an ``user choice'' kind of option, but a way to
% tell \pkg{zref-clever} a piece of information it has no means to retrieve
% automatically.  The setting must correspond to the actual ``current
% counter'', meaning here ``the counter underlying \cs{@currentlabel}'' in a
% given situation.  Its default value is, quite naturally,
% \cs{@currentcounter}.  Since this option only affects how labels are set, it
% is not available in \cs{zcref}.
%
%
%
% \section{Reference types}
% \zlabel{sec:reference-types}
%
% A ``reference type'' is the basic \pkg{zref-clever} setup unit for
% specifying how a cross-reference group of a certain kind is to be typeset.
% Though, usually, it will have the same name as the underlying \LaTeX{}
% \emph{counter}, they are conceptually different.  \pkg{zref-clever} sets up
% \emph{reference types} and an association between each \emph{counter} and
% its \emph{type}, it does not define the counters themselves, which are
% defined by your document.  One \emph{reference type} can be associated with
% one or more \emph{counters}, and a \emph{counter} can be associated with
% different \emph{types} at different points in your document.  But each label
% is stored with only one \emph{type}, as specified by the counter-type
% association at the moment it is set, and that determines how the reference
% to that label is typeset.  References to different \emph{counters} of the
% same \emph{type} are grouped together, and treated alike by \cs{zcref}.  A
% \emph{reference type} may be known to \pkg{zref-clever} when the
% \emph{counter} it is associated with is not actually defined, and this
% inconsequential.  In practice, the contrary may also happen, a
% \emph{counter} may be defined but we have no \emph{type} for it, but this
% must be handled by \pkg{zref-clever} as an error (at least, if we try to
% refer to it), usually a ``missing name'' error.
%
% \pkg{zref-clever} provides default settings for the following reference
% types: \texttt{part}, \texttt{chapter}, \texttt{section},
% \texttt{paragraph}, \texttt{appendix}, \texttt{subappendix}, \texttt{page},
% \texttt{line}, \texttt{figure}, \texttt{table}, \texttt{item},
% \texttt{footnote}, \texttt{note}, \texttt{equation}, \texttt{theorem},
% \texttt{lemma}, \texttt{corollary}, \texttt{proposition},
% \texttt{definition}, \texttt{proof}, \texttt{result}, \texttt{remark},
% \texttt{example}, \texttt{algorithm}, \texttt{listing}, \texttt{exercise},
% \texttt{solution}.  Therefore, if you are using a language for which
% \pkg{zref-clever} has built-in support (see
% \zcref{sec:internationalization}), these reference types are available for
% use out-of-the-box.\footnote{There may be slight availability differences
% depending on the language, but \pkg{zref-clever} strives to keep this
% complete list available for the languages it has built-in dictionaries.}
% And, in any case, it is always easy to setup custom reference types with
% \cs{zcRefTypeSetup} or \cs{zcLanguageSetup} (see \zcref{sec:user-interface,
% sec:reference-format, sec:internationalization}).
%
% The association of a \emph{counter} to its \emph{type} is controlled by the
% \opt{countertype} option.  As seen in its documentation, \pkg{zref-clever}
% presumes the \emph{type} to be the same as the \emph{counter} unless
% instructed otherwise by that option.  This association, as determined by the
% local value of the option, affects how the \emph{label} is set, which stores
% the type among its properties.  However, when it comes to typesetting, that
% is from the perspective of \cs{zcref}, only the \emph{type} matters.  In
% other words, how the reference is supposed to be typeset is determined at
% the point the \emph{label} gets set.  In sum, they may be namesakes (or
% not), but type is type and counter is counter.
%
% Indeed, a reference type can be associated with multiple counters because we
% may want to refer to different document elements, with different
% \emph{counters}, as a single \emph{type}, with a single name.  One prominent
% case of this are sectioning commands.  \cs{section}, \cs{subsection}, and
% \cs{subsubsection} have each their counter, but we'd like to refer to all of
% them by ``sections'' and group them together.  The same for \cs{paragraph}
% and \cs{subparagraph}.
%
% There are also cases in which we may want to use different \emph{reference
% types} to refer to document objects sharing the same \emph{counter}.
% Notably, the environments created with \LaTeX{}'s \cs{newtheorem} command
% and the \cs{appendix}.
%
%
% One more observation about ``reference types'' is due here.  A \emph{type}
% is not really ``defined'' in the sense a variable or a function is.  It is
% more of a ``string'' which \pkg{zref-clever} uses to look for a whole set of
% type-specific reference format options (see \zcref{sec:reference-format}).
% Each of these options individually may be ``set'' or not, ``defined'' or
% not.  And, depending on the setup and the relevant precedence rules for
% this, some of them may be required and some not.  In practice,
% \pkg{zref-clever} uses the \emph{type} to look for these options when it
% needs one, and issues a compilation warning when it cannot find a suitable
% value.
%
%
%
% \section{Reference format}
% \zlabel{sec:reference-format}
%
% Formatting how the reference is to be typeset is, quite naturally, a big
% part of the user interface of \pkg{zref-clever}.  In this area, we tried to
% balance ``flexibility'' and ``user friendliness''.  But the former does
% place a big toll overall, since there are indeed many places where tweaking
% may be desired, and the settings may depend on at least two important
% dimensions of variation: the reference type and the language.  Combination
% of those necessarily makes for a large set of possibilities.  Hence, the
% attempt here is to provide a rich set of ``handles'' for fine tuning the
% reference format but, at the same time, do not \emph{require} detailed setup
% by the users, unless they really want it.
%
% With that in mind, we have settled with an user interface for reference
% formatting which allows settings to be done in different scopes, with more
% or less overarching effects, and some precedence rules to regulate the
% relation of settings given in each of these scopes.  There are four scopes
% in which reference formatting can be specified by the user, in the following
% precedence order: i) as general \emph{options}; ii) as \emph{type-specific
% options}; iii) as \emph{language-specific and type-specific translations};
% and iv) as \emph{default translations} (that is, language-specific but not
% type-specific).  Besides those, there's a fifth \emph{internal} scope, with
% the least priority of all, a ``fallback'', for the cases where it is
% meaningful to provide some value, even for an unknown language.
%
% General ``options'' (i) can be given by the user in the optional argument of
% \cs{zcref}, but also set through \cs{zcsetup} or even, depending on the
% case, as package options at load-time (see
% \zcref{sec:options}).\footnote{The use of \cs{zcsetup} for global reference
% format settings is recommended though.  Whether you can use load-time
% options or not depends on the values of the options: due to how \LaTeX{}
% handles package options, if the values of the options you are setting
% include \emph{commands} you can't set them at load-time, and rather
% \emph{must} use \cs{zcsetup}.}  ``Type-specific options'' (ii) are handled
% by \cs{zcRefTypeSetup} (see \zcref{sec:user-interface}).
% ``Language-specific translations'', be they ``type-specific'' (iii) or
% ``default'' (iv) have their user interface in \cs{zcLanguageSetup}, and have
% their values populated by the package's built-in dictionaries (see
% \zcref{sec:internationalization}).  Not all reference format specifications
% can be given in all of these scopes, though.  Some of them can't be
% type-specific, others must be type-specific, so the set available in each
% scope depends on the pertinence of the case.  \zcref{tab:reference-format}
% introduces the available reference format options, which will be discussed
% in more detail soon, and lists the scopes in which each is available.
%
%
% \begin{table}[htb]
%   \centering
%   \begin{tabular}{l>{\ttfamily}lcccc}
%     \toprule
%                     &            & General & Type    & Type-specific & Default      \\
%                     &            & options & options & translations  & translations \\
%                     &            &   (i)   &  (ii)   &    (iii)      &    (iv)      \\
%
%     \midrule
%     Necessarily not & tpairsep   &    ●    &         &               &      ●       \\
%     type-specific   & tlistsep   &    ●    &         &               &      ●       \\
%                     & tlastsep   &    ●    &         &               &      ●       \\
%                     & notesep    &    ●    &         &               &      ●       \\
%
%     \addlinespace
%     Possibly        & namesep    &    ●    &    ●    &       ●       &      ●       \\
%     type-specific   & pairsep    &    ●    &    ●    &       ●       &      ●       \\
%                     & listsep    &    ●    &    ●    &       ●       &      ●       \\
%                     & lastsep    &    ●    &    ●    &       ●       &      ●       \\
%                     & rangesep   &    ●    &    ●    &       ●       &      ●       \\
%                     & refpre     &    ●    &    ●    &       ●       &      ●       \\
%                     & refpos     &    ●    &    ●    &       ●       &      ●       \\
%
%     \addlinespace
%     Necessarily     & Name-sg    &         &    ●    &       ●       &              \\
%     type-specific   & name-sg    &         &    ●    &       ●       &              \\
%                     & Name-pl    &         &    ●    &       ●       &              \\
%                     & name-pl    &         &    ●    &       ●       &              \\
%                     & Name-sg-ab &         &    ●    &       ●       &              \\
%                     & name-sg-ab &         &    ●    &       ●       &              \\
%                     & Name-pl-ab &         &    ●    &       ●       &              \\
%                     & name-pl-ab &         &    ●    &       ●       &              \\
%
%     \addlinespace
%     Font            & namefont   &    ●    &    ●    &               &              \\
%     options         & reffont    &    ●    &    ●    &               &              \\
%     \bottomrule
%   \end{tabular}
%   \caption{Reference format options and their scopes}
%   \zlabel{tab:reference-format}
% \end{table}
%
%
% The package itself places the default setup for reference formatting at low
% precedence levels, and the users can easily and conveniently override them
% as desired.  Indeed, I expect most of the users' needs to be normally
% achievable with the general options and type-specific options, since
% references will normally be typeset in a single language (the document's
% main language) and, hence, multiple translations don't need to be provided.
%
% Understanding the role of each of these reference format options is likely
% eased by some visual schemes of how \pkg{zref-clever} builds a reference
% based on the labels' data and the value of these options. Take a
% \texttt{ref} to be that which a standard \LaTeX{} \cs{ref} would typeset.  A
% \pkg{zref-clever} ``reference block'', or \texttt{ref-block}, is constructed
% as:
%
% \begin{refformat}
% \item \reffmt{ref-block} \(\equiv\)
% \item \reffmt{refpre} \reffmt{ref} \reffmt{refpos}
% \end{refformat}
%
% A \texttt{ref-block} is built for \emph{each} label given as argument to
% \cs{zcref}.  When the \meta{labels} argument is comprised of multiple
% labels, each ``reference type group'', or \texttt{type-group} is potentially
% made from the combination of single reference blocks, ``reference block
% pairs'', ``reference block lists'', or ``reference block ranges'', where
% each is respectively built as:
%
% \begin{refformat}
% \item \reffmt{type-group} is a combination of:
% \item \reffmt{ref-block}
% \item \reffmt{ref-block1} \reffmt{pairsep} \reffmt{ref-block2}
% \item \reffmt{ref-block1} \reffmt{listsep} \reffmt{ref-block2}
%   \reffmt{listsep} \reffmt{ref-block3} \dots{} \par \qquad
%   \dots{}\reffmt{ref-blockN-1} \reffmt{lastsep} \reffmt{ref-blockN}
% \item \reffmt{ref-block1} \reffmt{rangesep} \reffmt{ref-blockN}
% \end{refformat}
%
% To complete a ``type-block'', a \texttt{type-group} only needs to be
% accompanied by the ``type name'':
%
% \begin{refformat}
% \item \reffmt{type-block} \(\equiv\)
% \item \reffmt{type-name} \reffmt{namesep} \reffmt{type-group}
% \end{refformat}
%
% The \texttt{type-name} is determined not by one single reference format
% option but by the appropriate one among the \opt{[Nn]ame-} options according
% to the composition of \texttt{type-group} and the general options.  The
% reference format name options are eight in total: \opt{Name-sg},
% \opt{name-sg}, \opt{Name-pl}, \opt{name-pl}, \opt{Name-sg-ab},
% \opt{name-sg-ab}, \opt{Name-pl-ab}, and \opt{name-pl-ab}.  The initial
% uppercase ``\texttt{N}'' signals the capitalized form of the type name.  The
% \texttt{-sg} suffix stands for singular, while \texttt{-pl} for plural.  The
% \texttt{-ab} is appended to the abbreviated type name form options.  When
% setting up a type, not necessarily all forms need to be provided.
% \pkg{zref-clever} will always use the non-abbreviated form as a fallback to
% the abbreviated one, if the later is not available.  Hence, if a reference
% type is not intended to be used with abbreviated names (the most common
% case), only the basic four forms are needed.  Besides that, if you are using
% the \opt{cap} option, only the capitalized forms will ever be required by
% \cs{zcref}, so you can get away setting only \opt{Name-sg} and
% \opt{Name-pl}.  You should not do the contrary though, and provide only the
% non-capitalized forms because, even if you are using the \opt{nocap} option,
% the capitalized forms will be still required for \opt{capfirst} and \opt{S}
% options to work.  Whatever the case may be, you need not worry too much
% about being remiss in this area: if \cs{zcref} does lack a name form in any
% given reference, it will let you know with a compilation warning (and will
% typeset the usual missing reference sign: ``\textbf{??}'').
%
% A complete reference typeset by \cs{zcref} may be comprised of multiple
% \texttt{type-block}s, in which case the ``type-block-group'' can also be
% made of single type blocks, ``type block pairs'' or ``type block lists'',
% where each is respectively built as:
%
% \begin{refformat}
% \item \reffmt{type-block-group} is one of:
% \item \reffmt{type-block}
% \item \reffmt{type-block1} \reffmt{tpairsep} \reffmt{type-block2}
% \item \reffmt{type-block1} \reffmt{tlistsep} \reffmt{type-block2}
%   \reffmt{tlistsep} \reffmt{type-block3} \dots{} \par \qquad \dots{}
%   \reffmt{type-blockN-1} \reffmt{tlastsep} \reffmt{type-blockN}
% \end{refformat}
%
% Finally, since \cs{zcref} can also receive an optional \opt{note}, its full
% typeset output is built as:
%
% \begin{refformat}
% \item A complete \reffmt{\cs{zcref}} reference:
% \item \reffmt{type-block-group} \reffmt{notesep} \reffmt{note}
% \end{refformat}
%
% Reference format options can yet be divided in two general categories: i)
% ``string'' options, the ones which we have seen thus far, as ``building
% blocks'' of the reference; and ii) ``font'' options, which control font
% attributes of parts of the reference, namely \opt{namefont} and
% \opt{reffont}.  These options set the font, respectively, for the
% \texttt{type-name} and for \texttt{ref} (to set the font for the whole
% reference, see the \opt{font} option in \zcref{sec:options}).  ``String''
% options is not really a strict denomination for the first category, but this
% set of options is intended exclusively for typesetting material: things you
% expect to see in the output of your references.  The ``font'' options, on
% the other hand, are intended exclusively for commands that only change font
% attributes: style, family, shape, weight, size, color, etc.  In either case,
% anything other than their intended uses is not supported.
%
% Finally, a comment about the internal ``fallback'' reference format values
% mentioned above.  These ``last resort'' option values are required by
% \pkg{zref-clever} for a clear particular case: if the user loads either
% \pkg{babel} or \pkg{polyglossia}, or explicitly sets a language, with a
% language that \pkg{zref-clever} does not know and has no dictionary for, it
% cannot guess what language that is, and thus has to provide some reasonable
% ``language agnostic'' default, at least for the options for which this makes
% sense (all the ``string'' options, except for the \texttt{[Nn]ame-} ones).
% Users do not need to have access to this scope, since they know the language
% of their document, or know the values they want for those options, and can
% set them as general options, type-specific options, or language options
% through the user interface provided for the purpose.  But the ``fallback''
% options are documented here so that you can recognize when you are getting
% these values and change them appropriately as desired.  Though hopefully
% reasonable, they may not be what you want.  The ``fallback'' option values
% are the following:
%
% \begin{verbatim}
%     tpairsep  = {, } ,
%     tlistsep  = {, } ,
%     tlastsep  = {, } ,
%     notesep   = { } ,
%     namesep   = {\nobreakspace} ,
%     pairsep   = {, } ,
%     listsep   = {, } ,
%     lastsep   = {, } ,
%     rangesep  = {\textendash} ,
%     refpre    = {} ,
%     refpos    = {} ,
% \end{verbatim}
%
%
% \section{Internationalization}
% \zlabel{sec:internationalization}
%
% \pkg{zref-clever} provides internationalization facilities and integrates
% with \pkg{babel} and \pkg{polyglossia} to adapt to the languages in use by
% either of these language packages, or to a language specified directly by
% the user.  This is primarily relevant for reference format options,
% particularly reference type \emph{names} (though not only, since most
% reference format options can have language-specific values, or
% ``translations'', see \zcref{sec:reference-format}).  But other features of
% the package also cater for language specific needs.
%
% As far as language selection is concerned, if the language is declared and
% \pkg{zref-clever} has a built-in ``dictionary'' for it, most use cases will
% likely be covered by the \opt{lang} option (see \zcref{sec:options}), and
% its values \texttt{main} and \texttt{current}.  When the \opt{lang} option
% is set to \texttt{main} or \texttt{current}, \pkg{zref-check} will use,
% respectively, the \emph{main} or \emph{current} language of the document, as
% defined by \pkg{babel} or \pkg{polyglossia}.\footnote{Technically,
% \pkg{zref-clever} uses \cs{bbl@main@language} and \cs{languagename} for
% \pkg{babel}, and \cs{mainbabelname} and \cs{babelname} for
% \pkg{polyglossia}, which boils down to \pkg{zref-clever} always using
% \emph{\pkg{babel} names} internally, regardless of which language package is
% in use.  Indeed, an acquainted user will note that
% \zcref{tab:languages-and-aliases} contains only \pkg{babel} language names.}
% Users can also set \opt{lang} to a specific language directly, in which case
% \pkg{babel} and \pkg{polyglossia} are disregarded.  \pkg{zref-clever}
% provides a number of built-in ``dictionaries'', for the languages listed in
% \zcref{tab:languages-and-aliases}, which also includes the declared aliases
% to those languages.
%
% \pkg{zref-clever}'s ``dictionaries'' are loaded sparingly and lazily.  A
% dictionary for a single language -- that specified by user options in the
% preamble, which by default is the main document language -- is loaded at
% \texttt{begindocument}.  If any other dictionary is needed, it is loaded on
% the fly, if and when required.  Of course, in either case, conditioned on
% availability.  This is done because the presumed common use case for
% \pkg{zref-clever} is to use a single language for cross-references (the
% ``main'' one), even in many multi-language documents scenarios.  Hence, even
% the ``loaded languages'' set, from \pkg{babel} or \pkg{polyglossia}, would
% tend to be an overshoot of the actual needs.  So, \pkg{zref-clever} loads as
% little as possible, but allows for convenient on the fly loading of
% dictionaries if the values are indeed required, without users having to
% worry about it at all.
%
% \begin{table}
%   \centering
%   \begin{tabular}{ll}
%     \toprule
%     Language   & Aliases      \\
%     \midrule
%     english    & american     \\
%                & australian   \\
%                & british      \\
%                & canadian     \\
%                & newzealand   \\
%                & UKenglish    \\
%                & USenglish    \\
%     french     & acadian      \\
%                & canadien     \\
%                & francais     \\
%                & frenchb      \\
%     \bottomrule
%   \end{tabular}
%   \quad
%   \begin{tabular}{ll}
%     \toprule
%     Language   & Aliases      \\
%     \midrule
%     german     & austrian     \\
%                & germanb      \\
%                & ngerman      \\
%                & naustrian    \\
%                & nswissgerman \\
%                & swissgerman  \\
%     portuguese & brazilian    \\
%                & brazil       \\
%                & portuges     \\
%     spanish    &              \\
%                &              \\
%     \bottomrule
%   \end{tabular}
%   \caption{Declared languages and aliases}
%   \zlabel{tab:languages-and-aliases}
% \end{table}
%
%
% But if the built-in dictionaries do not cover your language, or if you'd
% like to adjust some of the default language-specific options, this can be
% done with \cs{zcDeclareLanguage}, \cs{zcDeclareLanguageAlias}, and
% \cs{zcLanguageSetup}.\footnote{Needless to say, if you'd like to contribute
% a dictionary or improve an existing one, that is much welcome at
% \url{https://github.com/gusbrs/zref-clever/issues}.}
%
% \begin{function}{\zcDeclareLanguage}
%   Declare a new language for use with \pkg{zref-clever}.  If \meta{language}
%   has already been declared, just warn.  The \meta{options} argument
%   receives the usual \texttt{key=value} list and recognizes three keys:
%   \opt{declension}, \opt{gender}, and \opt{allcaps}.  \opt{declension}
%   receives a coma separated list of valid declension cases for
%   \meta{language}.  The first element of the list is considered to be the
%   default case, both for the \opt{d} option in \cs{zcref} and for the
%   \opt{case} option in \cs{zcLanguageSetup}.  Similarly, \opt{gender}
%   receives a comma separated list of genders for \meta{language}.  The
%   elements in this list are those which are recognized as valid for the
%   language for both the \opt{g} option in \cs{zcref} and the \opt{gender}
%   option in \cs{zcLanguageSetup}.  There is no default presumed in this
%   case.  Finally, \opt{allcaps} can be used with languages for which nouns
%   must be always capitalized for grammatical reasons.  For a language
%   declared with the \opt{allcaps} option, the \opt{cap} reference option
%   (see \zcref{sec:options}) is disregarded, and \cs{zcref} always uses the
%   capitalized type name forms.  This means that dictionaries for languages
%   with such a trait can be halved in size, and that user customization for
%   them is simplified, only requiring the capitalized name forms.  On the
%   other hand, the non-capitalized \texttt{name-\dots{}} reference format
%   options are rendered no-op for the language in question.
%   \zcref[S]{tab:language-options} presents an overview of the options in
%   effect for the languages declared by \pkg{zref-clever}.
%   \cs{zcDeclareLanguage} is preamble only.
%   \begin{syntax}
%     \cs{zcDeclareLanguage} \oarg{options} \marg{language}
%   \end{syntax}
% \end{function}
%
% \begin{table}
%   \centering
%   \begin{tabular}{l>{\ttfamily}c>{\ttfamily}c>{\ttfamily}c}
%     \toprule
%     Language   & declension & gender & allcaps \\
%     \midrule
%     english    & --         & --     & --      \\
%     french     & --         & f,m    & --      \\
%     german     & N,A,D,G    & f,m,n  & yes     \\
%     portuguese & --         & f,m    & --      \\
%     spanish    & --         & f,m    & --      \\
%     \bottomrule
%   \end{tabular}
%   \caption{Options for declared languages}
%   \zlabel{tab:language-options}
% \end{table}
%
% \begin{function}{\zcDeclareLanguageAlias}
%   Declare \meta{language alias} to be an alias of \meta{aliased language}.
%   \meta{aliased language} must be already known to \pkg{zref-clever}.  Once
%   set, the \meta{language alias} is treated by \pkg{zref-clever} as
%   completely equivalent to the \meta{aliased language} for any language
%   specification by the user. \cs{zcDeclareLanguageAlias} is preamble only.
%   \begin{syntax}
%     \cs{zcDeclareLanguageAlias} \marg{language alias} \marg{aliased language}
%   \end{syntax}
% \end{function}
%
%
% \begin{function}{\zcLanguageSetup}
%   \begin{syntax}
%     \cs{zcLanguageSetup} \marg{language} \marg{options}
%   \end{syntax}
%   Sets language-specific reference format options for \meta{language} (see
%   \zcref{sec:reference-format}), be they type-specific or
%   not. \meta{language} must be already known to \pkg{zref-clever}.  Besides
%   reference format options, \cs{zcLanguageSetup} knows three other keys:
%   \opt{type}, \opt{case}, and \opt{gender}.  The first two work like a
%   ``switch'' affecting the options \emph{following} it.  For example, if
%   \texttt{type=foo} is given in \meta{options} the options following it will
%   be set as type-specific options for reference type \texttt{foo}.
%   Similarly, after \texttt{case=X} (provided \texttt{X} is a valid
%   declension case for \meta{language}), the following
%   \texttt{[Nn]ame-\dots{}} options will set values for the \texttt{X}
%   declension case (other reference format options are not affected by
%   \opt{case}).  Before the first occurrence of either \opt{type} or
%   \opt{case} default values are set.  For \opt{case} this means the default
%   declension case, which is the first element of the list provided to the
%   \opt{declension} option in \cs{zcDeclareLanguage}.  For \opt{type} this
%   means ``default translations'', which are language-specific but not
%   type-specific option values (see \zcref{sec:reference-format}).  An empty
%   valued \texttt{type=} key can also ``unset'' the type.  The \opt{gender}
%   key sets the gender of the current \texttt{type} (provided the value it
%   receives is one of the declared genders for \meta{language}).
%   \cs{zcLanguageSetup} is preamble only.
% \end{function}
%
% A couple of examples to illustrate the syntax of \cs{zcLanguageSetup}:
%
% \begin{verbatim}
%   \zcLanguageSetup{french}{
%     type = section ,
%       gender = f ,
%       Name-sg = Section ,
%       name-sg = section ,
%       Name-pl = Sections ,
%       name-pl = sections ,
%   }
%   \zcLanguageSetup{german}{
%     type = section ,
%       gender = m ,
%       case = N ,
%         Name-sg = Abschnitt ,
%         Name-pl = Abschnitte ,
%       case = A ,
%         Name-sg = Abschnitt ,
%         Name-pl = Abschnitte ,
%       case = D ,
%         Name-sg = Abschnitt ,
%         Name-pl = Abschnitten ,
%       case = G ,
%         Name-sg = Abschnitts ,
%         Name-pl = Abschnitte ,
%   }
% \end{verbatim}
%
%
% As already noted, \pkg{zref-clever} has some support for languages with
% declension.  This means mainly the declension of \emph{nouns}, which is used
% for the reference type names.  But some tools are also provided to support
% the user in getting better results for the text surrounding a reference,
% particularly for numbered and gendered articles, even if those don't have
% their typeset output automated.
%
% For reference type names, the declension cases for each language must be
% declared with \cs{zcDeclareLanguage}, and the name reference format options
% must be provided for each case, which is done for built-in dictionaries of
% languages which have noun declension, and can be done by the user with
% \cs{zcLanguageSetup}, as we've seen.  \pkg{zref-clever} does not try to
% guess or infer the case though, you must tell it to \cs{zcref}.  And this is
% done by means of the \opt{d} option (see \zcref{sec:options}).  So you may
% write something like ``\texttt{nach den
% \cs{zcref}[d=D]\{sec:section-1,sec:section-2\}}'' to get ``nach den
% Abschnitten 1 und 2''.  Or ``\texttt{trotz des
% \cs{zcref}[d=G]\{eq:theorem-1\}}'' to get ``trotz des Theorems 1''.
%
% Regarding the text surrounding the reference -- the inflected article, the
% passing preposition, etc.\ --, the issue is more delicate.
% \pkg{zref-clever} cannot and intends not to typeset those for you.  But,
% depending on the language, it is true that the kind of automation provided
% by \pkg{zref-clever} may betray your best efforts to get a proper
% surrounding text.  Multiple labels passed to \cs{zcref} may result in
% singular type names, either because the labels are of different types, or
% because they got compressed into a single reference.  References comprised
% of multiple type blocks may have each a name with a different gender.  Or,
% worse, \opt{tpairsep}, \opt{tpairsep}, and \opt{tlastsep} may not provide a
% general enough way to separate different type blocks in your language
% altogether.  You may change something in your document that causes a label
% to change its type, and hence the gender of the type name.  A page reference
% to a couple of floats which were by chance on the same page and all of a
% sudden no longer are.  And so on.
%
% In this area, the approach taken by \pkg{zref-clever} is to identify some
% typical situations in which your attention may be required in reviewing the
% surrounding text, and signal it at compilation time.  Just like bad boxes,
% for example.  This feature can be enabled by the \opt{nudge} option (which
% is opt-in, see \zcref{sec:options}).  There are three ``nudges'' available
% for this purpose which trigger messages at different events:
% \opt{multitype}, \opt{comptosing}, and \opt{gender}.  \opt{multitype} nudges
% when a reference is comprised of multiple type blocks.  \opt{comptosing}
% when multiple labels of the same type block have been compressed into a
% single one and, hence, the type name used is singular.  Finally,
% \opt{gender} nudges when there is a mismatch between the gender specified in
% \cs{zcref} with the \opt{g} option and the gender of the type name, as
% stored in the dictionary or language settings, for each type block.  Which
% nudges to use is configurable with the option \opt{nudgeif}.  And, if you're
% sure of the results for a particular \cs{zcref} call, you can always silence
% the nudges locally with the \opt{nonudge} option.
%
% The main reason to watch for multiple type references with the
% \opt{multitype} nudge is that bundling together automatically a list of type
% blocks is less smooth an operation than it is for a single reference type.
% While it arguably works reasonably well for English, even there it is not
% always flawless, and depending on the language, results may range from
% ``lack of style'' to outright wrong.  A typical case would be of that of a
% language with inflected articles and a reference with multiple types of
% different genders or numbers.  For example, in French, with a standard
% ``\texttt{au \cs{zcref}\{cha:chapter-3, sec:section-3.1\}}'' we get ``au
% chapitre 3 et section 3.1'' which just sounds like a shrieking chalkboard.
% So we may be better off writing instead ``\texttt{au
% \cs{zcref}\{cha:chapter-3\} et à la \cs{zcref}\{sec:section-3.1\}}''.  Or
% something else, of course.  But the general point is that, depending on
% circumstances and on the language, the results of automating the grouping of
% multiple reference types, as \pkg{zref-clever} is able to do, may leave
% things to be desired for.  Hence it lets you know when one such case occurs,
% so that you can review it for best results.
%
% The case of the \opt{comptosing} and \opt{gender} nudges is more objective
% in nature, they respectively signal mismatches of number and gender.  When a
% reference is made with \cs{zcref} to a single label we are sure the type
% name will be a singular form.  However, when \cs{zcref} receives multiple
% labels of the same type, the type name will normally be a plural, but not
% necessarily so, since the labels may be compressed into a single one (see
% the \opt{comp} option in \zcref{sec:options}), in which case the singular is
% used.  The compression of multiple labels into a single reference should be
% an exception for default references, but not so for \opt{page} references,
% where it is easy to conceive practical situations where it may occur.
% Suppose, for example, you have two contiguous floats in your document and
% make a page reference to both of them.  Will they end up in the same page or
% not?  Maybe we know what the current state is, but we cannot know what may
% happen as the document keeps being edited.  As a consequence, we don't know
% whether that reference will end up having a plural or a singular type name.
% That given, the logic of the \opt{comptosing} nudge is the following.  If we
% are giving multiple labels to \cs{zcref} we can \emph{presume} a plural type
% name, but we get a nudge in case the compression of the labels results in a
% singular type name form.  If one such compression did happen to one of your
% references you can use a singular article and then tell \cs{zcref} you did
% so with option \opt{sg}.  The effect of the \opt{sg} option is to inhibit
% the nudge when a compression to singular occurs, but to do it instead when
% the compression \emph{ceases} to occur, that is, if we get a plural type
% name again at some point.
%
% The \opt{gender} nudge aims to guard against one particular situation:
% possible changes of a reference's type.  This does not occur by reason of
% any internal behavior of \pkg{zref-clever}, but it may be caused by changes
% in the document.  You may wish to change one \texttt{theorem} into a
% \texttt{proposition} and, if you're writing in French or Portuguese, for
% example, that implies that the reference to it changes gender and the likely
% preceding article will no longer pass to the reference.  The \opt{gender}
% nudge requires that the gender of each type name and of each reference be
% explicitly specified.  For the type names, this is done for the built-in
% dictionaries of languages were this matters, and can be done with
% \cs{zcLanguageSetup} as well.  For the references, that is the purpose of
% the \opt{g} option.  When there is a mismatch between the two for any type
% block, the nudge is triggered.  Of couse, this means that the gender markup
% has to be supplied in the document at each reference.  And considering such
% type changes may not be frequent for you, or considered not particularly
% problematic, you'll have to balance if doing so is worth it.  Still, the
% feature is available, and it's up to you.
%
%
%
% \section{How tos}
%
% This section gathers some usage examples, or ``how tos'', of cases which may
% require some \pkg{zref-clever} setup, and each item is set around a
% cross-reference ``task'' we'd like to perform with \pkg{zref-clever}.
%
%
% \subsection{\cs{newtheorem}}
%
% Since \LaTeX{}'s \cs{newtheorem} allows users to create arbitrary numbered
% environments, with respective arbitrary counters, the most \pkg{zref-clever}
% can do in this regard is to provide some ``typical'' built-in reference
% types to smooth user setup but, in the general case, some user setup may be
% indeed required.  The examples below are equaly valid for \pkg{amsthm}'s
% \cs{newtheorem} since, even it provides features beyond those available in
% the kernel, its syntax and underlying relation with counters is pretty much
% the same.  The same for \pkg{ntheorem}.  For \pkg{thmtools}'
% \cs{declaretheorem} the situation is also similar.  Though some adjustments
% to the examples below may be required, the basic logic is the same.  There
% is no integration with the \opt{Refname}, \opt{refname}, and \opt{label}
% options, which are targeted to the standard reference system, but you don't
% actually need them to get things working conveniently.
%
%
% \subsubsection*{Simple case}
%
% \zctask{Setup up a new theorem environment created with \cs{newtheorem} to
% be referred to with \cs{zcref}.  The theorem environment does not share its
% counter with other theorem environments, and one of \pkg{zref-clever}
% built-in reference types is adequate for my needs.}
%
% Suppose you set a ``Lemma'' environment with:
%
% \begin{zchowto}
%   \newtheorem{lemma}{Lemma}
% \end{zchowto}
%
% Or with:
%
% \begin{zchowto}
%   \newtheorem{lemma}{Lemma}[section]
% \end{zchowto}
%
% In this case, since \pkg{zref-clever} provides a built-in \texttt{lemma}
% type (for supported languages) and presumes the reference type to be the
% same name as the counter, there is no need for setup, and things just work
% out-of-the-box.  So, you can go ahead with:
%
% \begin{zchowto}
%   \documentclass{article}
%   \usepackage{zref-clever}
%   \newtheorem{lemma}{Lemma}[section]
%   \begin{document}
%   \section{Section 1}
%   \begin{lemma}\zlabel{lemma-1}
%     A lemma.
%   \end{lemma}
%   \zcref{lemma-1}
%   \end{document}
% \end{zchowto}
%
% If, however, you had chosen an environment name which did not happen to
% coincide with the built-in reference type, all you'd need to do is instruct
% \pkg{zref-clever} to associate the counter for your environment to the
% desired type with the \opt{countertype} option:
%
% \begin{zchowto}
%   \documentclass{article}
%   \usepackage{zref-clever}
%   \zcsetup{countertype={lem=lemma}}
%   \newtheorem{lem}{Lemma}[section]
%   \begin{document}
%   \section{Section 1}
%   \begin{lem}\zlabel{lemma-1}
%     A lemma.
%   \end{lem}
%   \zcref{lemma-1}
%   \end{document}
% \end{zchowto}
%
%
% \subsubsection*{Shared counter}
%
% \zctask{Setup up two new theorem environments created with \cs{newtheorem}
% to be referred to with \cs{zcref}.  The theorem environments share the same
% counter, and the available \pkg{zref-clever} built-in reference types are
% adequate for my needs.}
%
% In this case, we need to set the \opt{countertype} option in the appropriate
% contexts, so that the labels of each environment get set with the expected
% reference type.  As we've seen (at \zcref{sec:user-interface}), \cs{zcsetup}
% has local effects, so it can be issued inside the respective environments
% for the purpose.  Even better, we can leverage the kernel's new hook
% management system and just set it for all occurrences with
% \cs{AddToHook}\texttt{\{env/\meta{myenv}/begin\}}.
%
% \begin{zchowto}
%   \documentclass{article}
%   \usepackage{zref-clever}
%   \AddToHook{env/mytheorem/begin}{%
%     \zcsetup{countertype={mytheorem=theorem}}}
%   \AddToHook{env/myproposition/begin}{%
%     \zcsetup{countertype={mytheorem=proposition}}}
%   \newtheorem{mytheorem}{Theorem}[section]
%   \newtheorem{myproposition}[mytheorem]{Proposition}
%   \begin{document}
%   \section{Section 1}
%   \begin{mytheorem}\zlabel{theorem-1}
%     A theorem.
%   \end{mytheorem}
%   \begin{myproposition}\zlabel{proposition-1}
%     A proposition.
%   \end{myproposition}
%   \zcref{theorem-1,proposition-1}
%   \end{document}
% \end{zchowto}
%
%
% \subsubsection*{Custom type}
%
% \zctask{Setup up a new theorem environment created with \cs{newtheorem} to
% be referred to with \cs{zcref}.  The theorem environment does not share its
% counter with other theorem environments, but none of \pkg{zref-clever}
% built-in reference types is adequate for my needs.}
%
% In this case, we need to provide \pkg{zref-clever} with settings pertaining
% to the custom reference type we'd like to use.  Unless you need to typeset
% your cross-references in multiple languages, in which case you'd require
% \cs{zcLanguageSetup}, the most convenient way to setup a reference type is
% \cs{zcRefTypeSetup}.  In most cases, what we really need to provide for a
% custom type are the ``type names'' and other reference format options can
% rely on default translations already provided by the package (assuming the
% language is supported).
%
% \begin{zchowto}
%   \documentclass{article}
%   \usepackage{zref-clever}
%   \newtheorem{myconjecture}{Conjecture}[section]
%   \zcRefTypeSetup{myconjecture}{
%     Name-sg = Conjecture ,
%     name-sg = conjecture ,
%     Name-pl = Conjectures ,
%     name-pl = conjectures ,
%   }
%   \begin{document}
%   \section{Section 1}
%   \begin{myconjecture}\zlabel{conjecture-1}
%     A conjecture.
%   \end{myconjecture}
%   \zcref{conjecture-1}
%   \end{document}
% \end{zchowto}
%
%
%
% \subsection{\pkg{enumitem}}
%
% \zctask{Setup a custom enumerate environment created with \pkg{enumitem} to
% be referred to with \cs{zcref}.}
%
% Since the \texttt{enumerate} environment's counters are reset at each
% nesting level, but not with the usual \cs{refstepcounter} machinery, we have
% to inform \pkg{zref-clever} of this resetting behavior with the
% \opt{counterresetby} option.  Also, given the naming of the underlying
% counters is tied with the environment's name and the level's number, we
% cannot really rely on an implicit counter-type association, and have to set
% it explicitly with the \opt{countertype} option.
%
% \begin{zchowto}
%   \documentclass{article}
%   \usepackage{zref-clever}
%   \zcsetup{
%     countertype = {
%       myenumeratei   = item ,
%       myenumerateii  = item ,
%       myenumerateiii = item ,
%       myenumerateiv  = item ,
%     } ,
%     counterresetby = {
%       myenumerateii  = myenumeratei ,
%       myenumerateiii = myenumerateii ,
%       myenumerateiv  = myenumerateiii ,
%     }
%   }
%   \usepackage{enumitem}
%   \newlist{myenumerate}{enumerate}{4}
%   \setlist[myenumerate,1]{label=(\arabic*)}
%   \setlist[myenumerate,2]{label=(\Roman*)}
%   \setlist[myenumerate,3]{label=(\Alph*)}
%   \setlist[myenumerate,4]{label=(\roman*)}
%   \begin{document}
%   \begin{myenumerate}
%   \item An item.\zlabel{item-1}
%     \begin{myenumerate}
%     \item An item.\zlabel{item-2}
%       \begin{myenumerate}
%       \item An item.\zlabel{item-3}
%         \begin{myenumerate}
%         \item An item.\zlabel{item-4}
%         \end{myenumerate}
%       \end{myenumerate}
%     \end{myenumerate}
%   \end{myenumerate}
%   \zcref{item-1,item-2,item-3,item-4}
%   \end{document}
% \end{zchowto}
%
%
% \subsection{\pkg{zref-xr}}
%
% \zctask{Make references to labels set in an external document with
% \cs{zcref}.}
%
% \pkg{zref} itself offers this functionality with module \pkg{zref-xr}, and
% \pkg{zref-clever} is prepared to make use of it.  Just a couple of details
% have to be taken care of, for it to work as intended: i) \pkg{zref-clever}
% must be loaded in both the main document and the external document, so that
% the imported labels also contain the properties required by
% \pkg{zref-clever}; ii) since \cs{zexternaldocument} defines any properties
% it finds in the labels from the external document when it imports them, it
% must be called after \pkg{zref-clever} is loaded, otherwise the later will
% find its own internal properties already defined when it does get loaded,
% and will justifiably complain.  Note as well that the starred version of
% \cs{zexternaldocument*}, which imports the standard labels from the external
% document, is not sufficient for \pkg{zref-clever}, since the imported labels
% will not contain all the required properties.
%
% Assuming here \file{documentA.tex} as the main file and \file{documentB.tex}
% as the external one, and also assuming we just want to refer in
% ``\texttt{A}'' to the labels from ``\texttt{B}'', and not the contrary, a
% minimum setup would be the following.
%
% \bigskip{}
%
% \file{documentA.tex}:
%
% \begin{zchowto}
%   \documentclass{article}
%   \usepackage{zref-clever}
%   \usepackage{zref-xr}
%   \zexternaldocument[B-]{documentB}
%   \usepackage{hyperref}
%   \begin{document}
%   \section{Section A1}
%   \zlabel{sec:section-a1}
%   \zcref{sec:section-a1,B-sec:section-b1}
%   \end{document}
% \end{zchowto}
%
% \file{documentB.tex}:
%
% \begin{zchowto}
%   \documentclass{article}
%   \usepackage{zref-clever}
%   \usepackage{hyperref}
%   \begin{document}
%   \section{Section B1}
%   \zlabel{sec:section-b1}
%   \end{document}
% \end{zchowto}
%
%
%
% \section{Limitations}
%
% Being based on \pkg{zref} entails one quite sizable advantage for
% \pkg{zref-clever}: the extensible referencing system of the former allows
% \pkg{zref-clever} to store and retrieve the information it needs to work
% without having to redefine some core \LaTeX{} commands.  This alone makes
% for reduced compatibility problems and less load order issues than the
% average package in this functionality area.  On the other hand, being based
% on \pkg{zref} also does impair the supported scope of \pkg{zref-clever}.
% Not because of any particular limitation of either, but because any class or
% package which implements some special handling for reference labels
% universally does so aiming at the standard referencing system, and whether
% specific support for \pkg{zref} is included, or whether things work by
% spillover of the particular technique employed, is not guaranteed.
%
% The limitation here is less one of \pkg{zref-clever} than that of a
% potencial lack of support for \pkg{zref} itself.  Broadly speaking, what
% \pkg{zref-clever} does is setup \pkg{zref} so that its \cs{zref@newlabel}s
% contain the information we need using \pkg{zref}'s API.  Once the
% \cs{zlabel} is set correctly, there is little in the way of
% \pkg{zref-clever}, it can just extract the label's information, again using
% \pkg{zref}'s API, and do its job.  Therefore, the problems that may arise
% are really in \emph{label setting}.
%
% For \cs{zlabel} to be able to set a label with everything \pkg{zref-clever}
% needs, some conditions must be fulfilled, most of which are pretty much the
% same as that of a regular label, but not only.  In particular, the
% \cs{@currentcounter} is critical for the task of \pkg{zref-clever}: the
% reference type will depend on that and, consequently, sorting and
% compression as well, counter resetting behavior information is also
% retrieved based on it, and so on.  \pkg{zref} provides the
% \pkg{zref-counter} module which stores it in the \texttt{counter} property.
%
% Despite that, as far as I've been experimenting, the following label setting
% requirements seem to be potentially problematic and are not necessarily
% granted for \cs{zlabel}:
%
% \begin{enumerate}
% \item One must be able to call \cs{zlabel} at the appropriate scope/location
%   so as to set the label.  Alternatively, \cs{zlabel} must be called with
%   the desired label by some other means.
% \item When \cs{zlabel} is set, it must see the proper value of
%   \cs{@currentcounter}.
% \end{enumerate}
%
% As for the first, it is not everywhere we technically can set a (z)label.
% On verbatim-like environments it depends on how they are defined and whether
% they provide a proper place or option to do so.  But not only those,
% \pkg{amsmath} environments also handle \cs{label} specially and the same
% work is not done for \cs{zlabel}.
%
% Regarding \cs{@currentcounter}, since the 2020-10-01 kernel release, it is
% set by \cs{refstepcounter} alongside \cs{@currentlabel}.  Hence, as long as
% \cs{refstepcounter} is the tool used for the purpose, \cs{zlabel} will tend
% to have all information within its grasp at label setting time.  But that's
% not always the case.  For this reason \pkg{zref-clever} has the option
% \opt{currentcounter} which at least allows for some viable workarounds when
% the kernel's \cs{@currentcounter} itself cannot be relied upon.  Whether we
% have a proper opening to set it, depends on the case.  Still,
% \cs{refstepcounter} is ubiquitous enough a tool that we can count on
% \cs{@currentcounter} most of the time.
%
% All in all, most things work, but some things don't.  And if the later will
% eventually work depends essentially on whether support for \pkg{zref}
% (\pkg{zref-counter} included) is provided by these packages/features or not.
% In what follows of this section, I list cases of known lack of support.
% Even if, formally, this should probably belong to \pkg{zref} rather than
% here, \pkg{zref-clever} is more of a ``user side'' kind of package than
% \pkg{zref} itself, so hopefully it is not inappropriate to do so.
%
%
%
% \section{Acknowledgments}
%
% TODO Include: \pkg{cleveref}, \pkg{zref}, Ulrike Fischer, Frank Mittelbach,
% Phelype Oleinik, Jonathan P.\ Spratte, aka `Skillmon', Enrico Gregorio, aka
% `egreg', Steven B.\ Segletes, \texttt{@samcarter}, Alan Munn, Florent
% Rougon, David Carlisle.
%
%
%
%
%
% \section{Change history}
%
%
% A change log with relevant changes for each version, eventual upgrade
% instructions, and upcoming changes, is maintained in the package's
% repository, at
% \url{https://github.com/gusbrs/zref-clever/blob/main/CHANGELOG.md}.
%
%
%
%
% \end{documentation}
%
%
% \begin{implementation}
%
% \section{Initial setup}
%
% Start the \pkg{DocStrip} guards.
%    \begin{macrocode}
%<*package>
%    \end{macrocode}
%
% Identify the internal prefix (\LaTeX3 \pkg{DocStrip} convention).
%    \begin{macrocode}
%<@@=zrefclever>
%    \end{macrocode}
%
% Taking a stance on backward compatibility of the package.  During initial
% development, we have used freely recent features of the kernel (albeit
% refraining from \pkg{l3candidates}, even though I'd have loved to have used
% \cs{bool_case_true:}\dots{}).  We presume \pkg{xparse} (which made to the
% kernel in the 2020-10-01 release), and \pkg{expl3} as well (which made to
% the kernel in the 2020-02-02 release).  We also just use UTF-8 for the
% dictionaries (which became the default input encoding in the 2018-04-01
% release).  Finally, a couple of changes came with the 2021-11-15 kernel
% release, which are important here.  First, a fix was made to the new hook
% management system (\pkg{ltcmdhooks}), with implications to the hook we add
% to \cs{appendix} (see \url{https://tex.stackexchange.com/q/617905} and
% \url{https://github.com/latex3/latex2e/pull/699}, thanks Phelype Oleinik).
% Second, the support for \cs{@currentcounter} has been improved, including
% \cs{footnote} and \pkg{amsmath} (see
% \url{https://github.com/latex3/latex2e/issues/687}, thanks Frank Mittelbach
% and Ulrike Fischer).  Hence, since we would not be able to go much backwards
% without special handling anyway, we make the cut at the 2021-11-15 kernel
% release.
%
% TODO Bump this to 2021-11-15 when the release comes.
%
%    \begin{macrocode}
\providecommand\IfFormatAtLeastTF{\@ifl@t@r\fmtversion}
\IfFormatAtLeastTF{2021-06-01}
  {}
  {%
    \PackageError{zref-clever}{LaTeX kernel too old}
      {%
        'zref-clever' requires a LaTeX kernel newer than 2021-06-01.%
        \MessageBreak Loading will abort!%
      }%
    \endinput
  }%
%    \end{macrocode}
%
%
% Identify the package.
%    \begin{macrocode}
\ProvidesExplPackage {zref-clever} {2021-09-29} {0.1.0-alpha}
  {Clever LaTeX cross-references based on zref}
%    \end{macrocode}
%
%
% \section{Dependencies}
%
% Required packages.  Besides these, \pkg{zref-hyperref} may also be required
% depending on the presence of \pkg{hyperref} itself and on the \opt{hyperref}
% option.
%
%    \begin{macrocode}
\RequirePackage { zref-base }
\RequirePackage { zref-user }
\RequirePackage { zref-abspage }
\RequirePackage { l3keys2e }
\RequirePackage { ifdraft }
%    \end{macrocode}
%
%
% \section{\pkg{zref} setup}
%
% For the purposes of the package, we need to store some information with the
% labels, some of it standard, some of it not so much.  So, we have to setup
% \pkg{zref} to do so.
%
% Some basic properties are handled by \pkg{zref} itself, or some of its
% modules.  The \texttt{default} and \texttt{page} properties are provided by
% \pkg{zref-base}, while \pkg{zref-abspage} provides the \texttt{abspage}
% property which gives us a safe and easy way to sort labels for page
% references.
%
% The \texttt{counter} property, in most cases, will be just the kernel's
% \cs{@currentcounter}, set by \cs{refstepcounter}.  However, not everywhere
% is it assured that \cs{@currentcounter} gets updated as it should, so we
% need to have some means to manually tell \pkg{zref-clever} what the current
% counter actually is.  This is done with the \opt{currentcounter} option, and
% stored in \cs{l_@@_current_counter_tl}, whose default is
% \cs{@currentcounter}.
%
%    \begin{macrocode}
\zref@newprop { zc@counter } { \l_@@_current_counter_tl }
\zref@addprop \ZREF@mainlist { zc@counter }
%    \end{macrocode}
%
% The reference itself, stored by \pkg{zref-base} in the \texttt{default}
% property, is somewhat a disputed real estate.  In particular, the use of
% \cs{labelformat} (previously from \pkg{varioref}, now in the kernel) will
% include there the reference ``prefix'' and complicate the job we are trying
% to do here.  Hence, we isolate \cs{the}\meta{counter} and store it ``clean''
% in \texttt{zc@thecnt} for reserved use.  Since \cs{@currentlabel}, which
% populates the \texttt{default} property, is \emph{more reliable} than
% \cs{@currentcounter}, \texttt{zc@thecnt} is meant to be kept as an
% \emph{option} (\opt{ref} option), in case there's need to use
% \pkg{zref-clever} together with \cs{labelformat}.  Based on the definition
% of \cs{@currentlabel} done inside \cs{refstepcounter} in `texdoc source2e',
% section `ltxref.dtx'.  We just drop the \cs{p@...} prefix.
%
%    \begin{macrocode}
\zref@newprop { zc@thecnt }
  {
    \cs_if_exist:cTF { c@ \l_@@_current_counter_tl }
      { \use:c { the \l_@@_current_counter_tl } }
      {
        \cs_if_exist:cT { c@ \@currentcounter }
          { \use:c { the \@currentcounter } }
      }
  }
\zref@addprop \ZREF@mainlist { zc@thecnt }
%    \end{macrocode}
%
%
% Much of the work of \pkg{zref-clever} relies on the association between a
% label's ``counter'' and its ``type'' (see the User manual section on
% ``Reference types'').  Superficially examined, one might think this relation
% could just be stored in a global property list, rather than in the label
% itself.  However, there are cases in which we want to distinguish different
% types for the same counter, depending on the document context.  Hence, we
% need to store the ``type'' of the ``counter'' for each ``label''.  In
% setting this, the presumption is that the label's type has the same name as
% its counter, unless it is specified otherwise by the \opt{countertype}
% option, as stored in \cs{l_@@_counter_type_prop}.
%
%    \begin{macrocode}
\zref@newprop { zc@type }
  {
    \exp_args:NNe \prop_if_in:NnTF \l_@@_counter_type_prop
      \l_@@_current_counter_tl
      {
        \exp_args:NNe \prop_item:Nn \l_@@_counter_type_prop
          { \l_@@_current_counter_tl }
      }
      { \l_@@_current_counter_tl }
  }
\zref@addprop \ZREF@mainlist { zc@type }
%    \end{macrocode}
%
%
% Since the \texttt{default}, \texttt{zc@thecnt}, and \texttt{page} properties
% store the ``\emph{printed} representation'' of their respective counters,
% for sorting and compressing purposes, we are also interested in their
% numeric values.  So we store them in \texttt{zc@cntval} and
% \texttt{zc@pgval}.  For this, we use \cs{c@}\meta{counter}, which contains
% the counter's numerical value (see `texdoc source2e', section
% `ltcounts.dtx').
%    \begin{macrocode}
\zref@newprop { zc@cntval } [0]
  {
    \cs_if_exist:cTF { c@ \l_@@_current_counter_tl }
      { \int_use:c { c@ \l_@@_current_counter_tl } }
      {
        \cs_if_exist:cT { c@ \@currentcounter }
          { \int_use:c { c@ \@currentcounter } }
      }
  }
\zref@addprop \ZREF@mainlist { zc@cntval }
\zref@newprop* { zc@pgval } [0] { \int_use:c { c@page } }
\zref@addprop \ZREF@mainlist { zc@pgval }
%    \end{macrocode}
%
%
% However, since many counters (may) get reset along the document, we require
% more than just their numeric values.  We need to know the reset chain of a
% given counter, in order to sort and compress a group of references.  Also
% here, the ``printed representation'' is not enough, not only because it is
% easier to work with the numeric values but, given we occasionally group
% multiple counters within a single type, sorting this group requires to know
% the actual counter reset chain (the counters' names and values).  Indeed,
% the set of counters grouped into a single type cannot be arbitrary: all of
% them must belong to the same reset chain, and must be nested within each
% other (they cannot even just share the same parent).
%
% Furthermore, even if it is true that most of the definitions of counters,
% and hence of their reset behavior, is likely to be defined in the preamble,
% this is not necessarily true.  Users can create counters, newtheorems
% mid-document, and alter their reset behavior along the way.  Was that not
% the case, we could just store the desired information at
% \texttt{begindocument} in a variable and retrieve it when needed.  But since
% it is, we need to store the information with the label, with the values as
% current when the label is set.
%
% Though counters can be reset at any time, and in different ways at that, the
% most important use case is the automatic resetting of counters when some
% other counter is stepped, as performed by the standard mechanisms of the
% kernel (optional argument of \cs{newcounter}, \cs{@addtoreset},
% \cs{counterwithin}, and related infrastructure).  The canonical optional
% argument of \cs{newcounter} establishes that the counter being created (the
% mandatory argument) gets reset every time the ``enclosing counter'' gets
% stepped (this is called in the usual sources ``within-counter'', ``old
% counter'', ``supercounter'', ``parent counter'' etc.).  This information is
% a little trickier to get.  For starters, the counters which may reset the
% current counter are not retrievable from the counter itself, because this
% information is stored with the counter that does the resetting, not with the
% one that gets reset (the list is stored in \cs{cl@}\meta{counter} with
% format
% \texttt{\cs{@elt}\{countera\}\cs{@elt}\{counterb\}\cs{@elt}\{counterc\}},
% see section `ltcounts.dtx' in `source2e').  Besides, there may be a chain of
% resetting counters, which must be taken into account: if `counterC' gets
% reset by `counterB', and `counterB' gets reset by `counterA', stepping the
% latter affects all three of them.
%
% The procedure below examines a set of counters, those included in
% \cs{l_@@_counter_resetters_seq}, and for each of them retrieves the set of
% counters it resets, as stored in \cs{cl@}\meta{counter}, looking for the
% counter for which we are trying to set a label
% (\cs{\l_@@_current_counter_tl}, by default \cs{@currentcounter}, passed as
% an argument to the functions).  There is one relevant caveat to this
% procedure: \cs{l_@@_counter_resetters_seq} is populated by hand with the
% ``usual suspects'', there is no way (that I know of) to ensure it is
% exhaustive.  However, it is not that difficult to create a reasonable
% ``usual suspects'' list which, of course, should include the counters for
% the sectioning commands to start with, and it is easy to add more counters
% to this list if needed, with the option \opt{counterresetters}.
% Unfortunately, not all counters are created alike, or reset alike.  Some
% counters, even some kernel ones, get reset by other mechanisms (notably, the
% \texttt{enumerate} environment counters do not use the regular counter
% machinery for resetting on each level, but are nested nevertheless by other
% means).  Therefore, inspecting \cs{cl@}\meta{counter} cannot possibly fully
% account for all of the automatic counter resetting which takes place in the
% document.  And there's also no other ``general rule'' we could grab on for
% this, as far as I know.  So we provide a way to manually tell
% \pkg{zref-clever} of these cases, by means of the \opt{counterresetby}
% option, whose information is stored in \cs{l_@@_counter_resetby_prop}.  This
% manual specification has precedence over the search through
% \cs{l_@@_counter_resetters_seq}, and should be handled with care, since
% there is no possible verification mechanism for this.
%
%
% \begin{macro}[EXP]{\@@_get_enclosing_counters_value:n}
%   Recursively generate a \emph{sequence} of ``enclosing counters'' values,
%   for a given \meta{counter} and leave it in the input stream.  These
%   functions must be expandable, since they get called from \cs{zref@newprop}
%   and are the ones responsible for generating the desired information when
%   the label is being set.  Note that the order in which we are getting this
%   information is reversed, since we are navigating the counter reset chain
%   bottom-up.  But it is very hard to do otherwise here where we need
%   expandable functions, and easy to handle at the reading side.
%     \begin{syntax}
%       \cs{@@_get_enclosing_counters_value:n} \Arg{counter}
%     \end{syntax}
%    \begin{macrocode}
\cs_new:Npn \@@_get_enclosing_counters_value:n #1
  {
    \cs_if_exist:cT { c@ \@@_counter_reset_by:n {#1} }
      {
        { \int_use:c { c@ \@@_counter_reset_by:n {#1} } }
        \@@_get_enclosing_counters_value:e
          { \@@_counter_reset_by:n {#1} }
      }
  }
%    \end{macrocode}
%
% Both \texttt{e} and \texttt{f} expansions work for this particular recursive
% call.  I'll stay with the \texttt{e} variant, since conceptually it is what
% I want (\texttt{x} itself is not expandable), and this package is anyway not
% compatible with older kernels for which the performance penalty of the
% \texttt{e} expansion would ensue (see also
% \url{https://tex.stackexchange.com/q/611370/#comment1529282_611385}, thanks
% Enrico Gregorio, aka `egreg').
%    \begin{macrocode}
\cs_generate_variant:Nn \@@_get_enclosing_counters_value:n { e }
%    \end{macrocode}
% \end{macro}
%
%
% \begin{macro}[EXP]{\@@_counter_reset_by:n}
%   Auxiliary function for \cs{@@_get_enclosing_counters_value:n}, and useful
%   on its own standing.  It is broken in parts to be able to use the
%   expandable mapping functions.  \cs{@@_counter_reset_by:n} leaves in the
%   stream the ``enclosing counter'' which resets \meta{counter}.
%   \begin{syntax}
%     \cs{@@_counter_reset_by:n} \Arg{counter}
%   \end{syntax}
%    \begin{macrocode}
\cs_new:Npn \@@_counter_reset_by:n #1
  {
    \bool_if:nTF
      { \prop_if_in_p:Nn \l_@@_counter_resetby_prop {#1} }
      { \prop_item:Nn  \l_@@_counter_resetby_prop {#1} }
      {
        \seq_map_tokens:Nn \l_@@_counter_resetters_seq
          { \@@_counter_reset_by_aux:nn {#1} }
      }
  }
\cs_new:Npn \@@_counter_reset_by_aux:nn #1#2
  {
    \cs_if_exist:cT { c@ #2 }
      {
        \tl_if_empty:cF { cl@ #2 }
          {
            \tl_map_tokens:cn { cl@ #2 }
              { \@@_counter_reset_by_auxi:nnn {#2} {#1} }
          }
      }
  }
\cs_new:Npn \@@_counter_reset_by_auxi:nnn #1#2#3
  {
    \str_if_eq:nnT {#2} {#3}
      { \tl_map_break:n { \seq_map_break:n {#1} } }
  }
%    \end{macrocode}
% \end{macro}
%
%
% Finally, we create the \texttt{zc@enclval} property, and add it to the
% \texttt{main} property list.
%    \begin{macrocode}
\zref@newprop { zc@enclval }
  {
    \@@_get_enclosing_counters_value:e
      \l_@@_current_counter_tl
  }
\zref@addprop \ZREF@mainlist { zc@enclval }
%    \end{macrocode}
%
%
% Another piece of information we need is the page numbering format being used
% by \cs{thepage}, so that we know when we can (or not) group a set of page
% references in a range.  Unfortunately, \texttt{page} is not a typical
% counter in ways which complicates things.  First, it does commonly get reset
% along the document, not necessarily by the usual counter reset chains, but
% rather with \cs{pagenumbering} or variations thereof.  Second, the format of
% the page number commonly changes in the document (roman, arabic, etc.), not
% necessarily, though usually, together with a reset.  Trying to ``parse''
% \cs{thepage} to retrieve such information is bound to go wrong: we don't
% know, and can't know, what is within that macro, and that's the business of
% the user, or of the documentclass, or of the loaded packages.  The technique
% used by \pkg{cleveref}, which we borrow here, is simple and smart: store
% with the label what \cs{thepage} would return, if the counter \cs{c@page}
% was ``\(1\)''.  That does not allow us to \emph{sort} the references,
% luckily however, we have \texttt{abspage} which solves this problem.  But we
% can decide whether two labels can be compressed into a range or not based on
% this format: if they are identical, we can compress them, otherwise, we
% can't.  To do so, we locally redefine \cs{c@page} to return ``1'', thus
% avoiding any global spillovers of this trick.  Since this operation is not
% expandable we cannot run it directly from the property definition.  Hence,
% we use a shipout hook, and set \cs{g_@@_page_format_tl}, which can then be
% retrieved by the starred definition of
% \texttt{\cs{zref@newprop}*\{zc@pgfmt\}}.
%
%    \begin{macrocode}
\tl_new:N \g_@@_page_format_tl
\cs_new_protected:Npx \@@_page_format_aux: { \int_eval:n { 1 } }
\AddToHook { shipout / before }
  {
    \group_begin:
    \cs_set_eq:NN \c@page \@@_page_format_aux:
    \tl_gset:Nx \g_@@_page_format_tl { \thepage }
    \group_end:
  }
\zref@newprop* { zc@pgfmt } { \g_@@_page_format_tl }
\zref@addprop \ZREF@mainlist { zc@pgfmt }
%    \end{macrocode}
%
%
% Still some other properties which we don't need to handle at the data
% provision side, but need to cater for at the retrieval side, are the ones
% from the \pkg{zref-xr} module, which are added to the labels imported from
% external documents, and needed to construct hyperlinks to them and to
% distinguish them from the current document ones at sorting and compressing:
% \texttt{urluse}, \texttt{url} and \texttt{externaldocument}.
%
%
%
% \section{Plumbing}
%
%
% \subsection{Messages}
%
%
%    \begin{macrocode}
\msg_new:nnn { zref-clever } { option-not-type-specific }
  {
    Option~'#1'~is~not~type-specific~\msg_line_context:.~
    Set~it~in~'\iow_char:N\\zcLanguageSetup'~before~first~'type'
    ~switch~or~as~package~option.
  }
\msg_new:nnn { zref-clever } { option-only-type-specific }
  {
    No~type~specified~for~option~'#1'~\msg_line_context:.~
    Set~it~after~'type'~switch.
  }
\msg_new:nnn { zref-clever } { key-requires-value }
  { The~'#1'~key~'#2'~requires~a~value~\msg_line_context:. }
\msg_new:nnn { zref-clever } { language-declared }
  { Language~'#1'~is~already~declared~\msg_line_context:.~Nothing~to~do. }
\msg_new:nnn { zref-clever } { unknown-language-alias }
  {
    Language~'#1'~is~unknown~\msg_line_context:.~Can't~alias~to~it.~
    See~documentation~for~'\iow_char:N\\zcDeclareLanguage'~and~
    '\iow_char:N\\zcDeclareLanguageAlias'.
  }
\msg_new:nnn { zref-clever } { unknown-language-setup }
  {
    Language~'#1'~is~unknown~\msg_line_context:.~Can't~set~it~up.~
    See~documentation~for~'\iow_char:N\\zcDeclareLanguage'~and~
    '\iow_char:N\\zcDeclareLanguageAlias'.
  }
\msg_new:nnn { zref-clever } { unknown-language-opt }
  {
    Language~'#1'~is~unknown~\msg_line_context:.~Using~default.~
    See~documentation~for~'\iow_char:N\\zcDeclareLanguage'~and~
    '\iow_char:N\\zcDeclareLanguageAlias'.
  }
\msg_new:nnn { zref-clever } { unknown-language-decl }
  {
    Can't~set~declension~'#1'~for~unknown~language~'#2'~\msg_line_context:.~
    See~documentation~for~'\iow_char:N\\zcDeclareLanguage'~and~
    '\iow_char:N\\zcDeclareLanguageAlias'.
  }
\msg_new:nnn { zref-clever } { language-no-decl-ref }
  {
    Language~'#1'~has~no~declared~declension~cases~\msg_line_context:.~
    Nothing~to~do~with~option~'d=#2'.
  }
\msg_new:nnn { zref-clever } { language-no-gender }
  {
    Language~'#1'~has~no~declared~gender~\msg_line_context:.~
    Nothing~to~do~with~option~'#2=#3'.
  }
\msg_new:nnn { zref-clever } { language-no-decl-setup }
  {
    Language~'#1'~has~no~declared~declension~cases~\msg_line_context:.~
    Nothing~to~do~with~option~'case=#2'.
  }
\msg_new:nnn { zref-clever } { unknown-decl-case }
  {
    Declension~case~'#1'~unknown~for~language~'#2'~\msg_line_context:.~
    Using~default~declension~case.
  }
\msg_new:nnn { zref-clever } { nudge-multitype }
  {
    Reference~with~multiple~types~\msg_line_context:.~
    You~may~wish~to~separate~them~or~review~language~around~it.
  }
\msg_new:nnn { zref-clever } { nudge-comptosing }
  {
    Multiple~labels~have~been~compressed~into~singular~type~name~
    for~type~'#1'~\msg_line_context:.
  }
\msg_new:nnn { zref-clever } { nudge-plural-when-sg }
  {
    Option~'sg'~signals~that~a~singular~type~name~was~expected~
    \msg_line_context:.~But~type~'#1'~has~plural~type~name.
  }
\msg_new:nnn { zref-clever } { gender-not-declared }
  { Language~'#1'~has~no~'#2'~gender~declared~\msg_line_context:. }
\msg_new:nnn { zref-clever } { nudge-gender-mismatch }
  {
    Gender~mismatch~for~type~'#1'~\msg_line_context:.~
    You've~specified~'g=#2'~but~type~name~is~'#3'~for~language~'#4'.
  }
\msg_new:nnn { zref-clever } { nudge-gender-not-declared-for-type }
  {
    You've~specified~'g=#1'~\msg_line_context:.~
    But~gender~for~type~'#2'~is~not~declared~for~language~'#3'.
  }
\msg_new:nnn { zref-clever } { nudgeif-unknown-value }
  { Unknown~value~'#1'~for~'nudgeif'~option~\msg_line_context:. }
\msg_new:nnn { zref-clever } { option-document-only }
  { Option~'#1'~is~only~available~after~\iow_char:N\\begin\{document\}. }
\msg_new:nnn { zref-clever } { dict-loaded }
  { Loaded~'#1'~dictionary. }
\msg_new:nnn { zref-clever } { dict-not-available }
  { Dictionary~for~'#1'~not~available~\msg_line_context:. }
\msg_new:nnn { zref-clever } { unknown-language-load }
  {
    Language~'#1'~is~unknown~\msg_line_context:.~Unable~to~load~dictionary.~
    See~documentation~for~'\iow_char:N\\zcDeclareLanguage'~and~
    '\iow_char:N\\zcDeclareLanguageAlias'.
  }
\msg_new:nnn { zref-clever } { missing-zref-titleref }
  {
    Option~'ref=title'~requested~\msg_line_context:.~
    But~package~'zref-titleref'~is~not~loaded,~falling-back~to~default~'ref'.
  }
\msg_new:nnn { zref-clever } { hyperref-preamble-only }
  {
    Option~'hyperref'~only~available~in~the~preamble~\msg_line_context:.~
    To~inhibit~hyperlinking~locally,~you~can~use~the~starred~version~of~
    '\iow_char:N\\zcref'.
  }
\msg_new:nnn { zref-clever } { missing-hyperref }
  { Missing~'hyperref'~package.~Setting~'hyperref=false'. }
\msg_new:nnn { zref-clever } { titleref-preamble-only }
  {
    Option~'titleref'~only~available~in~the~preamble~\msg_line_context:.~
    Did~you~mean~'ref=title'?.
  }
\msg_new:nnn { zref-clever } { missing-zref-check }
  {
    Option~'check'~requested~\msg_line_context:.~
    But~package~'zref-check'~is~not~loaded,~can't~run~the~checks.
  }
\msg_new:nnn { zref-clever } { missing-type }
  { Reference~type~undefined~for~label~'#1'~\msg_line_context:. }
\msg_new:nnn { zref-clever } { missing-name }
  { Reference~format~option~'#1'~undefined~for~type~'#2'~\msg_line_context:. }
\msg_new:nnn { zref-clever } { missing-string }
  {
    We~couldn't~find~a~value~for~reference~option~'#1'~\msg_line_context:.~
    But~we~should~have:~throw~a~rock~at~the~maintainer.
  }
\msg_new:nnn { zref-clever } { single-element-range }
  { Range~for~type~'#1'~resulted~in~single~element~\msg_line_context:. }
\msg_new:nnn { zref-clever } { compat-package }
  { Loaded~support~for~'#1'~package. }
\msg_new:nnn { zref-clever } { compat-class }
  { Loaded~support~for~'#1'~documentclass. }
%    \end{macrocode}
%
%
%
% \subsection{Data extraction}
%
%
% \begin{macro}{\@@_def_extract:Nnnn}
%   Extract property \meta{prop} from \meta{label} and sets variable \meta{tl
%   var} with extracted value.  Ensure \cs{zref@extractdefault} is expanded
%   exactly twice, but no further to retrieve the proper value.  In case the
%   property is not found, set \meta{tl var} with \meta{default}.
%   \begin{syntax}
%     \cs{@@_def_extract:Nnnn} \Arg{tl val}
%     ~~\Arg{label} \Arg{prop} \Arg{default}
%   \end{syntax}
%    \begin{macrocode}
\cs_new_protected:Npn \@@_def_extract:Nnnn #1#2#3#4
  {
    \exp_args:NNNo \exp_args:NNo \tl_set:Nn #1
      { \zref@extractdefault {#2} {#3} {#4} }
  }
\cs_generate_variant:Nn \@@_def_extract:Nnnn { NVnn }
%    \end{macrocode}
% \end{macro}
%
% \begin{macro}{\@@_extract_unexp:nnn}
%   Extract property \meta{prop} from \meta{label}.  Ensure that, in the
%   context of an x expansion, \cs{zref@extractdefault} is expanded exactly
%   twice, but no further to retrieve the proper value.  Thus, this is meant
%   to be use in an x expansion context, not in other situations.  In case the
%   property is not found, leave \meta{default} in the stream.
%   \begin{syntax}
%     \cs{@@_extract_unexp:nnn}\Arg{label}\Arg{prop}\Arg{default}
%   \end{syntax}
%    \begin{macrocode}
\cs_new:Npn \@@_extract_unexp:nnn #1#2#3
  {
    \exp_args:NNo \exp_args:No
      \exp_not:n { \zref@extractdefault {#1} {#2} {#3} }
  }
\cs_generate_variant:Nn \@@_extract_unexp:nnn { Vnn , nvn , Vvn }
%    \end{macrocode}
% \end{macro}
%
% \begin{macro}{\@@_extract:nnn}
%   An internal version for \cs{zref@extractdefault}.
%   \begin{syntax}
%     \cs{@@_extract:nnn}\Arg{label}\Arg{prop}\Arg{default}
%   \end{syntax}
%    \begin{macrocode}
\cs_new:Npn \@@_extract:nnn #1#2#3
  { \zref@extractdefault {#1} {#2} {#3} }
%    \end{macrocode}
% \end{macro}
%
%
% \subsection{Reference format}
%
% For a general discussion on the precedence rules for reference format
% options, see Section ``Reference format'' in the User manual.  Internally,
% these precedence rules are handled / enforced in \cs{@@_get_ref_string:nN},
% \cs{@@_get_ref_font:nN}, and \cs{@@_type_name_setup:} which are the basic
% functions to retrieve proper values for reference format settings.  The
% ``fallback'' settings are stored in \cs{g_@@_fallback_dict_prop}.
%
%
% \begin{variable}
%   {
%     \l_@@_setup_type_tl ,
%     \l_@@_dict_language_tl ,
%     \l_@@_dict_decl_case_tl ,
%     \l_@@_dict_declension_seq ,
%     \l_@@_dict_gender_seq ,
%   }
%   Store ``current'' type, language, and declension cases in different places
%   for option and translation handling, notably in
%   \cs{@@_provide_dictionary:n}, \cs{zcRefTypeSetup}, and
%   \cs{zcLanguageSetup}.  But also for translations retrieval, in
%   \cs{@@_get_type_transl:nnnN} and \cs{@@_get_default_transl:nnN}.
%    \begin{macrocode}
\tl_new:N \l_@@_setup_type_tl
\tl_new:N \l_@@_dict_language_tl
\tl_new:N \l_@@_dict_decl_case_tl
\seq_new:N \l_@@_dict_declension_seq
\seq_new:N \l_@@_dict_gender_seq
%    \end{macrocode}
% \end{variable}
%
%
% \begin{variable}
%   {
%     \c_@@_ref_options_necessarily_not_type_specific_seq ,
%     \c_@@_ref_options_possibly_type_specific_seq ,
%     \c_@@_ref_options_type_names_seq ,
%     \c_@@_ref_options_genders_seq ,
%     \c_@@_ref_options_font_seq ,
%     \c_@@_ref_options_typesetup_seq ,
%     \c_@@_ref_options_reference_seq ,
%   }
%     Lists of reference format related options in ``categories''.  Since
%     these options are set in different scopes, and at different places,
%     storing the actual lists in centralized variables makes the job not only
%     easier later on, but also keeps things consistent.
%    \begin{macrocode}
\seq_const_from_clist:Nn
  \c_@@_ref_options_necessarily_not_type_specific_seq
  {
    tpairsep ,
    tlistsep ,
    tlastsep ,
    notesep ,
  }
\seq_const_from_clist:Nn
  \c_@@_ref_options_possibly_type_specific_seq
  {
    namesep ,
    pairsep ,
    listsep ,
    lastsep ,
    rangesep ,
    refpre ,
    refpos ,
  }
%    \end{macrocode}
% Only ``type names'' are ``necessarily type-specific'', which makes them
% somewhat special on the retrieval side of things.  In short, they don't have
% their values queried by \cs{@@_get_ref_string:nN}, but by
% \cs{@@_type_name_setup:}.
%    \begin{macrocode}
\seq_const_from_clist:Nn
  \c_@@_ref_options_type_names_seq
  {
    Name-sg ,
    name-sg ,
    Name-pl ,
    name-pl ,
    Name-sg-ab ,
    name-sg-ab ,
    Name-pl-ab ,
    name-pl-ab ,
  }
\seq_const_from_clist:Nn
  \c_@@_ref_options_genders_seq
  { f , m , n }
%    \end{macrocode}
% \cs{c_@@_ref_options_font_seq} are technically ``possibly type-specific'',
% but are not ``language-specific'', so we separate them.
%    \begin{macrocode}
\seq_const_from_clist:Nn
  \c_@@_ref_options_font_seq
  {
    namefont ,
    reffont ,
  }
%    \end{macrocode}
% And, finally, some combined groups of the above variables, for convenience.
%    \begin{macrocode}
\seq_new:N \c_@@_ref_options_typesetup_seq
\seq_gconcat:NNN \c_@@_ref_options_typesetup_seq
  \c_@@_ref_options_possibly_type_specific_seq
  \c_@@_ref_options_type_names_seq
\seq_gconcat:NNN \c_@@_ref_options_typesetup_seq
  \c_@@_ref_options_typesetup_seq
  \c_@@_ref_options_font_seq
\seq_new:N \c_@@_ref_options_reference_seq
\seq_gconcat:NNN \c_@@_ref_options_reference_seq
  \c_@@_ref_options_necessarily_not_type_specific_seq
  \c_@@_ref_options_possibly_type_specific_seq
\seq_gconcat:NNN \c_@@_ref_options_reference_seq
  \c_@@_ref_options_reference_seq
  \c_@@_ref_options_font_seq
%    \end{macrocode}
% \end{variable}
%
%
%
% \subsection{Languages}
%
% \begin{variable}{\g_@@_languages_prop}
%   Stores the names of known languages and the mapping from ``language name''
%   to ``dictionary name''.  Whether of not a language or alias is known to
%   \pkg{zref-clever} is decided by its presence in this property list.  A
%   ``base language'' (loose concept here, meaning just ``the name we gave for
%   the dictionary in that particular language'') is just like any other one,
%   the only difference is that the ``language name'' happens to be the same
%   as the ``dictionary name'', in other words, it is an ``alias to itself''.
%    \begin{macrocode}
\prop_new:N \g_@@_languages_prop
%    \end{macrocode}
% \end{variable}
%
%
% \begin{macro}[int]{\zcDeclareLanguage}
%   Declare a new language for use with \pkg{zref-clever}.  \meta{language} is
%   taken to be both the ``language name'' and the ``dictionary name''.
%   \oarg{options} receive a \texttt{k=v} set of options, with two valid
%   options.  The first, \opt{declension}, takes the noun declension cases
%   prefixes for \meta{language} as a comma separated list, whose first
%   element is taken to be the default case.  The second, \opt{allcaps},
%   receives no value, and indicates that for \meta{language} all nouns must
%   be capitalized for grammatical reasons, in which case, the \opt{cap}
%   option is disregarded for \meta{language}.  If \meta{language} is already
%   known, just warn.  This implies a particular restriction regarding
%   \oarg{options}, namely that these options, when defined by the package,
%   cannot be redefined by the user.  This is deliberate, otherwise the
%   built-in dictionaries would become much too sensitive to this particular
%   user input, and unnecessarily so.  \cs{zcDeclareLanguage} is preamble
%   only.
%   \begin{syntax}
%     \cs{zcDeclareLanguage} \oarg{options} \marg{language}
%   \end{syntax}
%    \begin{macrocode}
\NewDocumentCommand \zcDeclareLanguage { O { } m }
  {
    \group_begin:
    \tl_if_empty:nF {#2}
      {
        \prop_if_in:NnTF \g_@@_languages_prop {#2}
          { \msg_warning:nnn { zref-clever } { language-declared } {#2} }
          {
            \prop_gput:Nnn \g_@@_languages_prop {#2} {#2}
            \prop_new:c { g_@@_dict_ #2 _prop }
            \tl_set:Nn \l_@@_dict_language_tl {#2}
            \keys_set:nn { zref-clever / declarelang } {#1}
          }
      }
    \group_end:
  }
\@onlypreamble \zcDeclareLanguage
%    \end{macrocode}
% \end{macro}
%
%
% \begin{macro}[int]{\zcDeclareLanguageAlias}
%   Declare \meta{language alias} to be an alias of \meta{aliased language}.
%   \meta{aliased language} must be already known to \pkg{zref-clever}, as
%   stored in \cs{g_@@_languages_prop}.  \cs{zcDeclareLanguageAlias} is
%   preamble only.
%   \begin{syntax}
%     \cs{zcDeclareLanguageAlias} \marg{language alias} \marg{aliased language}
%   \end{syntax}
%    \begin{macrocode}
\NewDocumentCommand \zcDeclareLanguageAlias { m m }
  {
    \tl_if_empty:nF {#1}
      {
        \prop_if_in:NnTF \g_@@_languages_prop {#2}
          {
            \exp_args:NNnx
              \prop_gput:Nnn \g_@@_languages_prop {#1}
                { \prop_item:Nn \g_@@_languages_prop {#2} }
          }
          { \msg_warning:nnn { zref-clever } { unknown-language-alias } {#2} }
      }
  }
\@onlypreamble \zcDeclareLanguageAlias
%    \end{macrocode}
% \end{macro}
%
%
%
%    \begin{macrocode}
\keys_define:nn { zref-clever / declarelang }
  {
    declension .code:n =
      {
        \prop_gput:cnn
          { g_@@_dict_ \l_@@_dict_language_tl _prop }
          { declension } {#1}
      } ,
    declension .value_required:n = true ,
    gender .code:n =
      {
        \prop_gput:cnn
          { g_@@_dict_ \l_@@_dict_language_tl _prop }
          { gender } {#1}
      } ,
    gender .value_required:n = true ,
    allcaps .code:n =
      {
        \prop_gput:cnn
          { g_@@_dict_ \l_@@_dict_language_tl _prop }
          { allcaps } { true }
      } ,
    allcaps .value_forbidden:n = true ,
  }
%    \end{macrocode}
%
%
%
% \begin{macro}{\@@_process_language_options:}
%   Auxiliary function for \cs{@@_zcref:nnn}, responsible for processing
%   options from \cs{zcDeclareLanguage}.  It is necessary to separate them
%   from the reference options machinery because their behavior is language
%   dependent, but the language itself can also be set as an option
%   (\opt{lang}, value stored in \cs{l_@@_ref_language_tl}).  Hence, we must
%   validate these options after the reference options have been set.  It is
%   expected to be called right (or soon) after \cs{keys_set:nn} in
%   \cs{@@_zcref:nnn}, where current values for \cs{l_@@_ref_language_tl} and
%   \cs{l_@@_ref_decl_case_tl} are in place.
%    \begin{macrocode}
\cs_new_protected:Npn \@@_process_language_options:
  {
    \exp_args:NNx \prop_get:NnNTF \g_@@_languages_prop
      { \l_@@_ref_language_tl }
      \l_@@_dict_language_tl
      {
%    \end{macrocode}
% Validate the declension case (\opt{d}) option against the declared cases for
% the reference language.  If the user value for the latter does not match the
% declension cases declared for the former, the function sets an appropriate
% value for \cs{l_@@_ref_decl_case_tl}, either using the default case, or
% clearing the variable, depending on the language setup.  And also issues a
% warning about it.
%    \begin{macrocode}
        \exp_args:NNx \seq_set_from_clist:Nn
          \l_@@_dict_declension_seq
          {
            \prop_item:cn
              {
                g_@@_dict_
                \l_@@_dict_language_tl _prop
              }
              { declension }
          }
        \seq_if_empty:NTF \l_@@_dict_declension_seq
          {
            \tl_if_empty:NF \l_@@_ref_decl_case_tl
              {
                \msg_warning:nnxx { zref-clever }
                  { language-no-decl-ref }
                  { \l_@@_ref_language_tl }
                  { \l_@@_ref_decl_case_tl }
                \tl_clear:N \l_@@_ref_decl_case_tl
              }
          }
          {
            \tl_if_empty:NTF \l_@@_ref_decl_case_tl
              {
                \seq_get_left:NN \l_@@_dict_declension_seq
                  \l_@@_ref_decl_case_tl
              }
              {
                \seq_if_in:NVF \l_@@_dict_declension_seq
                  \l_@@_ref_decl_case_tl
                  {
                    \msg_warning:nnxx { zref-clever }
                      { unknown-decl-case }
                      { \l_@@_ref_decl_case_tl }
                      { \l_@@_ref_language_tl }
                    \seq_get_left:NN \l_@@_dict_declension_seq
                      \l_@@_ref_decl_case_tl
                  }
              }
          }
%    \end{macrocode}
% Validate the gender (\opt{g}) option against the declared genders for the
% reference language.  If the user value for the latter does not match the
% genders declared for the former, clear \cs{l_@@_ref_gender_tl} and warn.
%    \begin{macrocode}
        \exp_args:NNx \seq_set_from_clist:Nn
          \l_@@_dict_gender_seq
          {
            \prop_item:cn
              {
                g_@@_dict_
                \l_@@_dict_language_tl _prop
              }
              { gender }
          }
        \seq_if_empty:NTF \l_@@_dict_gender_seq
          {
            \tl_if_empty:NF \l_@@_ref_gender_tl
              {
                \msg_warning:nnxxx { zref-clever }
                  { language-no-gender }
                  { \l_@@_ref_language_tl }
                  { g }
                  { \l_@@_ref_gender_tl }
                \tl_clear:N \l_@@_ref_gender_tl
              }
          }
          {
            \tl_if_empty:NF \l_@@_ref_gender_tl
              {
                \seq_if_in:NVF \l_@@_dict_gender_seq
                  \l_@@_ref_gender_tl
                  {
                    \msg_warning:nnxx { zref-clever }
                      { gender-not-declared }
                      { \l_@@_ref_language_tl }
                      { \l_@@_ref_gender_tl }
                    \tl_clear:N \l_@@_ref_gender_tl
                  }
              }
          }
%    \end{macrocode}
% Ensure \cs{l_@@_capitalize_bool} is set to \texttt{true} when the language
% was declared with \opt{allcaps} option.
%    \begin{macrocode}
        \str_if_eq:eeT
          {
            \prop_item:cn
              {
                g_@@_dict_
                \l_@@_dict_language_tl _prop
              }
              { allcaps }
          }
          { true }
          { \bool_set_true:N \l_@@_capitalize_bool }
      }
      {
%    \end{macrocode}
% If the language itself is not declared, we still have to issue declension
% and gender warnings, if \opt{d} or \opt{g} options were used.
%    \begin{macrocode}
        \tl_if_empty:NF \l_@@_ref_decl_case_tl
          {
            \msg_warning:nnxx { zref-clever } { unknown-language-decl }
              { \l_@@_ref_decl_case_tl }
              { \l_@@_ref_language_tl }
            \tl_clear:N \l_@@_ref_decl_case_tl
          }
        \tl_if_empty:NF \l_@@_ref_gender_tl
          {
            \msg_warning:nnxxx { zref-clever }
              { language-no-gender }
              { \l_@@_ref_language_tl }
              { g }
              { \l_@@_ref_gender_tl }
            \tl_clear:N \l_@@_ref_gender_tl
          }
      }
  }
%    \end{macrocode}
% \end{macro}
%
%
%
% \subsection{Dictionaries}
%
% Contrary to general options and type options, which are always \emph{local},
% ``dictionaries'', ``translations'' or ``language-specific settings'' are
% always \emph{global}.  Hence, the loading of built-in dictionaries, as well
% as settings done with \cs{zcLanguageSetup}, should set the relevant
% variables globally.
%
% The built-in dictionaries and their related infrastructure are designed to
% perform ``on the fly'' loading of dictionaries, ``lazily'' as needed.  Much
% like \pkg{babel} does for languages not declared in the preamble, but used
% in the document.  This offers some convenience, of course, and that's one
% reason to do it.  But it also has the purpose of parsimony, of ``loading the
% least possible''.  My expectation is that for most use cases, users will
% require a single language of the functionality of \pkg{zref-clever} -- the
% main language of the document --, even in multilingual documents.  Hence,
% even the set of \pkg{babel} or \pkg{polyglossia} ``loaded languages'', which
% would be the most tenable set if loading were restricted to the preamble, is
% bound to be an overshoot in typical cases.  Therefore, we load at
% \texttt{begindocument} one single language (see \zcref[ref=title,
% noname]{impl:sec:lang-option}), as specified by the user in the preamble
% with the \opt{lang} option or, failing any specification, the main language
% of the document, which is the default.  Anything else is lazily loaded, on
% the fly, along the document.
%
% This design decision has also implications to the \emph{form} the dictionary
% files assumed.  As far as my somewhat impressionistic sampling goes,
% dictionary or localization files of the most common packages in this area of
% functionality, are usually a set of commands which perform the relevant
% definitions and assignments in the preamble or at \texttt{begindocument}.
% This includes \pkg{translator}, \pkg{translations}, but also \pkg{babel}'s
% \file{.ldf} files, and \pkg{biblatex}'s \file{.lbx} files.  I'm not really
% well acquainted with this machinery, but as far as I grasp, they all rely on
% some variation of \cs{ProvidesFile} and \cs{input}.  And they can be safely
% \cs{input} without generating spurious content, because they rely on being
% loaded before the document has actually started.  As far as I can tell,
% \pkg{babel}'s ``on the fly'' functionality is not based on the \file{.ldf}
% files, but on the \file{.ini} files, and on \cs{babelprovide}.  And the
% \file{.ini} files are not in this form, but actually resemble
% ``configuration files'' of sorts, which means they are read and processed
% somehow else than with just \cs{input}.  So we do the more or less the same
% here.  It seems a reasonable way to ensure we can load dictionaries on the
% fly robustly mid-document, without getting paranoid with the last bit of
% white-space in them, and without introducing any undue content on the stream
% when we cannot afford to do it.  Hence, \pkg{zref-clever}'s built-in
% dictionary files are a set of \emph{key-value options} which are read from
% the file, and fed to \texttt{\cs{keys_set:nn}\{zref-clever/dictionary\}} by
% \cs{@@_provide_dictionary:n}.  And they use the same syntax and options as
% \cs{zcLanguageSetup} does.  The dictionary file itself is read with
% \cs{ExplSyntaxOn} with the usual implications for white-space and catcodes.
%
% \cs{@@_provide_dictionary:n} is only meant to load the built-in
% dictionaries.  For languages declared by the user, or for any settings to a
% known language made with \cs{zcLanguageSetup}, values are populated directly
% to a variable \cs{g_@@_dict_\meta{language}_prop}, created as needed.
% Hence, there is no need to ``load'' anything in this case: definitions and
% assignments made by the user are performed immediately.
%
%
% \subsubsection*{Provide}
%
% \begin{variable}{\g_@@_loaded_dictionaries_seq}
%   Used to keep track of whether a dictionary has already been loaded or not.
%    \begin{macrocode}
\seq_new:N \g_@@_loaded_dictionaries_seq
%    \end{macrocode}
% \end{variable}
%
% \begin{variable}{\l_@@_load_dict_verbose_bool}
%   Controls whether \cs{@@_provide_dictionary:n} fails silently or verbosely
%   in case of unknown languages or dictionaries not found.
%    \begin{macrocode}
\bool_new:N \l_@@_load_dict_verbose_bool
%    \end{macrocode}
% \end{variable}
%
%
% \begin{macro}{\@@_provide_dictionary:n}
%   Load dictionary for known \meta{language} if it is available and if it has
%   not already been loaded.
%   \begin{syntax}
%     \cs{@@_provide_dictionary:n} \Arg{language}
%   \end{syntax}
%    \begin{macrocode}
\cs_new_protected:Npn \@@_provide_dictionary:n #1
  {
    \group_begin:
    \@bsphack
    \prop_get:NnNTF \g_@@_languages_prop {#1}
      \l_@@_dict_language_tl
      {
        \seq_if_in:NVF
          \g_@@_loaded_dictionaries_seq
          \l_@@_dict_language_tl
          {
            \exp_args:Nx \file_get:nnNTF
              { zref-clever- \l_@@_dict_language_tl .dict }
              { \ExplSyntaxOn }
              \l_tmpa_tl
              {
                \tl_clear:N \l_@@_setup_type_tl
                \exp_args:NNx \seq_set_from_clist:Nn
                  \l_@@_dict_declension_seq
                  {
                    \prop_item:cn
                      {
                        g_@@_dict_
                        \l_@@_dict_language_tl _prop
                      }
                      { declension }
                  }
                \seq_if_empty:NTF \l_@@_dict_declension_seq
                  { \tl_clear:N \l_@@_dict_decl_case_tl }
                  {
                    \seq_get_left:NN \l_@@_dict_declension_seq
                      \l_@@_dict_decl_case_tl
                  }
                \exp_args:NNx \seq_set_from_clist:Nn
                  \l_@@_dict_gender_seq
                  {
                    \prop_item:cn
                      {
                        g_@@_dict_
                        \l_@@_dict_language_tl _prop
                      }
                      { gender }
                  }
                \keys_set:nV { zref-clever / dictionary } \l_tmpa_tl
                \seq_gput_right:NV \g_@@_loaded_dictionaries_seq
                  \l_@@_dict_language_tl
                \msg_note:nnx { zref-clever } { dict-loaded }
                  { \l_@@_dict_language_tl }
              }
              {
                \bool_if:NT \l_@@_load_dict_verbose_bool
                  {
                    \msg_warning:nnx { zref-clever } { dict-not-available }
                      { \l_@@_dict_language_tl }
                  }
%    \end{macrocode}
% Even if we don't have the actual dictionary, we register it as ``loaded''.
% At this point, it is a known language, properly declared.  There is no point
% in trying to load it multiple times, because users cannot really provide the
% dictionary files (well, technically they could, but we are working so they
% don't need to, and have better ways to do what they want).  And if the users
% had provided some translations themselves, by means of \cs{zcLanguageSetup},
% everything would be in place, and they could use the \opt{lang} option
% multiple times, and the \texttt{dict-not-available} warning would never go
% away.
%    \begin{macrocode}
                \seq_gput_right:NV \g_@@_loaded_dictionaries_seq
                  \l_@@_dict_language_tl
              }
          }
      }
      {
        \bool_if:NT \l_@@_load_dict_verbose_bool
          { \msg_warning:nnn { zref-clever } { unknown-language-load } {#1} }
      }
    \@esphack
    \group_end:
  }
\cs_generate_variant:Nn \@@_provide_dictionary:n { x }
%    \end{macrocode}
% \end{macro}
%
%
% \begin{macro}{\@@_provide_dictionary_verbose:n}
%   Does the same as \cs{@@_provide_dictionary:n}, but warns if the loading of
%   the dictionary has failed.
%   \begin{syntax}
%     \cs{@@_provide_dictionary_verbose:n} \Arg{language}
%   \end{syntax}
%    \begin{macrocode}
\cs_new_protected:Npn \@@_provide_dictionary_verbose:n #1
  {
    \group_begin:
    \bool_set_true:N \l_@@_load_dict_verbose_bool
    \@@_provide_dictionary:n {#1}
    \group_end:
  }
\cs_generate_variant:Nn \@@_provide_dictionary_verbose:n { x }
%    \end{macrocode}
% \end{macro}
%
%
% \begin{macro}
%   {
%     \@@_provide_dict_type_transl:nn ,
%     \@@_provide_dict_default_transl:nn ,
%   }
%   A couple of auxiliary functions for the of
%   \texttt{{zref-clever/dictionary}} keys set in
%   \cs{@@_provide_dictionary:n}.  They respectively ``provide'' (i.e.  set if
%   it value does not exist, do nothing if it already does) ``type-specific''
%   and ``default'' translations.  Both receive \meta{key} and
%   \meta{translation} as arguments, but \cs{@@_provide_dict_type_transl:nn}
%   relies on the current value of \cs{l_@@_setup_type_tl}, as set by the
%   \texttt{type} key.
%   \begin{syntax}
%     \cs{@@_provide_dict_type_transl:nn} \Arg{key} \Arg{translation}
%     \cs{@@_provide_dict_default_transl:nn} \Arg{key} \Arg{translation}
%   \end{syntax}
%    \begin{macrocode}
\cs_new_protected:Npn \@@_provide_dict_type_transl:nn #1#2
  {
    \exp_args:Nnx \prop_gput_if_new:cnn
      { g_@@_dict_ \l_@@_dict_language_tl _prop }
      { type- \l_@@_setup_type_tl - #1 } {#2}
  }
\cs_new_protected:Npn \@@_provide_dict_default_transl:nn #1#2
  {
    \prop_gput_if_new:cnn
      { g_@@_dict_ \l_@@_dict_language_tl _prop }
      { default- #1 } {#2}
  }
%    \end{macrocode}
% \end{macro}
%
%
% The set of keys for \texttt{{zref-clever/dictionary}}, which is used to
% process the dictionary files in \cs{@@_provide_dictionary:n}.  The no-op
% cases for each category have their messages sent to ``info''.  These
% messages should not occur, as long as the dictionaries are well formed, but
% they're placed there nevertheless, and can be leveraged in regression tests.
%
%    \begin{macrocode}
\keys_define:nn { zref-clever / dictionary }
  {
    type .code:n =
      {
        \tl_if_empty:nTF {#1}
          { \tl_clear:N \l_@@_setup_type_tl }
          { \tl_set:Nn \l_@@_setup_type_tl {#1} }
      } ,
    case .code:n =
      {
        \seq_if_empty:NTF \l_@@_dict_declension_seq
          {
            \msg_info:nnxx { zref-clever } { language-no-decl-setup }
              { \l_@@_dict_language_tl } {#1}
          }
          {
            \seq_if_in:NnTF \l_@@_dict_declension_seq {#1}
              { \tl_set:Nn \l_@@_dict_decl_case_tl {#1} }
              {
                \msg_info:nnxx { zref-clever } { unknown-decl-case }
                  {#1} { \l_@@_dict_language_tl }
                \seq_get_left:NN \l_@@_dict_declension_seq
                  \l_@@_dict_decl_case_tl
              }
          }
      } ,
    case .value_required:n = true ,
    gender .code:n =
      {
        \seq_if_empty:NTF \l_@@_dict_gender_seq
          {
            \msg_info:nnxxx { zref-clever } { language-no-gender }
              { \l_@@_dict_language_tl } { gender } {#1}
          }
          {
            \tl_if_empty:NTF \l_@@_setup_type_tl
              {
                \msg_info:nnn { zref-clever }
                  { option-only-type-specific } { gender }
              }
              {
                \seq_if_in:NnTF \l_@@_dict_gender_seq {#1}
                  { \@@_provide_dict_type_transl:nn { gender } {#1} }
                  {
                    \msg_info:nnxx { zref-clever } { gender-not-declared }
                      { \l_@@_dict_language_tl } {#1}
                  }
              }
          }
      } ,
    gender .value_required:n = true ,
  }
\seq_map_inline:Nn
  \c_@@_ref_options_necessarily_not_type_specific_seq
  {
    \keys_define:nn { zref-clever / dictionary }
      {
        #1 .value_required:n = true ,
        #1 .code:n =
          {
            \tl_if_empty:NTF \l_@@_setup_type_tl
              { \@@_provide_dict_default_transl:nn {#1} {##1} }
              {
                \msg_info:nnn { zref-clever }
                  { option-not-type-specific } {#1}
              }
          } ,
      }
  }
\seq_map_inline:Nn
  \c_@@_ref_options_possibly_type_specific_seq
  {
    \keys_define:nn { zref-clever / dictionary }
      {
        #1 .value_required:n = true ,
        #1 .code:n =
          {
            \tl_if_empty:NTF \l_@@_setup_type_tl
              { \@@_provide_dict_default_transl:nn {#1} {##1} }
              { \@@_provide_dict_type_transl:nn {#1} {##1} }
          } ,
      }
  }
\seq_map_inline:Nn
  \c_@@_ref_options_type_names_seq
  {
    \keys_define:nn { zref-clever / dictionary }
      {
        #1 .value_required:n = true ,
        #1 .code:n =
          {
            \tl_if_empty:NTF \l_@@_setup_type_tl
              {
                \msg_info:nnn { zref-clever }
                  { option-only-type-specific } {#1}
              }
              {
                \tl_if_empty:NTF \l_@@_dict_decl_case_tl
                  { \@@_provide_dict_type_transl:nn {#1} {##1} }
                  {
                    \@@_provide_dict_type_transl:nn
                      { \l_@@_dict_decl_case_tl - #1 } {##1}
                  }
              }
          } ,
      }
  }
%    \end{macrocode}
%
%
%
% \subsubsection*{Fallback}
%
% All ``strings'' queried with \cs{@@_get_ref_string:nN} -- in practice, those
% in either \cs{c_@@_ref_options_necessarily_not_type_specific_seq} or
% \cs{c_@@_ref_options_possibly_type_specific_seq} -- must have their values
% set for ``fallback'', even if to empty ones, since this is what will be
% retrieved in the absence of a proper translation, which will be the case if
% \pkg{babel} or \pkg{polyglossia} is loaded and sets a language which
% \pkg{zref-clever} does not know.  On the other hand, ``type names'' are not
% looked for in ``fallback'', since it is indeed impossible to provide any
% reasonable value for them for a ``specified but unknown language''.  Also
% ``font'' options -- those in \cs{c_@@_ref_options_font_seq}, and queried
% with \cs{@@_get_ref_font:nN} -- do not need to be provided here, since the
% later function sets an empty value if the option is not found.
%
%    \begin{macrocode}
\prop_new:N \g_@@_fallback_dict_prop
\prop_gset_from_keyval:Nn \g_@@_fallback_dict_prop
  {
    tpairsep  = {,~} ,
    tlistsep  = {,~} ,
    tlastsep  = {,~} ,
    notesep   = {~} ,
    namesep   = {\nobreakspace} ,
    pairsep   = {,~} ,
    listsep   = {,~} ,
    lastsep   = {,~} ,
    rangesep  = {\textendash} ,
    refpre    = {} ,
    refpos    = {} ,
  }
%    \end{macrocode}
%
%
%
% \subsubsection*{Get translations}
%
% \begin{macro}{\@@_get_type_transl:nnnNF}
%   Get type-specific translation of \meta{key} for \meta{type} and
%   \meta{language}, and store it in \meta{tl variable} if found.  If not
%   found, leave the \meta{false code} on the stream, in which case the value
%   of \meta{tl variable} should not be relied upon.
%   \begin{syntax}
%     \cs{@@_get_type_transl:nnnNF} \Arg{language} \Arg{type} \Arg{key}
%     ~~\meta{tl variable} \Arg{false code}
%   \end{syntax}
%    \begin{macrocode}
\prg_new_protected_conditional:Npnn
  \@@_get_type_transl:nnnN #1#2#3#4 { F }
  {
    \prop_get:NnNTF \g_@@_languages_prop {#1}
      \l_@@_dict_language_tl
      {
        \prop_get:cnNTF
          { g_@@_dict_ \l_@@_dict_language_tl _prop }
          { type- #2 - #3 } #4
          { \prg_return_true:  }
          { \prg_return_false: }
      }
      { \prg_return_false: }
  }
\prg_generate_conditional_variant:Nnn
  \@@_get_type_transl:nnnN { xxxN , xxnN } { F }
%    \end{macrocode}
% \end{macro}
%
%
% \begin{macro}{\@@_get_default_transl:nnNF}
%   Get default translation of \meta{key} for \meta{language}, and store it in
%   \meta{tl variable} if found.  If not found, leave the \meta{false code} on
%   the stream, in which case the value of \meta{tl variable} should not be
%   relied upon.
%   \begin{syntax}
%     \cs{@@_get_default_transl:nnNF} \Arg{language} \Arg{key}
%     ~~\meta{tl variable} \Arg{false code}
%   \end{syntax}
%    \begin{macrocode}
\prg_new_protected_conditional:Npnn
  \@@_get_default_transl:nnN #1#2#3 { F }
  {
    \prop_get:NnNTF \g_@@_languages_prop {#1}
      \l_@@_dict_language_tl
      {
        \prop_get:cnNTF
          { g_@@_dict_ \l_@@_dict_language_tl _prop }
          { default- #2 } #3
          { \prg_return_true:  }
          { \prg_return_false: }
      }
      { \prg_return_false: }
  }
\prg_generate_conditional_variant:Nnn
  \@@_get_default_transl:nnN { xnN } { F }
%    \end{macrocode}
% \end{macro}
%
%
% \begin{macro}{\@@_get_fallback_transl:nNF}
%   Get fallback translation of \meta{key}, and store it in \meta{tl variable}
%   if found.  If not found, leave the \meta{false code} on the stream, in
%   which case the value of \meta{tl variable} should not be relied upon.
%   \begin{syntax}
%     \cs{@@_get_fallback_transl:nNF} \Arg{key}
%     ~~\meta{tl variable} \Arg{false code}
%   \end{syntax}
%    \begin{macrocode}
% {<key>}<tl var to set>
\prg_new_protected_conditional:Npnn
  \@@_get_fallback_transl:nN #1#2 { F }
  {
    \prop_get:NnNTF \g_@@_fallback_dict_prop
      { #1 } #2
      { \prg_return_true:  }
      { \prg_return_false: }
  }
%    \end{macrocode}
% \end{macro}
%
%
%
% \subsection{Options}
%
%
% \subsubsection*{Auxiliary}
%
%
% \begin{macro}{\@@_prop_put_non_empty:Nnn}
%   If \meta{value} is empty, remove \meta{key} from \meta{property list}.
%   Otherwise, add \meta{key} = \meta{value} to \meta{property list}.
%   \begin{syntax}
%     \cs{@@_prop_put_non_empty:Nnn} \meta{property list} \Arg{key} \Arg{value}
%   \end{syntax}
%    \begin{macrocode}
\cs_new_protected:Npn \@@_prop_put_non_empty:Nnn #1#2#3
  {
    \tl_if_empty:nTF {#3}
      { \prop_remove:Nn #1 {#2} }
      { \prop_put:Nnn #1 {#2} {#3} }
  }
%    \end{macrocode}
% \end{macro}
%
%
% \subsubsection*{\opt{ref} option}
%
% \cs{l_@@_ref_property_tl} stores the property to which the reference is
% being made.  Currently, we restrict \texttt{ref=} to these three (or four)
% alternatives -- \texttt{default}, \texttt{zc@thecnt}, \texttt{page}, and
% \texttt{title} if \pkg{zref-titleref} is loaded --, but there might be a
% case for making this more flexible.  The infrastructure can already handle
% receiving an arbitrary property, as long as one is satisfied with sorting
% and compressing from the current counter.  If more flexibility is granted,
% one thing \emph{must} be handled at this point: the existence of the
% property itself, as far as \pkg{zref} is concerned.  This because
% typesetting relies on the check \cs{zref@ifrefcontainsprop}, which
% \emph{presumes} the property is defined and silently expands the \emph{true}
% branch if it is not (see \url{https://github.com/ho-tex/zref/issues/13},
% thanks Ulrike Fischer).  Therefore, before adding anything to
% \cs{l_@@_ref_property_tl}, check if first here with
% \cs{zref@ifpropundefined}: close it at the door.
%
%
%    \begin{macrocode}
\tl_new:N \l_@@_ref_property_tl
\keys_define:nn { zref-clever / reference }
  {
    ref .choice: ,
    ref / default .code:n =
      { \tl_set:Nn \l_@@_ref_property_tl { default } } ,
    ref / zc@thecnt .code:n =
      { \tl_set:Nn \l_@@_ref_property_tl { zc@thecnt } } ,
    ref / page .code:n =
      { \tl_set:Nn \l_@@_ref_property_tl { page } } ,
    ref / title .code:n =
      {
        \AddToHook { begindocument }
          {
            \@ifpackageloaded { zref-titleref }
              { \tl_set:Nn \l_@@_ref_property_tl { title } }
              {
                \msg_warning:nn { zref-clever } { missing-zref-titleref }
                \tl_set:Nn \l_@@_ref_property_tl { default }
              }
          }
      } ,
    ref .initial:n = default ,
    ref .default:n = default ,
    page .meta:n = { ref = page },
    page .value_forbidden:n = true ,
  }
\AddToHook { begindocument }
  {
    \@ifpackageloaded { zref-titleref }
      {
        \keys_define:nn { zref-clever / reference }
          {
            ref / title .code:n =
              { \tl_set:Nn \l_@@_ref_property_tl { title } }
          }
      }
      {
        \keys_define:nn { zref-clever / reference }
          {
            ref / title .code:n =
              {
                \msg_warning:nn { zref-clever } { missing-zref-titleref }
                \tl_set:Nn \l_@@_ref_property_tl { default }
              }
          }
      }
  }
%    \end{macrocode}
%
%
%
% \subsubsection*{\opt{typeset} option}
%
%    \begin{macrocode}
\bool_new:N \l_@@_typeset_ref_bool
\bool_new:N \l_@@_typeset_name_bool
\keys_define:nn { zref-clever / reference }
  {
    typeset .choice: ,
    typeset / both .code:n =
      {
        \bool_set_true:N \l_@@_typeset_ref_bool
        \bool_set_true:N \l_@@_typeset_name_bool
      } ,
    typeset / ref .code:n =
      {
        \bool_set_true:N \l_@@_typeset_ref_bool
        \bool_set_false:N \l_@@_typeset_name_bool
      } ,
    typeset / name .code:n =
      {
        \bool_set_false:N \l_@@_typeset_ref_bool
        \bool_set_true:N \l_@@_typeset_name_bool
      } ,
    typeset .initial:n = both ,
    typeset .value_required:n = true ,

    noname .meta:n = { typeset = ref },
    noname .value_forbidden:n = true ,
  }
%    \end{macrocode}
%
%
%
% \subsubsection*{\opt{sort} option}
%
%    \begin{macrocode}
\bool_new:N \l_@@_typeset_sort_bool
\keys_define:nn { zref-clever / reference }
  {
    sort .bool_set:N = \l_@@_typeset_sort_bool ,
    sort .initial:n = true ,
    sort .default:n = true ,
    nosort .meta:n = { sort = false },
    nosort .value_forbidden:n = true ,
  }
%    \end{macrocode}
%
%
%
% \subsubsection*{\opt{typesort} option}
%
% \cs{l_@@_typesort_seq} is stored reversed, since the sort priorities are
% computed in the negative range in \cs{@@_sort_default_different_types:nn},
% so that we can implicitly rely on `0' being the ``last value'', and spare
% creating an integer variable using \cs{seq_map_indexed_inline:Nn}.
%
%    \begin{macrocode}
\seq_new:N \l_@@_typesort_seq
\keys_define:nn { zref-clever / reference }
  {
    typesort .code:n =
      {
        \seq_set_from_clist:Nn \l_@@_typesort_seq {#1}
        \seq_reverse:N \l_@@_typesort_seq
      } ,
    typesort .initial:n =
      { part , chapter , section , paragraph },
    typesort .value_required:n = true ,
    notypesort .code:n =
      { \seq_clear:N \l_@@_typesort_seq } ,
    notypesort .value_forbidden:n = true ,
  }
%    \end{macrocode}
%
%
%
% \subsubsection*{\opt{comp} option}
%
%    \begin{macrocode}
\bool_new:N \l_@@_typeset_compress_bool
\keys_define:nn { zref-clever / reference }
  {
    comp .bool_set:N = \l_@@_typeset_compress_bool ,
    comp .initial:n = true ,
    comp .default:n = true ,
    nocomp .meta:n = { comp = false },
    nocomp .value_forbidden:n = true ,
  }
%    \end{macrocode}
%
%
%
% \subsubsection*{\opt{range} option}
%
%    \begin{macrocode}
\bool_new:N \l_@@_typeset_range_bool
\keys_define:nn { zref-clever / reference }
  {
    range .bool_set:N = \l_@@_typeset_range_bool ,
    range .initial:n = false ,
    range .default:n = true ,
  }
%    \end{macrocode}
%
%
%
% \subsubsection*{\opt{cap} and \opt{capfirst} options}
%
%    \begin{macrocode}
\bool_new:N \l_@@_capitalize_bool
\bool_new:N \l_@@_capitalize_first_bool
\keys_define:nn { zref-clever / reference }
  {
    cap .bool_set:N = \l_@@_capitalize_bool ,
    cap .initial:n = false ,
    cap .default:n = true ,
    nocap .meta:n = { cap = false },
    nocap .value_forbidden:n = true ,

    capfirst .bool_set:N = \l_@@_capitalize_first_bool ,
    capfirst .initial:n = false ,
    capfirst .default:n = true ,
  }
%    \end{macrocode}
%
%
% \subsubsection*{\opt{abbrev} and \opt{noabbrevfirst} options}
%
%    \begin{macrocode}
\bool_new:N \l_@@_abbrev_bool
\bool_new:N \l_@@_noabbrev_first_bool
\keys_define:nn { zref-clever / reference }
  {
    abbrev .bool_set:N = \l_@@_abbrev_bool ,
    abbrev .initial:n = false ,
    abbrev .default:n = true ,
    noabbrev .meta:n = { abbrev = false },
    noabbrev .value_forbidden:n = true ,

    noabbrevfirst .bool_set:N = \l_@@_noabbrev_first_bool ,
    noabbrevfirst .initial:n = false ,
    noabbrevfirst .default:n = true ,
  }
%    \end{macrocode}
%
%
%
% \subsubsection*{\opt{S} option}
%
%    \begin{macrocode}
\keys_define:nn { zref-clever / reference }
  {
    S .meta:n =
      { capfirst = true , noabbrevfirst = true },
    S .value_forbidden:n = true ,
  }
%    \end{macrocode}
%
%
% \subsubsection*{\opt{hyperref} option}
%
%    \begin{macrocode}
\bool_new:N \l_@@_use_hyperref_bool
\bool_new:N \l_@@_warn_hyperref_bool
\keys_define:nn { zref-clever / reference }
  {
    hyperref .choice: ,
    hyperref / auto .code:n =
      {
        \bool_set_true:N \l_@@_use_hyperref_bool
        \bool_set_false:N \l_@@_warn_hyperref_bool
      } ,
    hyperref / true .code:n =
      {
        \bool_set_true:N \l_@@_use_hyperref_bool
        \bool_set_true:N \l_@@_warn_hyperref_bool
      } ,
    hyperref / false .code:n =
      {
        \bool_set_false:N \l_@@_use_hyperref_bool
        \bool_set_false:N \l_@@_warn_hyperref_bool
      } ,
    hyperref .initial:n = auto ,
    hyperref .default:n = auto
  }
%    \end{macrocode}
%
%    \begin{macrocode}
\AddToHook { begindocument }
  {
    \@ifpackageloaded { hyperref }
      {
        \bool_if:NT \l_@@_use_hyperref_bool
          { \RequirePackage { zref-hyperref } }
      }
      {
        \bool_if:NT \l_@@_warn_hyperref_bool
          { \msg_warning:nn { zref-clever } { missing-hyperref } }
        \bool_set_false:N \l_@@_use_hyperref_bool
      }
    \keys_define:nn { zref-clever / reference }
      {
        hyperref .code:n =
          { \msg_warning:nn { zref-clever } { hyperref-preamble-only } }
      }
  }
%    \end{macrocode}
%
%
%
% \subsubsection*{\opt{nameinlink} option}
%
%    \begin{macrocode}
\str_new:N \l_@@_nameinlink_str
\keys_define:nn { zref-clever / reference }
  {
    nameinlink .choice: ,
    nameinlink / true .code:n =
      { \str_set:Nn \l_@@_nameinlink_str { true } } ,
    nameinlink / false .code:n =
      { \str_set:Nn \l_@@_nameinlink_str { false } } ,
    nameinlink / single .code:n =
      { \str_set:Nn \l_@@_nameinlink_str { single } } ,
    nameinlink / tsingle .code:n =
      { \str_set:Nn \l_@@_nameinlink_str { tsingle } } ,
    nameinlink .initial:n = tsingle ,
    nameinlink .default:n = true ,
  }
%    \end{macrocode}
%
%
% \subsubsection*{\opt{preposinlink} option}
%
%    \begin{macrocode}
\bool_new:N \l_@@_preposinlink_bool
\keys_define:nn { zref-clever / reference }
  {
    preposinlink .bool_set:N =  \l_@@_preposinlink_bool ,
    preposinlink .initial:n = false ,
    preposinlink .default:n = true ,
  }
%    \end{macrocode}
%
%
% \subsubsection*{\opt{lang} option}
% \phantomsection{}\zlabel{impl:sec:lang-option}
%
% \cs{l_@@_current_language_tl} is an internal alias for \pkg{babel}'s
% \cs{languagename} or \pkg{polyglossia}'s \cs{mainbabelname} and, if none of
% them is loaded, we set it to \texttt{english}.  \cs{l_@@_main_language_tl}
% is an internal alias for \pkg{babel}'s \cs{bbl@main@language} or for
% \pkg{polyglossia}'s \cs{mainbabelname}, as the case may be. Note that for
% \pkg{polyglossia} we get \pkg{babel}'s language names, so that we only need
% to handle those internally.  \cs{l_@@_ref_language_tl} is the internal
% variable which stores the language in which the reference is to be made.
%
% The overall setup here seems a little roundabout, but this is actually
% required.  In the preamble, we (potentially) don't yet have values for the
% ``main'' and ``current'' document languages, this must be retrieved at a
% \texttt{begindocument} hook.  The \texttt{begindocument} hook is responsible
% to get values for \cs{l_@@_main_language_tl} and
% \cs{l_@@_current_language_tl}, and to set the default for
% \cs{l_@@_ref_language_tl}.  Package options, or preamble calls to
% \cs{zcsetup} are also hooked at \texttt{begindocument}, but come after the
% first hook, so that the pertinent variables have been set when they are
% executed.  Finally, we set a third \texttt{begindocument} hook, at
% \texttt{begindocument/before}, so that it runs after any options set in the
% preamble.  This hook redefines the \opt{lang} option for immediate execution
% in the document body, and ensures the \texttt{main} language's dictionary
% gets loaded, if it hadn't been already.
%
% For the \pkg{babel} and \pkg{polyglossia} variables which store the ``main''
% and ``current'' languages, see \url{https://tex.stackexchange.com/a/233178},
% including comments, particularly the one by Javier Bezos.  For the
% \pkg{babel} and \pkg{polyglossia} variables which store the list of loaded
% languages, see \url{https://tex.stackexchange.com/a/281220}, including
% comments, particularly PLK's.  Note, however, that languages loaded by
% \cs{babelprovide}, either directly, ``on the fly'', or with the
% \texttt{provide} option, \texttt{do not} get included in \cs{bbl@loaded}.
%

%    \begin{macrocode}
\tl_new:N \l_@@_ref_language_tl
\tl_new:N \l_@@_main_language_tl
\tl_new:N \l_@@_current_language_tl
\AddToHook { begindocument }
  {
    \@ifpackageloaded { babel }
      {
        \tl_set:Nn \l_@@_current_language_tl { \languagename }
        \tl_set:Nn \l_@@_main_language_tl { \bbl@main@language }
      }
      {
        \@ifpackageloaded { polyglossia }
          {
            \tl_set:Nn \l_@@_current_language_tl { \babelname }
            \tl_set:Nn \l_@@_main_language_tl { \mainbabelname }
          }
          {
            \tl_set:Nn \l_@@_current_language_tl { english }
            \tl_set:Nn \l_@@_main_language_tl { english }
          }
      }
%    \end{macrocode}
% Provide default value for \cs{l_@@_ref_language_tl} corresponding to option
% \opt{main}, but do so outside of the \pkg{l3keys} machinery (that is,
% instead of using \texttt{.initial:n}), so that we are able to distinguish
% when the user actually gave the option, in which case the dictionary loading
% is done verbosely, from when we are setting the default value (here), in
% which case the dictionary loading is done silently.
%    \begin{macrocode}
    \tl_set:Nn \l_@@_ref_language_tl
      { \l_@@_main_language_tl }
  }
%    \end{macrocode}
%
%
%    \begin{macrocode}
\keys_define:nn { zref-clever / reference }
  {
    lang .code:n =
      {
        \AddToHook { begindocument }
          {
            \str_case:nnF {#1}
              {
                { main }
                {
                  \tl_set:Nn \l_@@_ref_language_tl
                    { \l_@@_main_language_tl }
                  \@@_provide_dictionary_verbose:x
                    { \l_@@_ref_language_tl }
                }

                { current }
                {
                  \tl_set:Nn \l_@@_ref_language_tl
                    { \l_@@_current_language_tl }
                  \@@_provide_dictionary_verbose:x
                    { \l_@@_ref_language_tl }
                }
              }
              {
                \prop_if_in:NnTF \g_@@_languages_prop {#1}
                  {
                    \tl_set:Nn \l_@@_ref_language_tl {#1}
                  }
                  {
                    \msg_warning:nnn { zref-clever }
                      { unknown-language-opt } {#1}
                    \tl_set:Nn \l_@@_ref_language_tl
                      { \l_@@_main_language_tl }
                  }
                \@@_provide_dictionary_verbose:x
                  { \l_@@_ref_language_tl }
              }
          }
      } ,
    lang .value_required:n = true ,
  }
%    \end{macrocode}
%
%
%    \begin{macrocode}
\AddToHook { begindocument / before }
  {
    \AddToHook { begindocument }
      {
%    \end{macrocode}
% If any \opt{lang} option has been given by the user, the corresponding
% language is already loaded, otherwise, ensure the default one
% (\texttt{main}) gets loaded early, but not verbosely.
%    \begin{macrocode}
        \@@_provide_dictionary:x { \l_@@_ref_language_tl }
%    \end{macrocode}
% Redefinition of the \texttt{lang} key option for the document body.  Also,
% drop the verbose dictionary loading in the document body, as it can become
% intrusive depending on the use case, and does not provide much ``juice''
% anyway: in \cs{zcref} missing names warnings will already ensue.
%    \begin{macrocode}
        \keys_define:nn { zref-clever / reference }
          {
            lang .code:n =
              {
                \str_case:nnF {#1}
                  {
                    { main }
                    {
                      \tl_set:Nn \l_@@_ref_language_tl
                        { \l_@@_main_language_tl }
                      \@@_provide_dictionary:x
                        { \l_@@_ref_language_tl }
                    }

                    { current }
                    {
                      \tl_set:Nn \l_@@_ref_language_tl
                        { \l_@@_current_language_tl }
                      \@@_provide_dictionary:x
                        { \l_@@_ref_language_tl }
                    }
                  }
                  {
                    \prop_if_in:NnTF \g_@@_languages_prop {#1}
                      {
                        \tl_set:Nn \l_@@_ref_language_tl {#1}
                      }
                      {
                        \msg_warning:nnn { zref-clever }
                          { unknown-language-opt } {#1}
                        \tl_set:Nn \l_@@_ref_language_tl
                          { \l_@@_main_language_tl }
                      }
                    \@@_provide_dictionary:x
                      { \l_@@_ref_language_tl }
                  }
              } ,
            lang .value_required:n = true ,
          }
      }
  }
%    \end{macrocode}
%
%
%
% \subsubsection*{\opt{d} option}
%
% For setting the declension case.  Short for convenience and for not
% polluting the markup too much given that, for languages that need it, it may
% get to be used frequently.
%
% Thanks \texttt{@samcarter} and Alan Munn for useful comments about
% declension on the TeX.SX chat.  Also, Florent Rougon's efforts in this area,
% with the \pkg{xcref} package (\url{https://github.com/frougon/xcref}), have
% been an insightful source to frame the problem in general terms.
%
%    \begin{macrocode}
\tl_new:N \l_@@_ref_decl_case_tl
\keys_define:nn { zref-clever / reference }
  {
    d .code:n =
      { \msg_warning:nnn { zref-clever } { option-document-only } { d } } ,
  }
\AddToHook { begindocument }
  {
    \keys_define:nn { zref-clever / reference }
      {
%    \end{macrocode}
% We just store the value at this point, which is validated by
% \cs{@@_process_language_options:} after \cs{keys_set:nn}.
%    \begin{macrocode}
        d .tl_set:N = \l_@@_ref_decl_case_tl ,
        d .value_required:n = true ,
      }
  }
%    \end{macrocode}
%
%
%
% \subsubsection*{\opt{nudge} \& Co.\ options}
%
%    \begin{macrocode}
\bool_new:N \l_@@_nudge_enabled_bool
\bool_new:N \l_@@_nudge_multitype_bool
\bool_new:N \l_@@_nudge_comptosing_bool
\bool_new:N \l_@@_nudge_singular_bool
\bool_new:N \l_@@_nudge_gender_bool
\tl_new:N \l_@@_ref_gender_tl
\keys_define:nn { zref-clever / reference }
  {
    nudge .choice: ,
    nudge / true .code:n =
      { \bool_set_true:N \l_@@_nudge_enabled_bool } ,
    nudge / false .code:n =
      { \bool_set_false:N \l_@@_nudge_enabled_bool } ,
    nudge / obeydraft .code:n =
      {
        \ifdraft
          { \bool_set_false:N \l_@@_nudge_enabled_bool }
          { \bool_set_true:N \l_@@_nudge_enabled_bool }
      } ,
    nudge / obeyfinal .code:n =
      {
        \ifoptionfinal
          { \bool_set_true:N \l_@@_nudge_enabled_bool }
          { \bool_set_false:N \l_@@_nudge_enabled_bool }
      } ,
    nudge .initial:n = false ,
    nudge .default:n = true ,
    nonudge .meta:n = { nudge = false } ,
    nonudge .value_forbidden:n = true ,
    nudgeif .code:n =
      {
        \bool_set_false:N \l_@@_nudge_multitype_bool
        \bool_set_false:N \l_@@_nudge_comptosing_bool
        \bool_set_false:N \l_@@_nudge_gender_bool
        \clist_map_inline:nn {#1}
          {
            \str_case:nnF {##1}
              {
                { multitype }
                { \bool_set_true:N \l_@@_nudge_multitype_bool }
                { comptosing }
                { \bool_set_true:N \l_@@_nudge_comptosing_bool }
                { gender }
                { \bool_set_true:N \l_@@_nudge_gender_bool }
                { all }
                {
                  \bool_set_true:N \l_@@_nudge_multitype_bool
                  \bool_set_true:N \l_@@_nudge_comptosing_bool
                  \bool_set_true:N \l_@@_nudge_gender_bool
                }
              }
              {
                \msg_warning:nnn { zref-clever }
                  { nudgeif-unknown-value } {##1}
              }
          }
      } ,
    nudgeif .value_required:n = true ,
    nudgeif .initial:n = all ,
    sg .bool_set:N = \l_@@_nudge_singular_bool ,
    sg .initial:n = false ,
    sg .default:n = true ,
    g .code:n =
      { \msg_warning:nnn { zref-clever } { option-document-only } { g } } ,
  }
\AddToHook { begindocument }
  {
    \keys_define:nn { zref-clever / reference }
      {
%    \end{macrocode}
% We just store the value at this point, which is validated by
% \cs{@@_process_language_options:} after \cs{keys_set:nn}.
%    \begin{macrocode}
        g .tl_set:N = \l_@@_ref_gender_tl ,
        g .value_required:n = true ,
      }
  }
%    \end{macrocode}
%
%
%
% \subsubsection*{\opt{font} option}
%
% \opt{font} \emph{can't be used as a package option}, since the options get
% expanded by \LaTeX{} before being passed to the package (see
% \url{https://tex.stackexchange.com/a/489570}).  It can't be set in
% \cs{zcref} and, for global settings, with \cs{zcsetup}.  Note that,
% technically, the ``raw'' options are already available as
% \cs{@raw@opt@\meta{package}.sty} (see
% \url{https://tex.stackexchange.com/a/618439}, thanks David Carlisle).
%
%    \begin{macrocode}
\tl_new:N \l_@@_ref_typeset_font_tl
\keys_define:nn { zref-clever / reference }
  { font .tl_set:N = \l_@@_ref_typeset_font_tl }
%    \end{macrocode}
%
%
%
% \subsubsection*{\opt{titleref} option}
%
%    \begin{macrocode}
\keys_define:nn { zref-clever / reference }
  {
    titleref .code:n = { \RequirePackage { zref-titleref } } ,
    titleref .value_forbidden:n = true ,
  }
\AddToHook { begindocument }
  {
    \keys_define:nn { zref-clever / reference }
      {
        titleref .code:n =
          { \msg_warning:nn { zref-clever } { titleref-preamble-only } }
      }
  }
%    \end{macrocode}
%
%
% \subsubsection*{\opt{note} option}
%
%    \begin{macrocode}
\tl_new:N \l_@@_zcref_note_tl
\keys_define:nn { zref-clever / reference }
  {
    note .tl_set:N = \l_@@_zcref_note_tl ,
    note .value_required:n = true ,
  }
%    \end{macrocode}
%
%
% \subsubsection*{\opt{check} option}
%
% Integration with \pkg{zref-check}.
%
%    \begin{macrocode}
\bool_new:N \l_@@_zrefcheck_available_bool
\bool_new:N \l_@@_zcref_with_check_bool
\keys_define:nn { zref-clever / reference }
  {
    check .code:n = { \RequirePackage { zref-check } } ,
    check .value_forbidden:n = true ,
  }
\AddToHook { begindocument }
  {
    \@ifpackageloaded { zref-check }
      {
        \bool_set_true:N \l_@@_zrefcheck_available_bool
        \keys_define:nn { zref-clever / reference }
          {
            check .code:n =
              {
                \bool_set_true:N \l_@@_zcref_with_check_bool
                \keys_set:nn { zref-check / zcheck } {#1}
              } ,
            check .value_required:n = true ,
          }
      }
      {
        \bool_set_false:N \l_@@_zrefcheck_available_bool
        \keys_define:nn { zref-clever / reference }
          {
            check .value_forbidden:n = false ,
            check .code:n =
              { \msg_warning:nn { zref-clever } { missing-zref-check } } ,
          }
      }
  }
%    \end{macrocode}
%
%
% \subsubsection*{\opt{countertype} option}
%
% \cs{l_@@_counter_type_prop} is used by \texttt{zc@type} property, and stores
% a mapping from ``counter'' to ``reference type''.  Only those counters whose
% type name is different from that of the counter need to be specified, since
% \texttt{zc@type} presumes the counter as the type if the counter is not
% found in \cs{l_@@_counter_type_prop}.
%
%    \begin{macrocode}
\prop_new:N \l_@@_counter_type_prop
\keys_define:nn { zref-clever / label }
  {
    countertype .code:n =
      {
        \keyval_parse:nnn
          {
            \msg_warning:nnnn { zref-clever }
              { key-requires-value } { countertype }
          }
          {
            \@@_prop_put_non_empty:Nnn
              \l_@@_counter_type_prop
          }
          {#1}
      } ,
    countertype .value_required:n = true ,
    countertype .initial:n =
      {
        subsection    = section ,
        subsubsection = section ,
        subparagraph  = paragraph ,
        enumi         = item ,
        enumii        = item ,
        enumiii       = item ,
        enumiv        = item ,
        mpfootnote    = footnote ,
      } ,
  }
%    \end{macrocode}
%
%
% \subsubsection*{\opt{counterresetters} option}
%
% \cs{l_@@_counter_resetters_seq} is used by \cs{@@_counter_reset_by:n} to
% populate the \texttt{zc@enclval} property, and stores the list of counters
% which are potential ``enclosing counters'' for other counters.  This option
% is constructed such that users can only \emph{add} items to the variable.
% There would be little gain and some risk in allowing removal, and the syntax
% of the option would become unnecessarily more complicated.  Besides, users
% can already override, for any particular counter, the search done from the
% set in \cs{l_@@_counter_resetters_seq} with the \opt{counterresetby} option.
%
%    \begin{macrocode}
\seq_new:N \l_@@_counter_resetters_seq
\keys_define:nn { zref-clever / label }
  {
    counterresetters .code:n =
      {
        \clist_map_inline:nn {#1}
          {
            \seq_if_in:NnF \l_@@_counter_resetters_seq {##1}
              {
                \seq_put_right:Nn
                  \l_@@_counter_resetters_seq {##1}
              }
          }
      } ,
    counterresetters .initial:n =
      {
        part ,
        chapter ,
        section ,
        subsection ,
        subsubsection ,
        paragraph ,
        subparagraph ,
      },
    counterresetters .value_required:n = true ,
  }
%    \end{macrocode}
%
%
%
% \subsubsection*{\opt{counterresetby} option}
%
% \cs{l_@@_counter_resetby_prop} is used by \cs{@@_counter_reset_by:n} to
% populate the \texttt{zc@enclval} property, and stores a mapping from
% counters to the counter which resets each of them.  This mapping has
% precedence in \cs{@@_counter_reset_by:n} over the search through
% \cs{l_@@_counter_resetters_seq}.
%
%    \begin{macrocode}
\prop_new:N \l_@@_counter_resetby_prop
\keys_define:nn { zref-clever / label }
  {
    counterresetby .code:n =
      {
        \keyval_parse:nnn
          {
            \msg_warning:nnn { zref-clever }
              { key-requires-value } { counterresetby }
          }
          {
            \@@_prop_put_non_empty:Nnn
              \l_@@_counter_resetby_prop
          }
          {#1}
      } ,
    counterresetby .value_required:n = true ,
    counterresetby .initial:n =
      {
%    \end{macrocode}
% The counters for the \texttt{enumerate} environment do not use the regular
% counter machinery for resetting on each level, but are nested nevertheless
% by other means, treat them as exception.
%    \begin{macrocode}
        enumii  = enumi   ,
        enumiii = enumii  ,
        enumiv  = enumiii ,
      } ,
  }
%    \end{macrocode}
%
%
% \subsubsection*{\opt{currentcounter} option}
%
% \cs{l_@@_current_counter_tl} is pretty much the starting point of all of the
% data specification for label setting done by \pkg{zref} with our setup for
% it.  It exists because we must provide some ``handle'' to specify the
% current counter for packages/features that do not set \cs{@currentcounter}
% appropriately.
%
%    \begin{macrocode}
\tl_new:N \l_@@_current_counter_tl
\keys_define:nn { zref-clever / label }
  {
    currentcounter .tl_set:N = \l_@@_current_counter_tl ,
    currentcounter .value_required:n = true ,
    currentcounter .initial:n = \@currentcounter ,
  }
%    \end{macrocode}
%
%
%
% \subsubsection*{Reference options}
% \zlabel{impl:sec:reference-options}
%
% This is a set of options related to reference typesetting which receive
% equal treatment and, hence, are handled in batch.  Since we are dealing with
% options to be passed to \cs{zcref} or to \cs{zcsetup} or at load time, only
% ``not necessarily type-specific'' options are pertinent here.  However, they
% \emph{may} either be type-specific or language-specific, and thus must be
% stored in a property list, \cs{l_@@_ref_options_prop}, in order to be
% retrieved from the option \emph{name} by \cs{@@_get_ref_string:nN} and
% \cs{@@_get_ref_font:nN} according to context and precedence rules.
%
% The keys are set so that any value, including an empty one, is added to
% \cs{l_@@_ref_options_prop}, while a key with \emph{no value} removes the
% property from the list, so that these options can then fall back to lower
% precedence levels settings.  For discussion about the used technique, see
% \zcref{impl:sec:zcreftypesetup}.
%
%    \begin{macrocode}
\prop_new:N \l_@@_ref_options_prop
\seq_map_inline:Nn
  \c_@@_ref_options_reference_seq
  {
    \keys_define:nn { zref-clever / reference }
      {
        #1 .default:V = \c_novalue_tl ,
        #1 .code:n =
          {
            \tl_if_novalue:nTF {##1}
              { \prop_remove:Nn \l_@@_ref_options_prop {#1} }
              { \prop_put:Nnn \l_@@_ref_options_prop {#1} {##1} }
          } ,
      }
  }
%    \end{macrocode}
%
%
%
% \subsubsection*{Package options}
%
% The options have been separated in two different groups, so that we can
% potentially apply them selectively to different contexts: \texttt{label} and
% \texttt{reference}.  Currently, the only use of this selection is the
% ability to exclude label related options from \cs{zcref}'s options.  Anyway,
% for load-time package options and for \cs{zcsetup} we want the whole set, so
% we aggregate the two into \texttt{zref-clever/zcsetup}, and use that here.
%
%    \begin{macrocode}
\keys_define:nn { }
  {
    zref-clever / zcsetup .inherit:n =
      {
        zref-clever / label ,
        zref-clever / reference ,
      }
  }
%    \end{macrocode}
%
%
% Process load-time package options
% (\url{https://tex.stackexchange.com/a/15840}).
%    \begin{macrocode}
\ProcessKeysOptions { zref-clever / zcsetup }
%    \end{macrocode}
%
%
%
% \section{Configuration}
%
% \subsection{\cs{zcsetup}}
%
%
% \begin{macro}[int]{\zcsetup}
%   Provide \cs{zcsetup}.
%   \begin{syntax}
%     \cs{zcsetup}\marg{options}
%   \end{syntax}
%    \begin{macrocode}
\NewDocumentCommand \zcsetup { m }
  { \@@_zcsetup:n {#1} }
%    \end{macrocode}
% \end{macro}
%
% \begin{macro}{\@@_zcsetup:n}
%   A version of \cs{zcsetup} for internal use with variant.
%   \begin{syntax}
%     \cs{@@_zcsetup:n}\marg{options}
%   \end{syntax}
%    \begin{macrocode}
\cs_new_protected:Npn \@@_zcsetup:n #1
  { \keys_set:nn { zref-clever / zcsetup } {#1} }
\cs_generate_variant:Nn \@@_zcsetup:n { x }
%    \end{macrocode}
% \end{macro}
%
%
%
% \subsection{\cs{zcRefTypeSetup}}
% \zlabel{impl:sec:zcreftypesetup}
%
% \cs{zcRefTypeSetup} is the main user interface for ``type-specific''
% reference formatting.  Settings done by this command have a higher
% precedence than any translation, hence they override any language-specific
% setting, either done at \cs{zcLanguageSetup} or by the package's
% dictionaries.  On the other hand, they have a lower precedence than non
% type-specific general options.  The \meta{options} should be given in the
% usual \texttt{key=val} format.  The \meta{type} does not need to pre-exist,
% the property list variable to store the properties for the type gets created
% if need be.
%
% \begin{macro}[int]{\zcRefTypeSetup}
%   \begin{syntax}
%     \cs{zcRefTypeSetup} \marg{type} \marg{options}
%   \end{syntax}
%    \begin{macrocode}
\NewDocumentCommand \zcRefTypeSetup { m m }
  {
    \prop_if_exist:cF { l_@@_type_ #1 _options_prop }
      { \prop_new:c { l_@@_type_ #1 _options_prop } }
    \tl_set:Nn \l_@@_setup_type_tl {#1}
    \keys_set:nn { zref-clever / typesetup } {#2}
  }
%    \end{macrocode}
% \end{macro}
%
%
% Inside \cs{zcRefTypeSetup} any of the options \emph{can} receive empty
% values, and those values, if they exist in the property list, will override
% translations, regardless of their emptiness.  In principle, we could live
% with the situation of, once a setting has been made in
% \cs{l_@@_type_<type>_options_prop} or in \cs{l_@@_ref_options_prop} it stays
% there forever, and can only be overridden by a new value at the same
% precedence level or a higher one.  But it would be nice if an user can
% ``unset'' an option at either of those scopes to go back to the lower
% precedence level of the translations at any given point.  So both in
% \cs{zcRefTypeSetup} and in setting reference options (see
% \zcref{impl:sec:reference-options}), we leverage the distinction of an
% ``empty valued key'' (\texttt{key=} or \texttt{key=\{\}}) from a ``key with
% no value'' (\texttt{key}).  This distinction is captured internally by the
% lower-level key parsing, but must be made explicit at \cs{keys_set:nn} by
% means of the \texttt{.default:V} property of the key in \cs{keys_define:nn}.
% For the technique and some discussion about it, see
% \url{https://tex.stackexchange.com/q/614690} (thanks Jonathan P.\ Spratte,
% aka `Skillmon', and Phelype Oleinik) and
% \url{https://github.com/latex3/latex3/pull/988}.
%
%    \begin{macrocode}
\seq_map_inline:Nn
  \c_@@_ref_options_necessarily_not_type_specific_seq
  {
    \keys_define:nn { zref-clever / typesetup }
      {
        #1 .code:n =
          {
            \msg_warning:nnn { zref-clever }
              { option-not-type-specific } {#1}
          } ,
      }
  }
%    \end{macrocode}
%
%
%    \begin{macrocode}
\seq_map_inline:Nn
  \c_@@_ref_options_typesetup_seq
  {
    \keys_define:nn { zref-clever / typesetup }
      {
        #1 .default:V = \c_novalue_tl ,
        #1 .code:n =
          {
            \tl_if_novalue:nTF {##1}
              {
                \prop_remove:cn
                  {
                    l_@@_type_
                    \l_@@_setup_type_tl _options_prop
                  }
                  {#1}
              }
              {
                \prop_put:cnn
                  {
                    l_@@_type_
                    \l_@@_setup_type_tl _options_prop
                  }
                  {#1} {##1}
              }
          } ,
      }
  }
%    \end{macrocode}
%
%
%
% \subsection{\cs{zcLanguageSetup}}
%
% \cs{zcLanguageSetup} is the main user interface for ``language-specific''
% reference formatting, be it ``type-specific'' or not.  The difference
% between the two cases is captured by the \texttt{type} key, which works as a
% sort of a ``switch''.  Inside the \meta{options} argument of
% \cs{zcLanguageSetup}, any options made before the first \texttt{type} key
% declare ``default'' (non type-specific) translations.  When the
% \texttt{type} key is given with a value, the options following it will set
% ``type-specific'' translations for that type.  The current type can be
% switched off by an empty \texttt{type} key.  \cs{zcLanguageSetup} is
% preamble only.
%
% \begin{macro}[int]{\zcLanguageSetup}
%   \begin{syntax}
%     \cs{zcLanguageSetup}\marg{language}\marg{options}
%   \end{syntax}
%    \begin{macrocode}
\NewDocumentCommand \zcLanguageSetup { m m }
  {
    \group_begin:
    \prop_get:NnNTF \g_@@_languages_prop {#1}
      \l_@@_dict_language_tl
      {
        \tl_clear:N \l_@@_setup_type_tl
        \exp_args:NNx \seq_set_from_clist:Nn
          \l_@@_dict_declension_seq
          {
            \prop_item:cn
              {
                g_@@_dict_
                \l_@@_dict_language_tl _prop
              }
              { declension }
          }
        \seq_if_empty:NTF \l_@@_dict_declension_seq
          { \tl_clear:N \l_@@_dict_decl_case_tl }
          {
            \seq_get_left:NN \l_@@_dict_declension_seq
              \l_@@_dict_decl_case_tl
          }
        \exp_args:NNx \seq_set_from_clist:Nn
          \l_@@_dict_gender_seq
          {
            \prop_item:cn
              {
                g_@@_dict_
                \l_@@_dict_language_tl _prop
              }
              { gender }
          }
        \keys_set:nn { zref-clever / langsetup } {#2}
      }
      { \msg_warning:nnn { zref-clever } { unknown-language-setup } {#1} }
    \group_end:
  }
\@onlypreamble \zcLanguageSetup
%    \end{macrocode}
% \end{macro}
%
%
% \begin{macro}
%   {
%     \@@_declare_type_transl:nnnn ,
%     \@@_declare_default_transl:nnn ,
%   }
%   A couple of auxiliary functions for the of
%   \texttt{{zref-clever/translation}} keys set in \cs{zcLanguageSetup}.  They
%   respectively declare (unconditionally set) ``type-specific'' and
%   ``default'' translations.
%   \begin{syntax}
%     \cs{@@_declare_type_transl:nnnn} \Arg{language} \Arg{type}
%     ~~\Arg{key} \Arg{translation}
%     \cs{@@_declare_default_transl:nnn} \Arg{language}
%     ~~\Arg{key} \Arg{translation}
%   \end{syntax}
%    \begin{macrocode}
\cs_new_protected:Npn \@@_declare_type_transl:nnnn #1#2#3#4
  {
    \prop_gput:cnn { g_@@_dict_ #1 _prop }
      { type- #2 - #3 } {#4}
  }
\cs_generate_variant:Nn \@@_declare_type_transl:nnnn { VVnn , VVxn }
\cs_new_protected:Npn \@@_declare_default_transl:nnn #1#2#3
  {
    \prop_gput:cnn { g_@@_dict_ #1 _prop }
      { default- #2 } {#3}
  }
\cs_generate_variant:Nn \@@_declare_default_transl:nnn { Vnn }
%    \end{macrocode}
% \end{macro}
%
%
% The set of keys for \texttt{{zref-clever/langsetup}}, which is used to
% set language-specific translations in \cs{zcLanguageSetup}.
%
%    \begin{macrocode}
\keys_define:nn { zref-clever / langsetup }
  {
    type .code:n =
      {
        \tl_if_empty:nTF {#1}
          { \tl_clear:N \l_@@_setup_type_tl }
          { \tl_set:Nn \l_@@_setup_type_tl {#1} }
      } ,
    case .code:n =
      {
        \seq_if_empty:NTF \l_@@_dict_declension_seq
          {
            \msg_warning:nnxx { zref-clever } { language-no-decl-setup }
              { \l_@@_dict_language_tl } {#1}
          }
          {
            \seq_if_in:NnTF \l_@@_dict_declension_seq {#1}
              { \tl_set:Nn \l_@@_dict_decl_case_tl {#1} }
              {
                \msg_warning:nnxx { zref-clever } { unknown-decl-case }
                  {#1} { \l_@@_dict_language_tl }
                \seq_get_left:NN \l_@@_dict_declension_seq
                  \l_@@_dict_decl_case_tl
              }
          }
      } ,
    case .value_required:n = true ,
    gender .code:n =
      {
        \seq_if_empty:NTF \l_@@_dict_gender_seq
          {
            \msg_warning:nnxxx { zref-clever } { language-no-gender }
              { \l_@@_dict_language_tl } { gender } {#1}
          }
          {
            \tl_if_empty:NTF \l_@@_setup_type_tl
              {
                \msg_warning:nnn { zref-clever }
                  { option-only-type-specific } { gender }
              }
              {
                \seq_if_in:NnTF \l_@@_dict_gender_seq {#1}
                  {
                    \@@_declare_type_transl:VVnn
                      \l_@@_dict_language_tl
                      \l_@@_setup_type_tl
                      { gender } {#1}
                  }
                  {
                    \msg_warning:nnxx { zref-clever } { gender-not-declared }
                      { \l_@@_dict_language_tl } {#1}
                  }
              }
          }
      } ,
    gender .value_required:n = true ,
  }
\seq_map_inline:Nn
  \c_@@_ref_options_necessarily_not_type_specific_seq
  {
    \keys_define:nn { zref-clever / langsetup }
      {
        #1 .value_required:n = true ,
        #1 .code:n =
          {
            \tl_if_empty:NTF \l_@@_setup_type_tl
              {
                \@@_declare_default_transl:Vnn
                  \l_@@_dict_language_tl
                  {#1} {##1}
              }
              {
                \msg_warning:nnn { zref-clever }
                  { option-not-type-specific } {#1}
              }
          } ,
      }
  }
\seq_map_inline:Nn
  \c_@@_ref_options_possibly_type_specific_seq
  {
    \keys_define:nn { zref-clever / langsetup }
      {
        #1 .value_required:n = true ,
        #1 .code:n =
          {
            \tl_if_empty:NTF \l_@@_setup_type_tl
              {
                \@@_declare_default_transl:Vnn
                  \l_@@_dict_language_tl
                  {#1} {##1}
              }
              {
                \@@_declare_type_transl:VVnn
                  \l_@@_dict_language_tl
                  \l_@@_setup_type_tl
                  {#1} {##1}
              }
          } ,
      }
  }
\seq_map_inline:Nn
  \c_@@_ref_options_type_names_seq
  {
    \keys_define:nn { zref-clever / langsetup }
      {
        #1 .value_required:n = true ,
        #1 .code:n =
          {
            \tl_if_empty:NTF \l_@@_setup_type_tl
              {
                \msg_warning:nnn { zref-clever }
                  { option-only-type-specific } {#1}
              }
              {
                \tl_if_empty:NTF \l_@@_dict_decl_case_tl
                  {
                    \@@_declare_type_transl:VVnn
                      \l_@@_dict_language_tl
                      \l_@@_setup_type_tl
                      {#1} {##1}
                  }
                  {
                    \@@_declare_type_transl:VVxn
                      \l_@@_dict_language_tl
                      \l_@@_setup_type_tl
                      { \l_@@_dict_decl_case_tl - #1 } {##1}
                  }
              }
          } ,
      }
  }
%    \end{macrocode}
%
%
%
% \section{User interface}
%
% \subsection{\cs{zcref}}
%
%
% \begin{macro}[int]{\zcref}
%   The main user command of the package.
%   \begin{syntax}
%     \cs{zcref}\meta{*}\oarg{options}\marg{labels}
%   \end{syntax}
%    \begin{macrocode}
\NewDocumentCommand \zcref { s O { } m }
  { \zref@wrapper@babel \@@_zcref:nnn {#3} {#1} {#2} }
%    \end{macrocode}
% \end{macro}
%
%
% \begin{macro}{\@@_zcref:nnnn}
%   An intermediate internal function, which does the actual heavy lifting,
%   and places \Arg{labels} as first argument, so that it can be protected by
%   \cs{zref@wrapper@babel} in \cs{zcref}.
%   \begin{syntax}
%     \cs{@@_zcref:nnnn} \Arg{labels} \Arg{*} \Arg{options}
%   \end{syntax}
%    \begin{macrocode}
\cs_new_protected:Npn \@@_zcref:nnn #1#2#3
  {
    \group_begin:
%    \end{macrocode}
% Set options.
%    \begin{macrocode}
      \keys_set:nn { zref-clever / reference } {#3}
%    \end{macrocode}
% Store arguments values.
%    \begin{macrocode}
      \seq_set_from_clist:Nn \l_@@_zcref_labels_seq {#1}
      \bool_set:Nn \l_@@_link_star_bool {#2}
%    \end{macrocode}
% Ensure dictionary for reference language is loaded, if available.  We cannot
% rely on \cs{keys_set:nn} for the task, since if the \opt{lang} option is set
% for \texttt{current}, the actual language may have changed outside our
% control.  \cs{@@_provide_dictionary:x} does nothing if the dictionary is
% already loaded.
%    \begin{macrocode}
      \@@_provide_dictionary:x { \l_@@_ref_language_tl }
%    \end{macrocode}
% Process \cs{zcDeclareLanguage} options.
%    \begin{macrocode}
      \@@_process_language_options:
%    \end{macrocode}
% Integration with \pkg{zref-check}.
%    \begin{macrocode}
      \bool_lazy_and:nnT
        { \l_@@_zrefcheck_available_bool }
        { \l_@@_zcref_with_check_bool }
        { \zrefcheck_zcref_beg_label: }
%    \end{macrocode}
% Sort the labels.
%    \begin{macrocode}
      \bool_lazy_or:nnT
        { \l_@@_typeset_sort_bool }
        { \l_@@_typeset_range_bool }
        { \@@_sort_labels: }
%    \end{macrocode}
% Typeset the references.  Also, set the reference font, and group it, so that
% it does not leak to the note.
%    \begin{macrocode}
      \group_begin:
      \l_@@_ref_typeset_font_tl
      \@@_typeset_refs:
      \group_end:
%    \end{macrocode}
% Typeset \texttt{note}.
%    \begin{macrocode}
      \tl_if_empty:NF \l_@@_zcref_note_tl
        {
          \@@_get_ref_string:nN { notesep } \l_tmpa_tl
          \l_tmpa_tl
          \l_@@_zcref_note_tl
        }
%    \end{macrocode}
% Integration with \pkg{zref-check}.
%    \begin{macrocode}
      \bool_lazy_and:nnT
        { \l_@@_zrefcheck_available_bool }
        { \l_@@_zcref_with_check_bool }
        {
          \zrefcheck_zcref_end_label_maybe:
          \zrefcheck_zcref_run_checks_on_labels:n
            { \l_@@_zcref_labels_seq }
        }
%    \end{macrocode}
% Integration with \pkg{mathtools}.
%    \begin{macrocode}
    \bool_if:NT \l_@@_mathtools_showonlyrefs_bool
      {
        \@@_mathtools_showonlyrefs:n
          { \l_@@_zcref_labels_seq }
      }
    \group_end:
  }
%    \end{macrocode}
% \end{macro}
%
% \begin{variable}{\l_@@_zcref_labels_seq, \l_@@_link_star_bool}
%    \begin{macrocode}
\seq_new:N \l_@@_zcref_labels_seq
\bool_new:N \l_@@_link_star_bool
%    \end{macrocode}
% \end{variable}
%
%
%
% \subsection{\cs{zcpageref}}
%
%
% \begin{macro}[int]{\zcpageref}
%   A \cs{pageref} equivalent of \cs{zcref}.
%   \begin{syntax}
%     \cs{zcpageref}\meta{*}\oarg{options}\marg{labels}
%   \end{syntax}
%    \begin{macrocode}
\NewDocumentCommand \zcpageref { s O { } m }
  {
    \IfBooleanTF {#1}
      { \zcref*[#2, ref = page] {#3} }
      { \zcref [#2, ref = page] {#3} }
  }
%    \end{macrocode}
% \end{macro}
%
%
%
% \section{Sorting}
%
% Sorting is certainly a ``big task'' for \pkg{zref-clever} but, in the end,
% it boils down to ``carefully done branching'', and quite some of it.  The
% sorting of ``page'' references is very much lightened by the availability of
% \texttt{abspage}, from the \pkg{zref-abspage} module, which offers ``just
% what we need'' for our purposes.  The sorting of ``default'' references
% falls on two main cases: i) labels of the same type; ii) labels of different
% types.  The first case is sorted according to the priorities set by the
% \opt{typesort} option or, if that is silent for the case, by the order in
% which labels were given by the user in \cs{zcref}.  The second case is the
% most involved one, since it is possible for multiple counters to be bundled
% together in a single reference type.  Because of this, sorting must take
% into account the whole chain of ``enclosing counters'' for the counters of
% the labels at hand.
%
% \begin{variable}
%   {
%     \l_@@_label_type_a_tl ,
%     \l_@@_label_type_b_tl ,
%     \l_@@_label_enclval_a_tl ,
%     \l_@@_label_enclval_b_tl ,
%     \l_@@_label_extdoc_a_tl ,
%     \l_@@_label_extdoc_b_tl ,
%   }
%   Auxiliary variables, for use in sorting, and some also in typesetting.
%   Used to store reference information -- label properties -- of the
%   ``current'' (\texttt{a}) and ``next'' (\texttt{b}) labels.
%    \begin{macrocode}
\tl_new:N \l_@@_label_type_a_tl
\tl_new:N \l_@@_label_type_b_tl
\tl_new:N \l_@@_label_enclval_a_tl
\tl_new:N \l_@@_label_enclval_b_tl
\tl_new:N \l_@@_label_extdoc_a_tl
\tl_new:N \l_@@_label_extdoc_b_tl
%    \end{macrocode}
% \end{variable}
%
% \begin{variable}{\l_@@_sort_decided_bool}
%   Auxiliary variable for \cs{@@_sort_default_same_type:nn}, signals if the
%   sorting between two labels has been decided or not.
%    \begin{macrocode}
\bool_new:N \l_@@_sort_decided_bool
%    \end{macrocode}
% \end{variable}
%
% \begin{variable}{\l_@@_sort_prior_a_int,\l_@@_sort_prior_b_int}
%   Auxiliary variables for \cs{@@_sort_default_different_types:nn}.  Store
%   the sort priority of the ``current'' and ``next'' labels.
%    \begin{macrocode}
\int_new:N \l_@@_sort_prior_a_int
\int_new:N \l_@@_sort_prior_b_int
%    \end{macrocode}
% \end{variable}
%
% \begin{variable}{\l_@@_label_types_seq}
%   Stores the order in which reference types appear in the label list
%   supplied by the user in \cs{zcref}.  This variable is populated by
%   \cs{@@_label_type_put_new_right:n} at the start of \cs{@@_sort_labels:}.
%   This order is required as a ``last resort'' sort criterion between the
%   reference types, for use in \cs{@@_sort_default_different_types:nn}.
%    \begin{macrocode}
\seq_new:N \l_@@_label_types_seq
%    \end{macrocode}
% \end{variable}
%
%
% \begin{macro}{\@@_sort_labels:}
%   The main sorting function.  It does not receive arguments, but it is
%   expected to be run inside \cs{@@_zcref:nnnn} where a number of environment
%   variables are to be set appropriately.  In particular,
%   \cs{l_@@_zcref_labels_seq} should contain the labels received as argument
%   to \cs{zcref}, and the function performs its task by sorting this
%   variable.
%    \begin{macrocode}
\cs_new_protected:Npn \@@_sort_labels:
  {
%    \end{macrocode}
% Store label types sequence.
%    \begin{macrocode}
    \seq_clear:N \l_@@_label_types_seq
    \tl_if_eq:NnF \l_@@_ref_property_tl { page }
      {
        \seq_map_function:NN \l_@@_zcref_labels_seq
          \@@_label_type_put_new_right:n
      }
%    \end{macrocode}
% Sort.
%    \begin{macrocode}
    \seq_sort:Nn \l_@@_zcref_labels_seq
      {
        \zref@ifrefundefined {##1}
          {
            \zref@ifrefundefined {##2}
              {
                % Neither label is defined.
                \sort_return_same:
              }
              {
                % The second label is defined, but the first isn't, leave the
                % undefined first (to be more visible).
                \sort_return_same:
              }
          }
          {
            \zref@ifrefundefined {##2}
              {
                % The first label is defined, but the second isn't, bring the
                % second forward.
                \sort_return_swapped:
              }
              {
                % The interesting case: both labels are defined.  References
                % to the "default" property or to the "page" are quite
                % different with regard to sorting, so we branch them here to
                % specialized functions.
                \tl_if_eq:NnTF \l_@@_ref_property_tl { page }
                  { \@@_sort_page:nn {##1} {##2} }
                  { \@@_sort_default:nn {##1} {##2} }
              }
          }
      }
  }
%    \end{macrocode}
% \end{macro}
%
% \begin{macro}{\@@_label_type_put_new_right:n}
%   Auxiliary function used to store the order in which reference types appear
%   in the label list supplied by the user in \cs{zcref}.  It is expected to
%   be run inside \cs{@@_sort_labels:}, and stores the types sequence in
%   \cs{l_@@_label_types_seq}.  I have tried to handle the same task inside
%   \cs{seq_sort:Nn} in \cs{@@_sort_labels:} to spare mapping over
%   \cs{l_@@_zcref_labels_seq}, but it turned out it not to be easy to rely on
%   the order the labels get processed at that point, since the variable is
%   being sorted there.  Besides, the mapping is simple, not a particularly
%   expensive operation.  Anyway, this keeps things clean.
%   \begin{syntax}
%     \cs{@@_label_type_put_new_right:n} \Arg{label}
%   \end{syntax}
%    \begin{macrocode}
\cs_new_protected:Npn \@@_label_type_put_new_right:n #1
  {
    \@@_def_extract:Nnnn
      \l_@@_label_type_a_tl {#1} { zc@type } { \c_empty_tl }
    \seq_if_in:NVF \l_@@_label_types_seq
      \l_@@_label_type_a_tl
      {
        \seq_put_right:NV \l_@@_label_types_seq
          \l_@@_label_type_a_tl
      }
  }
%    \end{macrocode}
% \end{macro}
%
%
% \begin{macro}{\@@_sort_default:nn}
%   The heavy-lifting function for sorting of defined labels for ``default''
%   references (that is, a standard reference, not to ``page'').  This
%   function is expected to be called within the sorting loop of
%   \cs{@@_sort_labels:} and receives the pair of labels being considered for
%   a change of order or not.  It should \emph{always} ``return'' either
%   \cs{sort_return_same:} or \cs{sort_return_swapped:}.
%   \begin{syntax}
%     \cs{@@_sort_default:nn} \Arg{label a} \Arg{label b}
%   \end{syntax}
%    \begin{macrocode}
\cs_new_protected:Npn \@@_sort_default:nn #1#2
  {
    \@@_def_extract:Nnnn
      \l_@@_label_type_a_tl {#1} { zc@type } { zc@missingtype }
    \@@_def_extract:Nnnn
      \l_@@_label_type_b_tl {#2} { zc@type } { zc@missingtype }

    \tl_if_eq:NNTF
      \l_@@_label_type_a_tl
      \l_@@_label_type_b_tl
      { \@@_sort_default_same_type:nn {#1} {#2} }
      { \@@_sort_default_different_types:nn {#1} {#2} }
  }
%    \end{macrocode}
% \end{macro}
%
%
% \begin{macro}{\@@_sort_default_same_type:nn}
%   \begin{syntax}
%     \cs{@@_sort_default_same_type:nn} \Arg{label a} \Arg{label b}
%   \end{syntax}
%    \begin{macrocode}
\cs_new_protected:Npn \@@_sort_default_same_type:nn #1#2
  {
    \@@_def_extract:Nnnn \l_@@_label_enclval_a_tl
      {#1} { zc@enclval } { \c_empty_tl }
    \tl_reverse:N \l_@@_label_enclval_a_tl
    \@@_def_extract:Nnnn \l_@@_label_enclval_b_tl
      {#2} { zc@enclval } { \c_empty_tl }
    \tl_reverse:N \l_@@_label_enclval_b_tl
    \@@_def_extract:Nnnn \l_@@_label_extdoc_a_tl
      {#1} { externaldocument } { \c_empty_tl }
    \@@_def_extract:Nnnn \l_@@_label_extdoc_b_tl
      {#2} { externaldocument } { \c_empty_tl }

    \bool_set_false:N \l_@@_sort_decided_bool

    % First we check if there's any "external document" difference (coming
    % from 'zref-xr') and, if so, sort based on that.
    \tl_if_eq:NNF
      \l_@@_label_extdoc_a_tl
      \l_@@_label_extdoc_b_tl
      {
        \bool_if:nTF
          {
            \tl_if_empty_p:V \l_@@_label_extdoc_a_tl &&
            ! \tl_if_empty_p:V \l_@@_label_extdoc_b_tl
          }
          {
            \bool_set_true:N \l_@@_sort_decided_bool
            \sort_return_same:
          }
          {
            \bool_if:nTF
              {
                ! \tl_if_empty_p:V \l_@@_label_extdoc_a_tl &&
                \tl_if_empty_p:V \l_@@_label_extdoc_b_tl
              }
              {
                \bool_set_true:N \l_@@_sort_decided_bool
                \sort_return_swapped:
              }
              {
                \bool_set_true:N \l_@@_sort_decided_bool
                % Two different "external documents": last resort, sort by the
                % document name itself.
                \str_compare:eNeTF
                  { \l_@@_label_extdoc_b_tl } <
                  { \l_@@_label_extdoc_a_tl }
                  { \sort_return_swapped: }
                  { \sort_return_same:    }
              }
          }
      }

    \bool_until_do:Nn \l_@@_sort_decided_bool
      {
        \bool_if:nTF
          {
            % Both are empty: neither label has any (further) "enclosing
            % counters" (left).
            \tl_if_empty_p:V \l_@@_label_enclval_a_tl &&
            \tl_if_empty_p:V \l_@@_label_enclval_b_tl
          }
          {
            \bool_set_true:N \l_@@_sort_decided_bool
            \int_compare:nNnTF
              { \@@_extract:nnn {#1} { zc@cntval } { -1 } }
                >
              { \@@_extract:nnn {#2} { zc@cntval } { -1 } }
              { \sort_return_swapped: }
              { \sort_return_same:    }
          }
          {
            \bool_if:nTF
              {
                % `a' is empty (and `b' is not): `b' may be nested in `a'.
                \tl_if_empty_p:V \l_@@_label_enclval_a_tl
              }
              {
                \bool_set_true:N \l_@@_sort_decided_bool
                \int_compare:nNnTF
                  { \@@_extract:nnn {#1} { zc@cntval } { } }
                    >
                  { \tl_head:N \l_@@_label_enclval_b_tl }
                  { \sort_return_swapped: }
                  { \sort_return_same:    }
              }
              {
                \bool_if:nTF
                  {
                    % `b' is empty (and `a' is not): `a' may be nested in `b'.
                    \tl_if_empty_p:V \l_@@_label_enclval_b_tl
                  }
                  {
                    \bool_set_true:N \l_@@_sort_decided_bool
                    \int_compare:nNnTF
                      { \tl_head:N \l_@@_label_enclval_a_tl }
                        <
                      { \@@_extract:nnn {#2} { zc@cntval } { } }
                      { \sort_return_same:    }
                      { \sort_return_swapped: }
                  }
                  {
                    % Neither is empty: we can compare the values of the
                    % current enclosing counter in the loop, if they are
                    % equal, we are still in the loop, if they are not, a
                    % sorting decision can be made directly.
                    \int_compare:nNnTF
                      { \tl_head:N \l_@@_label_enclval_a_tl }
                        =
                      { \tl_head:N \l_@@_label_enclval_b_tl }
                      {
                        \tl_set:Nx \l_@@_label_enclval_a_tl
                          { \tl_tail:N \l_@@_label_enclval_a_tl }
                        \tl_set:Nx \l_@@_label_enclval_b_tl
                          { \tl_tail:N \l_@@_label_enclval_b_tl }
                      }
                      {
                        \bool_set_true:N \l_@@_sort_decided_bool
                        \int_compare:nNnTF
                          { \tl_head:N \l_@@_label_enclval_a_tl }
                            >
                          { \tl_head:N \l_@@_label_enclval_b_tl }
                          { \sort_return_swapped: }
                          { \sort_return_same:    }
                      }
                  }
              }
          }
      }
  }
%    \end{macrocode}
% \end{macro}
%
%
% \begin{macro}{\@@_sort_default_different_types:nn}
%   \begin{syntax}
%     \cs{@@_sort_default_different_types:nn} \Arg{label a} \Arg{label b}
%   \end{syntax}
%    \begin{macrocode}
\cs_new_protected:Npn \@@_sort_default_different_types:nn #1#2
  {
%    \end{macrocode}
% Retrieve sort priorities for \meta{label a} and \meta{label b}.
% \cs{l_@@_typesort_seq} was stored in reverse sequence, and we compute the
% sort priorities in the negative range, so that we can implicitly rely on `0'
% being the ``last value''.
%    \begin{macrocode}
    \int_zero:N \l_@@_sort_prior_a_int
    \int_zero:N \l_@@_sort_prior_b_int
    \seq_map_indexed_inline:Nn \l_@@_typesort_seq
      {
        \tl_if_eq:nnTF {##2} {{othertypes}}
          {
            \int_compare:nNnT { \l_@@_sort_prior_a_int } = { 0 }
              { \int_set:Nn \l_@@_sort_prior_a_int { - ##1 } }
            \int_compare:nNnT { \l_@@_sort_prior_b_int } = { 0 }
              { \int_set:Nn \l_@@_sort_prior_b_int { - ##1 } }
          }
          {
            \tl_if_eq:NnTF \l_@@_label_type_a_tl {##2}
              { \int_set:Nn \l_@@_sort_prior_a_int { - ##1 } }
              {
                \tl_if_eq:NnT \l_@@_label_type_b_tl {##2}
                  { \int_set:Nn \l_@@_sort_prior_b_int { - ##1 } }
              }
          }
      }
%    \end{macrocode}
% Then do the actual sorting.
%    \begin{macrocode}
    \bool_if:nTF
      {
        \int_compare_p:nNn
          { \l_@@_sort_prior_a_int } <
          { \l_@@_sort_prior_b_int }
      }
      { \sort_return_same: }
      {
        \bool_if:nTF
          {
            \int_compare_p:nNn
              { \l_@@_sort_prior_a_int } >
              { \l_@@_sort_prior_b_int }
          }
          { \sort_return_swapped: }
          {
            % Sort priorities are equal: the type that occurs first in
            % `labels', as given by the user, is kept (or brought) forward.
            \seq_map_inline:Nn \l_@@_label_types_seq
              {
                \tl_if_eq:NnTF \l_@@_label_type_a_tl {##1}
                  { \seq_map_break:n { \sort_return_same: } }
                  {
                    \tl_if_eq:NnT \l_@@_label_type_b_tl {##1}
                      { \seq_map_break:n { \sort_return_swapped: } }
                  }
              }
          }
      }
  }
%    \end{macrocode}
% \end{macro}
%
%
% \begin{macro}{\@@_sort_page:nn}
%   The sorting function for sorting of defined labels for references to
%   ``page''.  This function is expected to be called within the sorting loop
%   of \cs{@@_sort_labels:} and receives the pair of labels being considered
%   for a change of order or not.  It should \emph{always} ``return'' either
%   \cs{sort_return_same:} or \cs{sort_return_swapped:}.  Compared to the
%   sorting of default labels, this is a piece of cake (thanks to
%   \texttt{abspage}).
%   \begin{syntax}
%     \cs{@@_sort_page:nn} \Arg{label a} \Arg{label b}
%   \end{syntax}
%    \begin{macrocode}
\cs_new_protected:Npn \@@_sort_page:nn #1#2
  {
    \int_compare:nNnTF
      { \@@_extract:nnn {#1} { abspage } { -1 } }
        >
      { \@@_extract:nnn {#2} { abspage } { -1 } }
      { \sort_return_swapped: }
      { \sort_return_same:    }
  }
%    \end{macrocode}
% \end{macro}
%
%
%
% \section{Typesetting}
%
% ``Typesetting'' the reference, which here includes the parsing of the labels
% and eventual compression of labels in sequence into ranges, is definitely
% the ``crux'' of \pkg{zref-clever}.  This because we process the label set as
% a stack, in a single pass, and hence ``parsing'', ``compressing'', and
% ``typesetting'' must be decided upon at the same time, making it difficult
% to slice the job into more specific and self-contained tasks.  So, do bear
% this in mind before you curse me for the length of some of the functions
% below, or before a more orthodox ``docstripper'' complains about me not
% sticking to code commenting conventions to keep the code more readable in
% the \file{.dtx} file.
%
% While processing the label stack (kept in \cs{l_@@_typeset_labels_seq}),
% \cs{@@_typeset_refs:} ``sees'' two labels, and two labels only, the
% ``current'' one (kept in \cs{l_@@_label_a_tl}), and the ``next'' one (kept
% in \cs{l_@@_label_b_tl}).  However, the typesetting needs (a lot) more
% information than just these two immediate labels to make a number of
% critical decisions.  Some examples: i) We cannot know if labels ``current''
% and ``next'' of the same type are a ``pair'', or just ``elements in a
% list'', until we examine the label after ``next''; ii) If the ``next'' label
% is of the same type as the ``current'', and it is in immediate sequence to
% it, it potentially forms a ``range'', but we cannot know if ``next'' is
% actually the end of the range until we examined an arbitrary number of
% labels, and found one which is not in sequence from the previous one; iii)
% When processing a type block, the ``name'' comes first, however, we only
% know if that name should be plural, or if it should be included in the
% hyperlink, after processing an arbitrary number of labels and find one of a
% different type.  One could naively assume that just examining ``next'' would
% be enough for this, since we can know if it is of the same type or not.
% Alas, ``there be ranges'', and a compression operation may boil down to a
% single element, so we have to process the whole type block to know how its
% name should be typeset; iv) Similar issues apply to lists of type blocks,
% each of which is of arbitrary length: we can only know if two type blocks
% form a ``pair'' or are ``elements in a list'' when we finish the
% block. Etc.\ etc.\ etc.
%
% We handle this by storing the reference ``pieces'' in ``queues'', instead of
% typesetting them immediately upon processing.  The ``queues'' get typeset at
% the point where all the information needed is available, which usually
% happens when a type block finishes (we see something of a different type in
% ``next'', signaled by \cs{l_@@_last_of_type_bool}), or the stack itself
% finishes (has no more elements, signaled by \cs{l_@@_typeset_last_bool}).
% And, in processing a type block, the type ``name'' gets added last (on the
% left) of the queue.  The very first reference of its type always follows the
% name, since it may form a hyperlink with it (so we keep it stored
% separately, in \cs{l_@@_type_first_label_tl}, with
% \cs{l_@@_type_first_label_type_tl} being its type).  And, since we may need
% up to two type blocks in storage before typesetting, we have two of these
% ``queues'': \cs{l_@@_typeset_queue_curr_tl} and
% \cs{l_@@_typeset_queue_prev_tl}.
%
% Some of the relevant cases (e.g., distinguishing ``pair'' from ``list'') are
% handled by counters, the main ones are: one for the ``type''
% (\cs{l_@@_type_count_int}) and one for the ``label in the current type
% block'' (\cs{l_@@_label_count_int}).
%
% Range compression, in particular, relies heavily on counting to be able do
% distinguish relevant cases.  \cs{l_@@_range_count_int} counts the number of
% elements in the current sequential ``streak'', and
% \cs{l_@@_range_same_count_int} counts the number of \emph{equal} elements in
% that same ``streak''.  The difference between the two allows us to
% distinguish the cases in which a range actually ``skips'' a number in the
% sequence, in which case we should use a range separator, from when they are
% after all just contiguous, in which case a pair separator is called for.
% Since, as usual, we can only know this when a arbitrary long ``streak''
% finishes, we have to store the label which (potentially) begins a range
% (kept in \cs{l_@@_range_beg_label_tl}).  \cs{l_@@_next_maybe_range_bool}
% signals when ``next'' is potentially a range with ``current'', and
% \cs{l_@@_next_is_same_bool} when their values are actually equal.
%
%
% One further thing to discuss here -- to keep this ``on record'' -- is
% inhibition of compression for individual labels.  It is not difficult to
% handle it at the infrastructure side, what gets sloppy is the user facing
% syntax to signal such inhibition.  For some possible alternatives for this
% (and good ones at that) see \url{https://tex.stackexchange.com/q/611370}
% (thanks Enrico Gregorio, Phelype Oleinik, and Steven B.\ Segletes).  Yet
% another alternative would be an option receiving the label(s) not to be
% compressed, this would be a repetition, but would keep the syntax clean.
% All in all, probably the best is simply not to allow individual inhibition
% of compression.  We can already control compression of each \cs{zcref} call
% with existing options, this should be enough.  I don't think the small extra
% flexibility individual label control for this would grant is worth the
% syntax disruption it would entail.  Anyway, it would be easy to deal with
% this in case the need arose, by just adding another condition (coming from
% whatever the chosen syntax was) when we check for
% \cs{@@_labels_in_sequence:nn} in \cs{@@_typeset_refs_not_last_of_type:}.
% But I remain unconvinced of the pertinence of doing so.
%
%
% \subsection*{Variables}
%
% \begin{variable}
%   {
%     \l_@@_typeset_labels_seq ,
%     \l_@@_typeset_last_bool ,
%     \l_@@_last_of_type_bool ,
%   }
%   Auxiliary variables for \cs{@@_typeset_refs:} main stack control.
%    \begin{macrocode}
\seq_new:N \l_@@_typeset_labels_seq
\bool_new:N \l_@@_typeset_last_bool
\bool_new:N \l_@@_last_of_type_bool
%    \end{macrocode}
% \end{variable}
%
% \begin{variable}
%   {
%     \l_@@_type_count_int ,
%     \l_@@_label_count_int ,
%   }
%   Auxiliary variables for \cs{@@_typeset_refs:} main counters.
%    \begin{macrocode}
\int_new:N \l_@@_type_count_int
\int_new:N \l_@@_label_count_int
%    \end{macrocode}
% \end{variable}
%
% \begin{variable}
%   {
%     \l_@@_label_a_tl ,
%     \l_@@_label_b_tl ,
%     \l_@@_typeset_queue_prev_tl ,
%     \l_@@_typeset_queue_curr_tl ,
%     \l_@@_type_first_label_tl ,
%     \l_@@_type_first_label_type_tl
%   }
%   Auxiliary variables for \cs{@@_typeset_refs:} main ``queue'' control and
%   storage.
%    \begin{macrocode}
\tl_new:N \l_@@_label_a_tl
\tl_new:N \l_@@_label_b_tl
\tl_new:N \l_@@_typeset_queue_prev_tl
\tl_new:N \l_@@_typeset_queue_curr_tl
\tl_new:N \l_@@_type_first_label_tl
\tl_new:N \l_@@_type_first_label_type_tl
%    \end{macrocode}
% \end{variable}
%
% \begin{variable}
%   {
%     \l_@@_type_name_tl ,
%     \l_@@_name_in_link_bool ,
%     \l_@@_name_format_tl ,
%     \l_@@_name_format_fallback_tl ,
%     \l_@@_type_name_gender_tl ,
%   }
%   Auxiliary variables for \cs{@@_typeset_refs:} type name handling.
%    \begin{macrocode}
\tl_new:N \l_@@_type_name_tl
\bool_new:N \l_@@_name_in_link_bool
\tl_new:N \l_@@_name_format_tl
\tl_new:N \l_@@_name_format_fallback_tl
\tl_new:N \l_@@_type_name_gender_tl
%    \end{macrocode}
% \end{variable}
%
% \begin{variable}
%   {
%     \l_@@_range_count_int ,
%     \l_@@_range_same_count_int ,
%     \l_@@_range_beg_label_tl ,
%     \l_@@_next_maybe_range_bool ,
%     \l_@@_next_is_same_bool ,
%   }
%   Auxiliary variables for \cs{@@_typeset_refs:} range handling.
%    \begin{macrocode}
\int_new:N \l_@@_range_count_int
\int_new:N \l_@@_range_same_count_int
\tl_new:N \l_@@_range_beg_label_tl
\bool_new:N \l_@@_next_maybe_range_bool
\bool_new:N \l_@@_next_is_same_bool
%    \end{macrocode}
% \end{variable}
%
% \begin{variable}
%   {
%     \l_@@_tpairsep_tl ,
%     \l_@@_tlistsep_tl ,
%     \l_@@_tlastsep_tl ,
%     \l_@@_namesep_tl ,
%     \l_@@_pairsep_tl ,
%     \l_@@_listsep_tl ,
%     \l_@@_lastsep_tl ,
%     \l_@@_rangesep_tl ,
%     \l_@@_refpre_tl ,
%     \l_@@_refpos_tl ,
%     \l_@@_namefont_tl ,
%     \l_@@_reffont_tl ,
%   }
%   Auxiliary variables for \cs{@@_typeset_refs:} separators, refpre/pos and
%   font options.
%    \begin{macrocode}
\tl_new:N \l_@@_tpairsep_tl
\tl_new:N \l_@@_tlistsep_tl
\tl_new:N \l_@@_tlastsep_tl
\tl_new:N \l_@@_namesep_tl
\tl_new:N \l_@@_pairsep_tl
\tl_new:N \l_@@_listsep_tl
\tl_new:N \l_@@_lastsep_tl
\tl_new:N \l_@@_rangesep_tl
\tl_new:N \l_@@_refpre_tl
\tl_new:N \l_@@_refpos_tl
\tl_new:N \l_@@_namefont_tl
\tl_new:N \l_@@_reffont_tl
%    \end{macrocode}
% \end{variable}
%
%
% \begin{variable}{\l_@@_verbose_testing_bool}
%   Internal variable which enables extra log messaging at points of interest
%   in the code for purposes of regression testing.  Particularly relevant to
%   keep track of expansion control in \cs{l_@@_typeset_queue_curr_tl}.
%    \begin{macrocode}
\bool_new:N \l_@@_verbose_testing_bool
%    \end{macrocode}
% \end{variable}
%
%
% \subsection*{Main functions}
%
% \begin{macro}{\@@_typeset_refs:}
%   Main typesetting function for \cs{zcref}.
%    \begin{macrocode}
\cs_new_protected:Npn \@@_typeset_refs:
  {
    \seq_set_eq:NN \l_@@_typeset_labels_seq
      \l_@@_zcref_labels_seq
    \tl_clear:N \l_@@_typeset_queue_prev_tl
    \tl_clear:N \l_@@_typeset_queue_curr_tl
    \tl_clear:N \l_@@_type_first_label_tl
    \tl_clear:N \l_@@_type_first_label_type_tl
    \tl_clear:N \l_@@_range_beg_label_tl
    \int_zero:N \l_@@_label_count_int
    \int_zero:N \l_@@_type_count_int
    \int_zero:N \l_@@_range_count_int
    \int_zero:N \l_@@_range_same_count_int

    % Get type block options (not type-specific).
    \@@_get_ref_string:nN { tpairsep }
      \l_@@_tpairsep_tl
    \@@_get_ref_string:nN { tlistsep }
      \l_@@_tlistsep_tl
    \@@_get_ref_string:nN { tlastsep }
      \l_@@_tlastsep_tl

    % Process label stack.
    \bool_set_false:N \l_@@_typeset_last_bool
    \bool_until_do:Nn \l_@@_typeset_last_bool
      {
        \seq_pop_left:NN \l_@@_typeset_labels_seq
          \l_@@_label_a_tl
        \seq_if_empty:NTF \l_@@_typeset_labels_seq
          {
            \tl_clear:N \l_@@_label_b_tl
            \bool_set_true:N \l_@@_typeset_last_bool
          }
          {
            \seq_get_left:NN \l_@@_typeset_labels_seq
              \l_@@_label_b_tl
          }

        \tl_if_eq:NnTF \l_@@_ref_property_tl { page }
          {
            \tl_set:Nn \l_@@_label_type_a_tl { page }
            \tl_set:Nn \l_@@_label_type_b_tl { page }
          }
          {
            \@@_def_extract:NVnn \l_@@_label_type_a_tl
              \l_@@_label_a_tl { zc@type } { zc@missingtype }
            \@@_def_extract:NVnn \l_@@_label_type_b_tl
              \l_@@_label_b_tl { zc@type } { zc@missingtype }
          }

        % First, we establish whether the "current label" (i.e. `a') is the
        % last one of its type.  This can happen because the "next label"
        % (i.e. `b') is of a different type (or different definition status),
        % or because we are at the end of the list.
        \bool_if:NTF \l_@@_typeset_last_bool
          { \bool_set_true:N \l_@@_last_of_type_bool }
          {
            \zref@ifrefundefined { \l_@@_label_a_tl }
              {
                \zref@ifrefundefined { \l_@@_label_b_tl }
                  { \bool_set_false:N \l_@@_last_of_type_bool }
                  { \bool_set_true:N \l_@@_last_of_type_bool  }
              }
              {
                \zref@ifrefundefined { \l_@@_label_b_tl }
                  { \bool_set_true:N \l_@@_last_of_type_bool }
                  {
                    % Neither is undefined, we must check the types.
                    \tl_if_eq:NNTF
                      \l_@@_label_type_a_tl
                      \l_@@_label_type_b_tl
                      { \bool_set_false:N \l_@@_last_of_type_bool }
                      { \bool_set_true:N \l_@@_last_of_type_bool  }
                  }
              }
          }

        % Handle warnings in case of reference or type undefined.
        % Test: 'zc-typeset01.lvt': "Typeset refs: warn ref undefined"
        \zref@refused { \l_@@_label_a_tl }
        % Test: 'zc-typeset01.lvt': "Typeset refs: warn missing type"
        \zref@ifrefundefined { \l_@@_label_a_tl }
          {}
          {
            \tl_if_eq:NnT \l_@@_label_type_a_tl { zc@missingtype }
              {
                \msg_warning:nnx { zref-clever } { missing-type }
                  { \l_@@_label_a_tl }
              }
          }

        % Get type-specific separators, refpre/pos and font options, once per
        % type.
        \int_compare:nNnT { \l_@@_label_count_int } = { 0 }
          {
            \@@_get_ref_string:nN { namesep  }
              \l_@@_namesep_tl
            \@@_get_ref_string:nN { pairsep  }
              \l_@@_pairsep_tl
            \@@_get_ref_string:nN { listsep  }
              \l_@@_listsep_tl
            \@@_get_ref_string:nN { lastsep  }
              \l_@@_lastsep_tl
            \@@_get_ref_string:nN { rangesep }
              \l_@@_rangesep_tl
            \@@_get_ref_string:nN { refpre   }
              \l_@@_refpre_tl
            \@@_get_ref_string:nN { refpos   }
              \l_@@_refpos_tl
            \@@_get_ref_font:nN   { namefont }
              \l_@@_namefont_tl
            \@@_get_ref_font:nN   { reffont  }
              \l_@@_reffont_tl
          }

        % Here we send this to a couple of auxiliary functions.
        \bool_if:NTF \l_@@_last_of_type_bool
          % There exists no next label of the same type as the current.
          { \@@_typeset_refs_last_of_type: }
          % There exists a next label of the same type as the current.
          { \@@_typeset_refs_not_last_of_type: }
      }
  }
%    \end{macrocode}
% \end{macro}
%
%
% This is actually the one meaningful ``big branching'' we can do while
% processing the label stack: i) the ``current'' label is the last of its type
% block; or ii) the ``current'' label is \emph{not} the last of its type
% block.  Indeed, as mentioned above, quite a number of things can only be
% decided when the type block ends, and we only know this when we look at the
% ``next'' label and find something of a different ``type'' (loose here, maybe
% different definition status, maybe end of stack).  So, though this is not
% very strict, \cs{@@_typeset_refs_last_of_type:} is more of a ``wrapping
% up'' function, and it is indeed the one which does the actual typesetting,
% while \cs{@@_typeset_refs_not_last_of_type:} is more of an
% ``accumulation'' function.
%
%
% \begin{macro}{\@@_typeset_refs_last_of_type:}
%   Handles typesetting when the current label is the last of its type.
%    \begin{macrocode}
\cs_new_protected:Npn \@@_typeset_refs_last_of_type:
  {
    % Process the current label to the current queue.
    \int_case:nnF { \l_@@_label_count_int }
      {
        % It is the last label of its type, but also the first one, and that's
        % what matters here: just store it.
        % Test: 'zc-typeset01.lvt': "Last of type: single"
        { 0 }
        {
          \tl_set:NV \l_@@_type_first_label_tl
            \l_@@_label_a_tl
          \tl_set:NV \l_@@_type_first_label_type_tl
            \l_@@_label_type_a_tl
        }

        % The last is the second: we have a pair (if not repeated).
        % Test: 'zc-typeset01.lvt': "Last of type: pair"
        { 1 }
        {
          \int_compare:nNnF { \l_@@_range_same_count_int } = { 1 }
            {
              \tl_put_right:Nx \l_@@_typeset_queue_curr_tl
                {
                  \exp_not:V \l_@@_pairsep_tl
                  \@@_get_ref:V \l_@@_label_a_tl
                }
            }
        }
      }
      % Last is third or more of its type: without repetition, we'd have the
      % last element on a list, but control for possible repetition.
      {
        \int_case:nnF { \l_@@_range_count_int }
          {
            % There was no range going on.
            % Test: 'zc-typeset01.lvt': "Last of type: not range"
            { 0 }
            {
              \tl_put_right:Nx \l_@@_typeset_queue_curr_tl
                {
                  \exp_not:V \l_@@_lastsep_tl
                  \@@_get_ref:V \l_@@_label_a_tl
                }
            }
            % Last in the range is also the second in it.
            % Test: 'zc-typeset01.lvt': "Last of type: pair in sequence"
            { 1 }
            {
              \tl_put_right:Nx \l_@@_typeset_queue_curr_tl
                {
                  % We know `range_beg_label' is not empty, since this is the
                  % second element in the range, but the third or more in the
                  % type list.
                  \exp_not:V \l_@@_listsep_tl
                  \@@_get_ref:V \l_@@_range_beg_label_tl
                  \int_compare:nNnF
                    { \l_@@_range_same_count_int } = { 1 }
                    {
                      \exp_not:V \l_@@_lastsep_tl
                      \@@_get_ref:V \l_@@_label_a_tl
                    }
                }
            }
          }
          % Last in the range is third or more in it.
          {
            \int_case:nnF
              {
                \l_@@_range_count_int -
                \l_@@_range_same_count_int
              }
              {
                % Repetition, not a range.
                % Test: 'zc-typeset01.lvt': "Last of type: range to one"
                { 0 }
                {
                  % If `range_beg_label' is empty, it means it was also the
                  % first of the type, and hence was already handled.
                  \tl_if_empty:VF \l_@@_range_beg_label_tl
                    {
                      \tl_put_right:Nx \l_@@_typeset_queue_curr_tl
                        {
                          \exp_not:V \l_@@_lastsep_tl
                          \@@_get_ref:V
                            \l_@@_range_beg_label_tl
                        }
                    }
                }
                % A `range', but with no skipped value, treat as list.
                % Test: 'zc-typeset01.lvt': "Last of type: range to pair"
                { 1 }
                {
                  \tl_put_right:Nx \l_@@_typeset_queue_curr_tl
                    {
                      % Ditto.
                      \tl_if_empty:VF \l_@@_range_beg_label_tl
                        {
                          \exp_not:V \l_@@_listsep_tl
                          \@@_get_ref:V
                            \l_@@_range_beg_label_tl
                        }
                      \exp_not:V \l_@@_lastsep_tl
                      \@@_get_ref:V \l_@@_label_a_tl
                    }
                }
              }
              {
                % An actual range.
                % Test: 'zc-typeset01.lvt': "Last of type: range"
                \tl_put_right:Nx \l_@@_typeset_queue_curr_tl
                  {
                    % Ditto.
                    \tl_if_empty:VF \l_@@_range_beg_label_tl
                      {
                        \exp_not:V \l_@@_lastsep_tl
                        \@@_get_ref:V
                          \l_@@_range_beg_label_tl
                      }
                    \exp_not:V \l_@@_rangesep_tl
                    \@@_get_ref:V \l_@@_label_a_tl
                  }
              }
          }
      }

    % Handle "range" option.  The idea is simple: if the queue is not empty,
    % we replace it with the end of the range (or pair).  We can still
    % retrieve the end of the range from `label_a' since we know to be
    % processing the last label of its type at this point.
    \bool_if:NT \l_@@_typeset_range_bool
      {
        \tl_if_empty:NTF \l_@@_typeset_queue_curr_tl
          {
            \zref@ifrefundefined { \l_@@_type_first_label_tl }
              { }
              {
                \msg_warning:nnx { zref-clever } { single-element-range }
                  { \l_@@_type_first_label_type_tl }
              }
          }
          {
            \bool_set_false:N \l_@@_next_maybe_range_bool
            \zref@ifrefundefined { \l_@@_type_first_label_tl }
              { }
              {
                \@@_labels_in_sequence:nn
                  { \l_@@_type_first_label_tl }
                  { \l_@@_label_a_tl }
              }
            % Test: 'zc-typeset01.lvt': "Last of type: option range"
            % Test: 'zc-typeset01.lvt': "Last of type: option range to pair"
            \tl_set:Nx \l_@@_typeset_queue_curr_tl
              {
                \bool_if:NTF \l_@@_next_maybe_range_bool
                  { \exp_not:V \l_@@_pairsep_tl }
                  { \exp_not:V \l_@@_rangesep_tl }
                \@@_get_ref:V \l_@@_label_a_tl
              }
          }
      }

    % Now that the type block is finished, we can add the name and the first
    % ref to the queue.  Also, if "typeset" option is not "both", handle it
    % here as well.
    \@@_type_name_setup:
    \bool_if:nTF
      { \l_@@_typeset_ref_bool && \l_@@_typeset_name_bool }
      {
        \tl_put_left:Nx \l_@@_typeset_queue_curr_tl
          { \@@_get_ref_first: }
      }
      {
        \bool_if:NTF \l_@@_typeset_ref_bool
          {
            % Test: 'zc-typeset01.lvt': "Last of type: option typeset ref"
            \tl_put_left:Nx \l_@@_typeset_queue_curr_tl
              { \@@_get_ref:V \l_@@_type_first_label_tl }
          }
          {
            \bool_if:NTF \l_@@_typeset_name_bool
              {
                % Test: 'zc-typeset01.lvt': "Last of type: option typeset name"
                \tl_set:Nx \l_@@_typeset_queue_curr_tl
                  {
                    \bool_if:NTF \l_@@_name_in_link_bool
                      {
                        \exp_not:N \group_begin:
                        \exp_not:V \l_@@_namefont_tl
                        % It's two '@s', but escaped for DocStrip.
                        \exp_not:N \hyper@@@@link
                          {
                            \@@_extract_url_unexp:V
                              \l_@@_type_first_label_tl
                          }
                          {
                            \@@_extract_unexp:Vnn
                              \l_@@_type_first_label_tl
                              { anchor } { }
                          }
                          { \exp_not:V \l_@@_type_name_tl }
                        \exp_not:N \group_end:
                      }
                      {
                        \exp_not:N \group_begin:
                        \exp_not:V \l_@@_namefont_tl
                        \exp_not:V \l_@@_type_name_tl
                        \exp_not:N \group_end:
                      }
                  }
              }
              {
                % Logically, this case would correspond to "typeset=none", but
                % it should not occur, given that the options are set up to
                % typeset either "ref" or "name".  Still, leave here a
                % sensible fallback, equal to the behavior of "both".
                % Test: 'zc-typeset01.lvt': "Last of type: option typeset none"
                \tl_put_left:Nx \l_@@_typeset_queue_curr_tl
                  { \@@_get_ref_first: }
              }
          }
      }

    % Typeset the previous type block, if there is one.
    \int_compare:nNnT { \l_@@_type_count_int } > { 0 }
      {
        \int_compare:nNnT { \l_@@_type_count_int } > { 1 }
          { \l_@@_tlistsep_tl }
        \l_@@_typeset_queue_prev_tl
      }

    % Extra log for testing.
    \bool_if:NT \l_@@_verbose_testing_bool
      { \tl_show:N \l_@@_typeset_queue_curr_tl }

    % Wrap up loop, or prepare for next iteration.
    \bool_if:NTF \l_@@_typeset_last_bool
      {
        % We are finishing, typeset the current queue.
        \int_case:nnF { \l_@@_type_count_int }
          {
            % Single type.
            % Test: 'zc-typeset01.lvt': "Last of type: single type"
            { 0 }
            { \l_@@_typeset_queue_curr_tl }
            % Pair of types.
            % Test: 'zc-typeset01.lvt': "Last of type: pair of types"
            { 1 }
            {
              \l_@@_tpairsep_tl
              \l_@@_typeset_queue_curr_tl
            }
          }
          {
            % Last in list of types.
            % Test: 'zc-typeset01.lvt': "Last of type: list of types"
            \l_@@_tlastsep_tl
            \l_@@_typeset_queue_curr_tl
          }
        % And nudge in case of multitype reference.
        \bool_lazy_all:nT
          {
            { \l_@@_nudge_enabled_bool }
            { \l_@@_nudge_multitype_bool }
            { \int_compare_p:nNn { \l_@@_type_count_int } > { 0 } }
          }
          { \msg_warning:nn { zref-clever } { nudge-multitype } }
      }
      {
        % There are further labels, set variables for next iteration.
        \tl_set_eq:NN \l_@@_typeset_queue_prev_tl
          \l_@@_typeset_queue_curr_tl
        \tl_clear:N \l_@@_typeset_queue_curr_tl
        \tl_clear:N \l_@@_type_first_label_tl
        \tl_clear:N \l_@@_type_first_label_type_tl
        \tl_clear:N \l_@@_range_beg_label_tl
        \int_zero:N \l_@@_label_count_int
        \int_incr:N \l_@@_type_count_int
        \int_zero:N \l_@@_range_count_int
        \int_zero:N \l_@@_range_same_count_int
      }
  }
%    \end{macrocode}
% \end{macro}
%
%
% \begin{macro}{\@@_typeset_refs_not_last_of_type:}
%   Handles typesetting when the current label is not the last of its type.
%    \begin{macrocode}
\cs_new_protected:Npn \@@_typeset_refs_not_last_of_type:
  {
    % Signal if next label may form a range with the current one (only
    % considered if compression is enabled in the first place).
    \bool_set_false:N \l_@@_next_maybe_range_bool
    \bool_set_false:N \l_@@_next_is_same_bool
    \bool_if:NT \l_@@_typeset_compress_bool
      {
        \zref@ifrefundefined { \l_@@_label_a_tl }
          { }
          {
            \@@_labels_in_sequence:nn
              { \l_@@_label_a_tl } { \l_@@_label_b_tl }
          }
      }

    % Process the current label to the current queue.
    \int_compare:nNnTF { \l_@@_label_count_int } = { 0 }
      {
        % Current label is the first of its type (also not the last, but it
        % doesn't matter here): just store the label.
        \tl_set:NV \l_@@_type_first_label_tl
          \l_@@_label_a_tl
        \tl_set:NV \l_@@_type_first_label_type_tl
          \l_@@_label_type_a_tl

        % If the next label may be part of a range, we set `range_beg_label'
        % to "empty" (we deal with it as the "first", and must do it there, to
        % handle hyperlinking), but also step the range counters.
        % Test: 'zc-typeset01.lvt': "Not last of type: first is range"
        \bool_if:NT \l_@@_next_maybe_range_bool
          {
            \tl_clear:N \l_@@_range_beg_label_tl
            \int_incr:N \l_@@_range_count_int
            \bool_if:NT \l_@@_next_is_same_bool
              { \int_incr:N \l_@@_range_same_count_int }
          }
      }
      {
        % Current label is neither the first (nor the last) of its type.
        \bool_if:NTF \l_@@_next_maybe_range_bool
          {
            % Starting, or continuing a range.
            \int_compare:nNnTF
              { \l_@@_range_count_int } = { 0 }
              {
                % There was no range going, we are starting one.
                \tl_set:NV \l_@@_range_beg_label_tl
                  \l_@@_label_a_tl
                \int_incr:N \l_@@_range_count_int
                \bool_if:NT \l_@@_next_is_same_bool
                  { \int_incr:N \l_@@_range_same_count_int }
              }
              {
                % Second or more in the range, but not the last.
                \int_incr:N \l_@@_range_count_int
                \bool_if:NT \l_@@_next_is_same_bool
                  { \int_incr:N \l_@@_range_same_count_int }
              }
          }
          {
            % Next element is not in sequence: there was no range, or we are
            % closing one.
            \int_case:nnF { \l_@@_range_count_int }
              {
                % There was no range going on.
                % Test: 'zc-typeset01.lvt': "Not last of type: no range"
                { 0 }
                {
                  \tl_put_right:Nx \l_@@_typeset_queue_curr_tl
                    {
                      \exp_not:V \l_@@_listsep_tl
                      \@@_get_ref:V \l_@@_label_a_tl
                    }
                }
                % Last is second in the range: if `range_same_count' is also
                % `1', it's a repetition (drop it), otherwise, it's a "pair
                % within a list", treat as list.
                % Test: 'zc-typeset01.lvt': "Not last of type: range pair to one"
                % Test: 'zc-typeset01.lvt': "Not last of type: range pair"
                { 1 }
                {
                  \tl_put_right:Nx \l_@@_typeset_queue_curr_tl
                    {
                      \tl_if_empty:VF \l_@@_range_beg_label_tl
                        {
                          \exp_not:V \l_@@_listsep_tl
                          \@@_get_ref:V
                            \l_@@_range_beg_label_tl
                        }
                      \int_compare:nNnF
                        { \l_@@_range_same_count_int } = { 1 }
                        {
                          \exp_not:V \l_@@_listsep_tl
                          \@@_get_ref:V
                            \l_@@_label_a_tl
                        }
                    }
                }
              }
              {
                % Last is third or more in the range: if `range_count' and
                % `range_same_count' are the same, its a repetition (drop it),
                % if they differ by `1', its a list, if they differ by more,
                % it is a real range.
                \int_case:nnF
                  {
                    \l_@@_range_count_int -
                    \l_@@_range_same_count_int
                  }
                  {
                    % Test: 'zc-typeset01.lvt': "Not last of type: range to one"
                    { 0 }
                    {
                      \tl_put_right:Nx \l_@@_typeset_queue_curr_tl
                        {
                          \tl_if_empty:VF \l_@@_range_beg_label_tl
                            {
                              \exp_not:V \l_@@_listsep_tl
                              \@@_get_ref:V
                                \l_@@_range_beg_label_tl
                            }
                        }
                    }
                    % Test: 'zc-typeset01.lvt': "Not last of type: range to pair"
                    { 1 }
                    {
                      \tl_put_right:Nx \l_@@_typeset_queue_curr_tl
                        {
                          \tl_if_empty:VF \l_@@_range_beg_label_tl
                            {
                              \exp_not:V \l_@@_listsep_tl
                              \@@_get_ref:V
                                \l_@@_range_beg_label_tl
                            }
                          \exp_not:V \l_@@_listsep_tl
                          \@@_get_ref:V \l_@@_label_a_tl
                        }
                    }
                  }
                  {
                    % Test: 'zc-typeset01.lvt': "Not last of type: range"
                    \tl_put_right:Nx \l_@@_typeset_queue_curr_tl
                      {
                        \tl_if_empty:VF \l_@@_range_beg_label_tl
                          {
                            \exp_not:V \l_@@_listsep_tl
                            \@@_get_ref:V
                              \l_@@_range_beg_label_tl
                          }
                        \exp_not:V \l_@@_rangesep_tl
                        \@@_get_ref:V \l_@@_label_a_tl
                      }
                  }
              }
            % Reset counters.
            \int_zero:N \l_@@_range_count_int
            \int_zero:N \l_@@_range_same_count_int
          }
      }
    % Step label counter for next iteration.
    \int_incr:N \l_@@_label_count_int
  }
%    \end{macrocode}
% \end{macro}
%
%
%
% \subsection*{Aux functions}
%
% \cs{@@_get_ref:n} and \cs{@@_get_ref_first:} are the two functions which
% actually build the reference blocks for typesetting.  \cs{@@_get_ref:n}
% handles all references but the first of its type, and \cs{@@_get_ref_first:}
% deals with the first reference of a type.  Saying they do ``typesetting'' is
% imprecise though, they actually prepare material to be accumulated in
% \cs{l_@@_typeset_queue_curr_tl} inside \cs{@@_typeset_refs_last_of_type:}
% and \cs{@@_typeset_refs_not_last_of_type:}.  And this difference results
% quite crucial for the \TeX{}nical requirements of these functions.  This
% because, as we are processing the label stack and accumulating content in
% the queue, we are using a number of variables which are transient to the
% current label, the label properties among them, but not only.  Hence, these
% variables \emph{must} be expanded to their current values to be stored in
% the queue.  Indeed, \cs{@@_get_ref:n} and \cs{@@_get_ref_first:} get called,
% as they must, in the context of \texttt{x} type expansions.  But we don't
% want to expand the values of the variables themselves, so we need to get
% current values, but stop expansion after that.  In particular, reference
% options given by the user should reach the stream for its final typesetting
% (when the queue itself gets typeset) \emph{unmodified} (``no manipulation'',
% to use the \texttt{n} signature jargon).  We also need to prevent premature
% expansion of material that can't be expanded at this point (e.g. grouping,
% \cs{zref@default} or \cs{hyper@@@@link}).  In a nutshell, the job of these
% two functions is putting the pieces in place, but with proper expansion
% control.
%
%
% \begin{macro}{\@@_ref_default:, \@@_name_default:}
%   Default values for undefined references and undefined type names,
%   respectively.  We are ultimately using \cs{zref@default}, but calls to it
%   should be made through these internal functions, according to the case.
%   As a bonus, we don't need to protect them with \cs{exp_not:N}, as
%   \cs{zref@default} would require, since we already define them protected.
%    \begin{macrocode}
\cs_new_protected:Npn \@@_ref_default:
  { \zref@default }
\cs_new_protected:Npn \@@_name_default:
  { \zref@default }
%    \end{macrocode}
% \end{macro}
%
%
% \begin{macro}{\@@_get_ref:n}
%   Handles a complete reference block to be accumulated in the ``queue'',
%   including ``pre'' and ``pos'' elements, and hyperlinking.  For use with
%   all labels, except the first of its type, which is done by
%   \cs{@@_get_ref_first:}.
%   \begin{syntax}
%     \cs{@@_get_ref:n} \Arg{label}
%   \end{syntax}
%    \begin{macrocode}
\cs_new:Npn \@@_get_ref:n #1
  {
    \zref@ifrefcontainsprop {#1} { \l_@@_ref_property_tl }
      {
        \bool_if:nTF
          {
            \l_@@_use_hyperref_bool &&
            ! \l_@@_link_star_bool
          }
          {
            \bool_if:NF \l_@@_preposinlink_bool
              { \exp_not:V \l_@@_refpre_tl }
            % It's two `@s', but escaped for DocStrip.
            \exp_not:N \hyper@@@@link
              { \@@_extract_url_unexp:n {#1} }
              { \@@_extract_unexp:nnn {#1} { anchor } { } }
              {
                \bool_if:NT \l_@@_preposinlink_bool
                  { \exp_not:V \l_@@_refpre_tl }
                \exp_not:N \group_begin:
                \exp_not:V \l_@@_reffont_tl
                \@@_extract_unexp:nvn {#1}
                  { l_@@_ref_property_tl } { }
                \exp_not:N \group_end:
                \bool_if:NT \l_@@_preposinlink_bool
                  { \exp_not:V \l_@@_refpos_tl }
              }
            \bool_if:NF \l_@@_preposinlink_bool
              { \exp_not:V \l_@@_refpos_tl }
          }
          {
            \exp_not:V \l_@@_refpre_tl
            \exp_not:N \group_begin:
            \exp_not:V \l_@@_reffont_tl
            \@@_extract_unexp:nvn {#1}
              { l_@@_ref_property_tl } { }
            \exp_not:N \group_end:
            \exp_not:V \l_@@_refpos_tl
          }
      }
      { \@@_ref_default: }
  }
\cs_generate_variant:Nn \@@_get_ref:n { V }
%    \end{macrocode}
% \end{macro}
%
% \begin{macro}{\@@_get_ref_first:}
%   Handles a complete reference block for the first label of its type to be
%   accumulated in the ``queue'', including ``pre'' and ``pos'' elements,
%   hyperlinking, and the reference type ``name''.  It does not receive
%   arguments, but relies on being called in the appropriate place in
%   \cs{@@_typeset_refs_last_of_type:} where a number of variables are
%   expected to be appropriately set for it to consume.  Prominently among
%   those is \cs{l_@@_type_first_label_tl}, but it also expected to be called
%   right after \cs{@@_type_name_setup:} which sets \cs{l_@@_type_name_tl} and
%   \cs{l_@@_name_in_link_bool} which it uses.
%    \begin{macrocode}
\cs_new:Npn \@@_get_ref_first:
  {
    \zref@ifrefundefined { \l_@@_type_first_label_tl }
      { \@@_ref_default: }
      {
        \bool_if:NTF \l_@@_name_in_link_bool
          {
            \zref@ifrefcontainsprop
              { \l_@@_type_first_label_tl }
              { \l_@@_ref_property_tl }
              {
                % It's two `@s', but escaped for DocStrip.
                \exp_not:N \hyper@@@@link
                  {
                    \@@_extract_url_unexp:V
                      \l_@@_type_first_label_tl
                  }
                  {
                    \@@_extract_unexp:Vnn
                      \l_@@_type_first_label_tl { anchor } { }
                  }
                  {
                    \exp_not:N \group_begin:
                    \exp_not:V \l_@@_namefont_tl
                    \exp_not:V \l_@@_type_name_tl
                    \exp_not:N \group_end:
                    \exp_not:V \l_@@_namesep_tl
                    \exp_not:V \l_@@_refpre_tl
                    \exp_not:N \group_begin:
                    \exp_not:V \l_@@_reffont_tl
                    \@@_extract_unexp:Vvn
                      \l_@@_type_first_label_tl
                      { l_@@_ref_property_tl } { }
                    \exp_not:N \group_end:
                    \bool_if:NT \l_@@_preposinlink_bool
                      { \exp_not:V \l_@@_refpos_tl }
                  }
                \bool_if:NF \l_@@_preposinlink_bool
                  { \exp_not:V \l_@@_refpos_tl }
              }
              {
                \exp_not:N \group_begin:
                \exp_not:V \l_@@_namefont_tl
                \exp_not:V \l_@@_type_name_tl
                \exp_not:N \group_end:
                \exp_not:V \l_@@_namesep_tl
                \@@_ref_default:
              }
          }
          {
            \tl_if_empty:NTF \l_@@_type_name_tl
              {
                \@@_name_default:
                \exp_not:V \l_@@_namesep_tl
              }
              {
                \exp_not:N \group_begin:
                \exp_not:V \l_@@_namefont_tl
                \exp_not:V \l_@@_type_name_tl
                \exp_not:N \group_end:
                \exp_not:V \l_@@_namesep_tl
              }
            \zref@ifrefcontainsprop
              { \l_@@_type_first_label_tl }
              { \l_@@_ref_property_tl }
              {
                \bool_if:nTF
                  {
                    \l_@@_use_hyperref_bool &&
                    ! \l_@@_link_star_bool
                  }
                  {
                    \bool_if:NF \l_@@_preposinlink_bool
                      { \exp_not:V \l_@@_refpre_tl }
                    % It's two '@s', but escaped for DocStrip.
                    \exp_not:N \hyper@@@@link
                      {
                        \@@_extract_url_unexp:V
                          \l_@@_type_first_label_tl
                      }
                      {
                        \@@_extract_unexp:Vnn
                          \l_@@_type_first_label_tl { anchor } { }
                      }
                      {
                        \bool_if:NT \l_@@_preposinlink_bool
                          { \exp_not:V \l_@@_refpre_tl }
                        \exp_not:N \group_begin:
                        \exp_not:V \l_@@_reffont_tl
                        \@@_extract_unexp:Vvn
                          \l_@@_type_first_label_tl
                          { l_@@_ref_property_tl } { }
                        \exp_not:N \group_end:
                        \bool_if:NT \l_@@_preposinlink_bool
                          { \exp_not:V \l_@@_refpos_tl }
                      }
                    \bool_if:NF \l_@@_preposinlink_bool
                      { \exp_not:V \l_@@_refpos_tl }
                  }
                  {
                    \exp_not:V \l_@@_refpre_tl
                    \exp_not:N \group_begin:
                    \exp_not:V \l_@@_reffont_tl
                    \@@_extract_unexp:Vvn
                      \l_@@_type_first_label_tl
                      { l_@@_ref_property_tl } { }
                    \exp_not:N \group_end:
                    \exp_not:V \l_@@_refpos_tl
                  }
              }
              { \@@_ref_default: }
          }
      }
  }
%    \end{macrocode}
% \end{macro}
%
% \begin{macro}{\@@_type_name_setup:}
%   Auxiliary function to \cs{@@_typeset_refs_last_of_type:}.  It is
%   responsible for setting the type name variable \cs{l_@@_type_name_tl} and
%   \cs{l_@@_name_in_link_bool}.  If a type name can't be found,
%   \cs{l_@@_type_name_tl} is cleared.  The function takes no arguments, but
%   is expected to be called in \cs{@@_typeset_refs_last_of_type:} right
%   before \cs{@@_get_ref_first:}, which is the main consumer of the variables
%   it sets, though not the only one (and hence this cannot be moved into
%   \cs{@@_get_ref_first:} itself).  It also expects a number of relevant
%   variables to have been appropriately set, and which it uses, prominently
%   \cs{l_@@_type_first_label_type_tl}, but also the queue itself in
%   \cs{l_@@_typeset_queue_curr_tl}, which should be ``ready except for the
%   first label'', and the type counter \cs{l_@@_type_count_int}.
%    \begin{macrocode}
\cs_new_protected:Npn \@@_type_name_setup:
  {
    \zref@ifrefundefined { \l_@@_type_first_label_tl }
      { \tl_clear:N \l_@@_type_name_tl }
      {
        \tl_if_eq:NnTF
          \l_@@_type_first_label_type_tl { zc@missingtype }
          { \tl_clear:N \l_@@_type_name_tl }
          {
            % Determine whether we should use capitalization, abbreviation,
            % and plural.
            \bool_lazy_or:nnTF
              { \l_@@_capitalize_bool }
              {
                \l_@@_capitalize_first_bool &&
                \int_compare_p:nNn { \l_@@_type_count_int } = { 0 }
              }
              { \tl_set:Nn \l_@@_name_format_tl {Name} }
              { \tl_set:Nn \l_@@_name_format_tl {name} }
            % If the queue is empty, we have a singular, otherwise, plural.
            \tl_if_empty:NTF \l_@@_typeset_queue_curr_tl
              { \tl_put_right:Nn \l_@@_name_format_tl { -sg } }
              { \tl_put_right:Nn \l_@@_name_format_tl { -pl } }
            \bool_lazy_and:nnTF
              { \l_@@_abbrev_bool }
              {
                ! \int_compare_p:nNn
                    { \l_@@_type_count_int } = { 0 } ||
                ! \l_@@_noabbrev_first_bool
              }
              {
                \tl_set:NV \l_@@_name_format_fallback_tl
                  \l_@@_name_format_tl
                \tl_put_right:Nn \l_@@_name_format_tl { -ab }
              }
              { \tl_clear:N \l_@@_name_format_fallback_tl }

            % Handle number and gender nudges.
            \bool_if:NT \l_@@_nudge_enabled_bool
              {
                \bool_if:NTF \l_@@_nudge_singular_bool
                  {
                    \tl_if_empty:NF \l_@@_typeset_queue_curr_tl
                      {
                        \msg_warning:nnx { zref-clever }
                          { nudge-plural-when-sg }
                          { \l_@@_type_first_label_type_tl }
                      }
                  }
                  {
                    \bool_lazy_all:nT
                      {
                        { \l_@@_nudge_comptosing_bool }
                        { \tl_if_empty_p:N \l_@@_typeset_queue_curr_tl }
                        {
                          \int_compare_p:nNn
                            { \l_@@_label_count_int } > { 0 }
                        }
                      }
                      {
                        \msg_warning:nnx { zref-clever }
                          { nudge-comptosing }
                          { \l_@@_type_first_label_type_tl }
                      }
                  }
                \bool_lazy_and:nnT
                  { \l_@@_nudge_gender_bool }
                  { ! \tl_if_empty_p:N \l_@@_ref_gender_tl }
                  {
                    \@@_get_type_transl:xxnNF
                      { \l_@@_ref_language_tl }
                      { \l_@@_type_first_label_type_tl }
                      { gender }
                      \l_@@_type_name_gender_tl
                      { \tl_clear:N \l_@@_type_name_gender_tl }
                    \tl_if_eq:NNF
                      \l_@@_ref_gender_tl
                      \l_@@_type_name_gender_tl
                      {
                        \tl_if_empty:NTF \l_@@_type_name_gender_tl
                          {
                            \msg_warning:nnxxx { zref-clever }
                              { nudge-gender-not-declared-for-type }
                              { \l_@@_ref_gender_tl }
                              { \l_@@_type_first_label_type_tl }
                              { \l_@@_ref_language_tl }
                          }
                          {
                            \msg_warning:nnxxxx { zref-clever }
                              { nudge-gender-mismatch }
                              { \l_@@_type_first_label_type_tl }
                              { \l_@@_ref_gender_tl }
                              { \l_@@_type_name_gender_tl }
                              { \l_@@_ref_language_tl }
                          }
                      }
                  }
              }

            \tl_if_empty:NTF \l_@@_name_format_fallback_tl
              {
                \prop_get:cVNF
                  {
                    l_@@_type_
                    \l_@@_type_first_label_type_tl _options_prop
                  }
                  \l_@@_name_format_tl
                  \l_@@_type_name_tl
                  {
                    \tl_if_empty:NF \l_@@_ref_decl_case_tl
                      {
                        \tl_put_left:Nn \l_@@_name_format_tl { - }
                        \tl_put_left:NV \l_@@_name_format_tl
                          \l_@@_ref_decl_case_tl
                      }
                    \@@_get_type_transl:xxxNF
                      { \l_@@_ref_language_tl }
                      { \l_@@_type_first_label_type_tl }
                      { \l_@@_name_format_tl }
                      \l_@@_type_name_tl
                      {
                        \tl_clear:N \l_@@_type_name_tl
                        \msg_warning:nnxx { zref-clever } { missing-name }
                          { \l_@@_name_format_tl }
                          { \l_@@_type_first_label_type_tl }
                      }
                  }
              }
              {
                \prop_get:cVNF
                  {
                    l_@@_type_
                    \l_@@_type_first_label_type_tl _options_prop
                  }
                  \l_@@_name_format_tl
                  \l_@@_type_name_tl
                  {
                    \prop_get:cVNF
                      {
                        l_@@_type_
                        \l_@@_type_first_label_type_tl _options_prop
                      }
                      \l_@@_name_format_fallback_tl
                      \l_@@_type_name_tl
                      {
                        \tl_if_empty:NF \l_@@_ref_decl_case_tl
                          {
                            \tl_put_left:Nn
                              \l_@@_name_format_tl { - }
                            \tl_put_left:NV \l_@@_name_format_tl
                              \l_@@_ref_decl_case_tl
                            \tl_put_left:Nn
                              \l_@@_name_format_fallback_tl { - }
                            \tl_put_left:NV
                              \l_@@_name_format_fallback_tl
                              \l_@@_ref_decl_case_tl
                          }
                        \@@_get_type_transl:xxxNF
                          { \l_@@_ref_language_tl }
                          { \l_@@_type_first_label_type_tl }
                          { \l_@@_name_format_tl }
                          \l_@@_type_name_tl
                          {
                            \@@_get_type_transl:xxxNF
                              { \l_@@_ref_language_tl }
                              { \l_@@_type_first_label_type_tl }
                              { \l_@@_name_format_fallback_tl }
                              \l_@@_type_name_tl
                              {
                                \tl_clear:N \l_@@_type_name_tl
                                \msg_warning:nnxx { zref-clever }
                                  { missing-name }
                                  { \l_@@_name_format_tl }
                                  { \l_@@_type_first_label_type_tl }
                              }
                          }
                      }
                  }
              }
          }
      }

    % Signal whether the type name is to be included in the hyperlink or not.
    \bool_lazy_any:nTF
      {
        { ! \l_@@_use_hyperref_bool }
        { \l_@@_link_star_bool }
        { \tl_if_empty_p:N \l_@@_type_name_tl }
        { \str_if_eq_p:Vn \l_@@_nameinlink_str { false } }
      }
      { \bool_set_false:N \l_@@_name_in_link_bool }
      {
        \bool_lazy_any:nTF
          {
            { \str_if_eq_p:Vn \l_@@_nameinlink_str { true } }
            {
              \str_if_eq_p:Vn \l_@@_nameinlink_str { tsingle } &&
              \tl_if_empty_p:N \l_@@_typeset_queue_curr_tl
            }
            {
              \str_if_eq_p:Vn \l_@@_nameinlink_str { single } &&
              \tl_if_empty_p:N \l_@@_typeset_queue_curr_tl &&
              \l_@@_typeset_last_bool &&
              \int_compare_p:nNn { \l_@@_type_count_int } = { 0 }
            }
          }
          { \bool_set_true:N \l_@@_name_in_link_bool }
          { \bool_set_false:N \l_@@_name_in_link_bool }
      }
  }
%    \end{macrocode}
% \end{macro}
%
%
% \begin{macro}{\@@_extract_url_unexp:n}
%   A convenience auxiliary function for extraction of the \texttt{url} /
%   \texttt{urluse} property, provided by the \pkg{zref-xr} module.  Ensure
%   that, in the context of an x expansion, \cs{zref@extractdefault} is
%   expanded exactly twice, but no further to retrieve the proper value.  See
%   documentation for \cs{@@_extract_unexp:nnn}.
%    \begin{macrocode}
\cs_new:Npn \@@_extract_url_unexp:n #1
  {
    \zref@ifpropundefined { urluse }
      { \@@_extract_unexp:nnn {#1} { url } { \c_empty_tl } }
      {
        \zref@ifrefcontainsprop {#1} { urluse }
          { \@@_extract_unexp:nnn {#1} { urluse } { \c_empty_tl } }
          { \@@_extract_unexp:nnn {#1} { url } { \c_empty_tl } }
      }
  }
\cs_generate_variant:Nn \@@_extract_url_unexp:n { V }
%    \end{macrocode}
% \end{macro}
%
%
% \begin{macro}{\@@_labels_in_sequence:nn}
%   Auxiliary function to \cs{@@_typeset_refs_not_last_of_type:}. Sets
%   \cs{l_@@_next_maybe_range_bool} to true if \meta{label b} comes in
%   immediate sequence from \meta{label a}.  And sets both
%   \cs{l_@@_next_maybe_range_bool} and \cs{l_@@_next_is_same_bool} to true if
%   the two labels are the ``same'' (that is, have the same counter value).
%   These two boolean variables are the basis for all range and compression
%   handling inside \cs{@@_typeset_refs_not_last_of_type:}, so this function
%   is expected to be called at its beginning, if compression is enabled.
%   \begin{syntax}
%     \cs{@@_labels_in_sequence:nn} \Arg{label a} \Arg{label b}
%   \end{syntax}
%    \begin{macrocode}
\cs_new_protected:Npn \@@_labels_in_sequence:nn #1#2
  {
    \@@_def_extract:Nnnn \l_@@_label_extdoc_a_tl
      {#1} { externaldocument } { \c_empty_tl }
    \@@_def_extract:Nnnn \l_@@_label_extdoc_b_tl
      {#2} { externaldocument } { \c_empty_tl }

    \tl_if_eq:NNT
      \l_@@_label_extdoc_a_tl
      \l_@@_label_extdoc_b_tl
      {
        \tl_if_eq:NnTF \l_@@_ref_property_tl { page }
          {
            \exp_args:Nxx \tl_if_eq:nnT
              { \@@_extract_unexp:nnn {#1} { zc@pgfmt } { } }
              { \@@_extract_unexp:nnn {#2} { zc@pgfmt } { } }
              {
                \int_compare:nNnTF
                  { \@@_extract:nnn {#1} { zc@pgval } { -2 } + 1 }
                    =
                  { \@@_extract:nnn {#2} { zc@pgval } { -1 } }
                  { \bool_set_true:N \l_@@_next_maybe_range_bool }
                  {
                    \int_compare:nNnT
                      { \@@_extract:nnn {#1} { zc@pgval } { -1 } }
                        =
                      { \@@_extract:nnn {#2} { zc@pgval } { -1 } }
                      {
                        \bool_set_true:N \l_@@_next_maybe_range_bool
                        \bool_set_true:N \l_@@_next_is_same_bool
                      }
                  }
              }
          }
          {
            \exp_args:Nxx \tl_if_eq:nnT
              { \@@_extract_unexp:nnn {#1} { zc@counter } { } }
              { \@@_extract_unexp:nnn {#2} { zc@counter } { } }
              {
                \exp_args:Nxx \tl_if_eq:nnT
                  { \@@_extract_unexp:nnn {#1} { zc@enclval } { } }
                  { \@@_extract_unexp:nnn {#2} { zc@enclval } { } }
                  {
                    \int_compare:nNnTF
                      { \@@_extract:nnn {#1} { zc@cntval } { -2 } + 1 }
                        =
                      { \@@_extract:nnn {#2} { zc@cntval } { -1 } }
                      { \bool_set_true:N \l_@@_next_maybe_range_bool }
                      {
                        \int_compare:nNnT
                          { \@@_extract:nnn {#1} { zc@cntval } { -1 } }
                            =
                          { \@@_extract:nnn {#2} { zc@cntval } { -1 } }
                          {
                            \bool_set_true:N
                              \l_@@_next_maybe_range_bool
                            \exp_args:Nxx \tl_if_eq:nnT
                              {
                                \@@_extract_unexp:nvn {#1}
                                  { l_@@_ref_property_tl } { }
                              }
                              {
                                \@@_extract_unexp:nvn {#2}
                                  { l_@@_ref_property_tl } { }
                              }
                              {
                                \bool_set_true:N
                                  \l_@@_next_is_same_bool
                              }
                          }
                      }
                  }
              }
          }
      }
  }
%    \end{macrocode}
% \end{macro}
%
%
%
% Finally, a couple of functions for retrieving options values, according to
% the relevant precedence rules.  They both receive an \meta{option} as
% argument, and store the retrieved value in \meta{tl variable}.  Though these
% are mostly general functions (for a change\dots{}), they are not completely
% so, they rely on the current state of \cs{l_@@_label_type_a_tl}, as set
% during the processing of the label stack.  This could be easily generalized,
% of course, but I don't think it is worth it, \cs{l_@@_label_type_a_tl} is
% indeed what we want in all practical cases.  The difference between
% \cs{@@_get_ref_string:nN} and \cs{@@_get_ref_font:nN} is the kind of option
% each should be used for.  \cs{@@_get_ref_string:nN} is meant for the general
% options, and attempts to find values for them in all precedence levels (four
% plus ``fallback'').  \cs{@@_get_ref_font:nN} is intended for ``font''
% options, which cannot be ``language-specific'', thus for these we just
% search general options and type options.
%
% \begin{macro}{\@@_get_ref_string:nN}
%   \begin{syntax}
%     \cs{@@_get_ref_string:nN} \Arg{option} \Arg{tl variable}
%   \end{syntax}
%    \begin{macrocode}
\cs_new_protected:Npn \@@_get_ref_string:nN #1#2
  {
    % First attempt: general options.
    \prop_get:NnNF \l_@@_ref_options_prop {#1} #2
      {
        % If not found, try type specific options.
        \bool_lazy_and:nnTF
          {
            \prop_if_exist_p:c
              {
                l_@@_type_
                \l_@@_label_type_a_tl _options_prop
              }
          }
          {
            \prop_if_in_p:cn
              {
                l_@@_type_
                \l_@@_label_type_a_tl _options_prop
              }
              {#1}
          }
          {
            \prop_get:cnN
              {
                l_@@_type_
                \l_@@_label_type_a_tl _options_prop
              }
              {#1} #2
          }
          {
            % If not found, try type specific translations.
            \@@_get_type_transl:xxnNF
              { \l_@@_ref_language_tl }
              { \l_@@_label_type_a_tl }
              {#1} #2
              {
                % If not found, try default translations.
                \@@_get_default_transl:xnNF
                  { \l_@@_ref_language_tl }
                  {#1} #2
                  {
                    % If not found, try fallback.
                    \@@_get_fallback_transl:nNF {#1} #2
                      {
                        \tl_clear:N #2
                        \msg_warning:nnn { zref-clever }
                          { missing-string } {#1}
                      }
                  }
              }
          }
      }
  }
%    \end{macrocode}
% \end{macro}
%
% \begin{macro}{\@@_get_ref_font:nN}
%   \begin{syntax}
%     \cs{@@_get_ref_font:nN} \Arg{option} \Arg{tl variable}
%   \end{syntax}
%    \begin{macrocode}
\cs_new_protected:Npn \@@_get_ref_font:nN #1#2
  {
    % First attempt: general options.
    \prop_get:NnNF \l_@@_ref_options_prop {#1} #2
      {
        % If not found, try type specific options.
        \bool_if:nTF
          {
            \prop_if_exist_p:c
              {
                l_@@_type_
                \l_@@_label_type_a_tl _options_prop
              }
          }
          {
            \prop_get:cnNF
              {
                l_@@_type_
                \l_@@_label_type_a_tl _options_prop
              }
              {#1} #2
              { \tl_clear:N #2 }
          }
          { \tl_clear:N #2 }
      }
  }
%    \end{macrocode}
% \end{macro}
%
%
%
% \section{Compatibility}
%
% This section is meant to aggregate any ``special handling'' needed for
% \LaTeX{} kernel features, document classes, and packages, needed for
% \pkg{zref-clever} to work properly with them.
%
%
% \subsection{\cs{footnote}}
%
% I'd love not to have to tamper with the \cs{footnote}'s machinery\dots{}
% However, it is too basic a feature not to work out-of-the-box and,
% unfortunately, it neither uses \cs{refstepcounter} nor sets
% \cs{@currentcounter}.  So there's really not much to do here except trust in
% the new hook management system.
%
% I have made a feature request though, for having \cs{@currentcounter}
% recorded there too: \url{https://github.com/latex3/latex2e/issues/687}.
%
% TODO The FR has been implemented, remove this when the new release comes.
%
%    \begin{macrocode}
\IfFormatAtLeastTF { 2021-11-15 }
  { }
  {
    \tl_new:N \l_@@_footnote_type_tl
    \tl_set:Nn \l_@@_footnote_type_tl { footnote }
    \AddToHook { env / minipage / begin }
      { \tl_set:Nn \l_@@_footnote_type_tl { mpfootnote } }
    \AddToHook { cmd / @makefntext / before }
      {
        \@@_zcsetup:x
          { currentcounter = \l_@@_footnote_type_tl }
      }
  }
%    \end{macrocode}
%
%
%
% \subsection{\cs{appendix}}
%
% One relevant case of different reference types sharing the same counter is
% the \cs{appendix} which in some document classes, including the standard
% ones, change the sectioning commands looks but, of course, keep using the
% same counter.  \file{book.cls} and \file{report.cls} reset counters
% \texttt{chapter} and \texttt{section} to 0, change \cs{@chapapp} to use
% \cs{appendixname} and use \cs{@Alph} for \cs{thechapter}. \file{article.cls}
% resets counters \texttt{section} and \texttt{subsection} to 0, and uses
% \cs{@Alph} for \cs{thesection}.  \file{memoir.cls}, \file{scrbook.cls} and
% \file{scrarticle.cls} do the same as their corresponding standard classes,
% and sometimes a little more, but what interests us here is pretty much the
% same.  See also the \pkg{appendix} package.
%
% The standard \cs{appendix} command is a one way switch, in other words, it
% cannot be reverted (see \url{https://tex.stackexchange.com/a/444057}).  So,
% even if the fact that it is a ``switch'' rather than an environment
% complicates things, because we have to make ungrouped settings to correspond
% to its effects, in practice this is not a big deal, since these settings are
% never really reverted (by default, at least).  Hence, hooking into
% \cs{appendix} is a viable and natural alternative.  The \cls{memoir} class
% and the \pkg{appendix} package define the \texttt{appendices} and
% \texttt{subappendices} environments, which provide for a way for the
% appendix to ``end'', but in this case, of course, we can hook into the
% environment instead.
%
%    \begin{macrocode}
\AddToHook { cmd / appendix / before }
  {
    \@@_zcsetup:n
      {
        countertype =
          {
            chapter       = appendix ,
            section       = appendix ,
            subsection    = appendix ,
            subsubsection = appendix ,
          }
      }
  }
%    \end{macrocode}
%
% Depending on the definition of \cs{appendix}, using the hook may lead to
% trouble with the first released version of \pkg{ltcmdhooks} (the one
% released with the 2021-06-01 kernel).  Particularly, if the definition of
% the command being hooked at contains a double hash mark (\texttt{\#\#}) the
% patch to add the hook, if it needs to be done with the \cs{scantokens}
% method, may fail noisily (see \url{https://tex.stackexchange.com/q/617905},
% thanks Phelype Oleinik).  The 2021-11-15 kernel release already handle this
% gracefully (see \url{https://github.com/latex3/latex2e/pull/699}, thanks
% Phelype Oleinik).  In the meantime, given we cannot really expect to know
% what \cs{appendix} may contain in general, since it potentially gets
% redefined in quite a number of classes and packages, a user facing
% workaround may be needed in case of trouble.  Phelype Oleinik recommends
% activating/providing the generic hook in question, so that \pkg{ltcmdhooks}
% considers the patch as already done, and do the patch ourselves with
% \pkg{etoolbox} (\url{https://tex.stackexchange.com/a/617998}).  Like so:
%
% \begin{verbatim}
%   \IfFormatAtLeastTF{2021-11-15}%
%     {\ActivateGenericHook}%
%     {\ProvideHook}%
%       {cmd/appendix/before}
%   \usepackage{etoolbox}
%   \pretocmd\appendix
%     {\UseHook{cmd/appendix/before}}
%     {}{\FAILED}
% \end{verbatim}
%
%
%
% \subsection{\pkg{appendix} package}
%
% These settings also apply to the \cls{memoir} class, since it ``emulates''
% the loading of the \pkg{appendix} package.
%
%    \begin{macrocode}
\AddToHook { begindocument }
  {
    \@ifpackageloaded { appendix }
      {
        \newcounter { zc@appendix }
        \newcounter { zc@save@appendix }
        \setcounter { zc@appendix } { 0 }
        \setcounter { zc@save@appendix } { 0 }
        \cs_if_exist:cTF { chapter }
          {
            \@@_zcsetup:n
              { counterresetby = { chapter = zc@appendix } }
          }
          {
            \cs_if_exist:cT { section }
              {
                \@@_zcsetup:n
                  { counterresetby = { section = zc@appendix } }
              }
          }
        \AddToHook { env / appendices / begin }
          {
            \stepcounter { zc@save@appendix }
            \setcounter { zc@appendix } { \value { zc@save@appendix } }
            \@@_zcsetup:n
              {
                countertype =
                  {
                    chapter       = appendix ,
                    section       = appendix ,
                    subsection    = appendix ,
                    subsubsection = appendix ,
                  }
              }
          }
        \AddToHook { env / appendices / end }
          { \setcounter { zc@appendix } { 0 } }
        \AddToHook { cmd / appendix / before }
          {
            \stepcounter { zc@save@appendix }
            \setcounter { zc@appendix } { \value { zc@save@appendix } }
          }
        \AddToHook { env / subappendices / begin }
          {
            \@@_zcsetup:n
              {
                countertype =
                  {
                    section       = appendix ,
                    subsection    = appendix ,
                    subsubsection = appendix ,
                  } ,
              }
          }
        \msg_info:nnn { zref-clever } { compat-package } { appendix }
      }
      {}
  }
%    \end{macrocode}
%
%
%
% \subsection{\pkg{amsmath} package}
%
% About this, see \url{https://tex.stackexchange.com/a/402297}.
%
%    \begin{macrocode}
\AddToHook { begindocument }
  {
    \@ifpackageloaded { amsmath }
      {
%    \end{macrocode}
% First, we define a function for label setting inside \pkg{amsmath} math
% environments, we want it to set both \cs{zlabel} and \cs{label}.  We may
% ``get a ride'' but not steal the place altogether.  This makes for
% potentially redundant labels, but seems a good compromise.  We \emph{must}
% use the lower level \cs{zref@label} in this context, and hence also handle
% protection with \cs{zref@wrapper@babel}, because \cs{zlabel} makes itself
% no-op when \cs{label} is equal to \cs{ltx@gobble}, and that's precisely the
% case inside the \env{multline} environment (and, damn!, I took a beating of
% this detail\dots{}).
%    \begin{macrocode}
        \cs_set_nopar:Npn \@@_ltxlabel:n #1
          {
            \@@_orig_ltxlabel:n {#1}
            \zref@wrapper@babel \zref@label {#1}
          }
%    \end{macrocode}
% Then we must store the original value of \cs{ltx@label}, which is the macro
% actually responsible for setting the labels inside \pkg{amsmath}'s math
% environments.  And, after that, redefine it to be \cs{@@_ltxlabel:n}
% instead.  We must handle \pkg{hyperref} here, which comes very late in the
% preamble, and which loads \pkg{nameref} at \texttt{begindocument}, which in
% turn, lets \cs{ltx@label} be \cs{label}.  This has to come after
% \pkg{nameref}.  \pkg{cleveref} also redefines it, and comes even later, but
% this procedure is not compatible with it.  Technically, some care is needed
% here, probably mostly on the documentation side.  If \pkg{cleveref} comes
% last and hence its redefinition takes precedence, this is of little
% consequence to \pkg{zref-clever} except that we won't be able to refer to
% the labels in \pkg{amsmath}'s environments with \cs{zcref}.  However, if
% \pkg{cleveref}'s definition is overwritten by \pkg{zref-clever}, this may be
% a substantial problem for \pkg{cleveref}, since it will find the label, but
% it won't contain the data it is expecting.  Therefore, if for some reason
% \pkg{cleveref} is being used alongside \pkg{cleveref}, it is due to follow
% the latter's documented recommendation to load it last.  And use \cs{cref}
% to make references to those.
% CHECK Should I just make this no-op in case 'cleveref' is loaded?
% TODO Remove this compatibility conditional when 2021-11-15 release comes.
%    \begin{macrocode}
        \IfFormatAtLeastTF { 2021-11-15 }
          {
            \@ifpackageloaded { hyperref }
              {
                \AddToHook { package / nameref / after }
                  {
                    \cs_new_eq:NN \@@_orig_ltxlabel:n \ltx@label
                    \cs_set_eq:NN \ltx@label \@@_ltxlabel:n
                  }
              }
              {
                \cs_new_eq:NN \@@_orig_ltxlabel:n \ltx@label
                \cs_set_eq:NN \ltx@label \@@_ltxlabel:n
              }
          }
          {
            \@ifpackageloaded { hyperref }
              {
                \@ifpackageloaded { nameref }
                  {
                    \cs_new_eq:NN \@@_orig_ltxlabel:n \ltx@label
                    \cs_set_eq:NN \ltx@label \@@_ltxlabel:n
                  }
                  {
                    \AddToHook { package / after / nameref }
                      {
                        \cs_new_eq:NN \@@_orig_ltxlabel:n \ltx@label
                        \cs_set_eq:NN \ltx@label \@@_ltxlabel:n
                      }
                  }
              }
              {
                \cs_new_eq:NN \@@_orig_ltxlabel:n \ltx@label
                \cs_set_eq:NN \ltx@label \@@_ltxlabel:n
              }
          }
%    \end{macrocode}
% The \env{subequations} environment uses \texttt{parentequation} and
% \texttt{equation} as counters, but only the later is subject to
% \cs{refstepcounter}.  What happens is: at the start, \texttt{equation} is
% refstepped, it is then stored in \texttt{parentequation} and set to `0' and,
% at the end of the environment it is restored to the value of
% \texttt{parentequation}.  So, here, we really must specify manually
% \opt{currentcounter} and the resetting.  Note that, for \env{subequations},
% \cs{zlabel} works just fine (that is, if given immediately after
% \texttt{\cs{begin}\{subequations\}}, to refer to the parent equation).  See
% \url{https://github.com/latex3/latex2e/issues/687#issuecomment-951451024}
% and subsequent discussion.
%    \begin{macrocode}
        \AddToHook { env / subequations / begin }
          {
            \@@_zcsetup:x
              {
                counterresetby =
                  {
                    parentequation =
                      \@@_counter_reset_by:n { equation } ,
                    equation = parentequation ,
                  } ,
                currentcounter = parentequation ,
                countertype = { parentequation = equation } ,
              }
          }
%    \end{macrocode}
% \pkg{amsmath} does use \cs{refstepcounter} for the \texttt{equation} counter
% throughout.  But we still have to set \opt{currentcounter} manually for two
% reasons.  First: \cs{tag}, which naturally does not change the counter, and
% just sets \cs{@currentlabel}.  Thus a label to a tag gets
% \cs{@currentcounter} from whatever came last, normally the current
% sectioning command.  And we also include the starred environments here, so
% that we can get proper data for \cs{tag}ed equations even if the environment
% is unnumbered.  Second, since we had to manually set \opt{currentcounter} to
% \texttt{parentequation} in \env{subequations}, we also have to manually set
% it to \env{equation} in environments which may be used within it.  The
% \env{xxalignat} environment is not included, because it is ``starred'' by
% default (i.e.\ unnumbered), and does not display or accepts labels or tags
% anyway.  The \env{-ed} (\env{gathered}, \env{aligned}, and \env{alignedat})
% and \env{cases} environments ``must appear within an enclosing math
% environment''.  Same logic applies to other environments defined or
% redefined by the package, like \env{array}, \env{matrix} and variations.
% Finally, \env{split} too can only be used as part of another environment.
%    \begin{macrocode}
        \clist_map_inline:nn
          {
            equation ,
            equation* ,
            align ,
            align* ,
            alignat ,
            alignat* ,
            flalign ,
            flalign* ,
            xalignat ,
            xalignat* ,
            gather ,
            gather* ,
            multline ,
            multline* ,
          }
          {
            \AddToHook { env / #1 / begin }
              { \@@_zcsetup:n { currentcounter = equation } }
          }
%    \end{macrocode}
% And a last touch of care for \pkg{amsmath}'s refinements: make the equation
% references \cs{textup}.
%    \begin{macrocode}
        \zcRefTypeSetup { equation }
          {
            reffont = \upshape ,
            refpre  = {\textup{(}} ,
            refpos  = {\textup{)}} ,
          }
        \msg_info:nnn { zref-clever } { compat-package } { amsmath }
      }
      {}
  }
%    \end{macrocode}
%
%
%
% \subsection{\pkg{mathtools} package}
%
% All math environments defined by \pkg{mathtools}, extending the
% \pkg{amsmath} set, are meant to be used within enclosing math environments,
% hence we don't need to handle them specially, since the numbering and the
% counting is being done on the side of \pkg{amsmath}.  This includes the new
% \env{cases} and \env{matrix} variants, and also \env{multlined}.
%
% Hence, as far as I can tell, the only cross-reference related feature to
% deal with is the \opt{showonlyrefs} option, whose machinery involves writing
% an extra internal label to the \file{.aux} file to track for labels which
% get actually referred to.  This is a little more involved, and implies in
% doing special handling inside \cs{zcref}, but the feature is very cool, so
% it's worth it.
%
%    \begin{macrocode}
\bool_new:N \l_@@_mathtools_showonlyrefs_bool
\AddToHook { begindocument }
  {
    \@ifpackageloaded { mathtools }
      {
        \MH_if_boolean:nT { show_only_refs }
          {
            \bool_set_true:N \l_@@_mathtools_showonlyrefs_bool
            \cs_new_protected:Npn \@@_mathtools_showonlyrefs:n #1
              {
                \@bsphack
                \seq_map_inline:Nn #1
                  {
                    \exp_args:Nx \tl_if_eq:nnTF
                      { \@@_extract_unexp:nnn {##1} { zc@type } { } }
                      { equation }
                      {
                        \protected@write \@auxout { }
                          { \string \MT@newlabel {##1} }
                      }
                      {
                        \exp_args:Nx \tl_if_eq:nnT
                          { \@@_extract_unexp:nnn {##1} { zc@type } { } }
                          { parentequation }
                          {
                            \protected@write \@auxout { }
                              { \string \MT@newlabel {##1} }
                          }
                      }
                  }
                \@esphack
              }
            \msg_info:nnn { zref-clever } { compat-package } { mathtools }
          }
      }
      {}
  }
%    \end{macrocode}
%
%
% \subsection{\pkg{breqn} package}
%
% From the \pkg{breqn} documentation: \textquote{Use of the normal \cs{label}
% command instead of the \opt{label} option works, I think, most of the time
% (untested)}.  Indeed, light testing suggest it does work for \cs{zlabel}
% just as well.  However, if it happens not to work, there was no easy
% alternative handle I could find.  In particular, it does not seem viable to
% leverage the \opt{label=} option without hacking the package internals, even
% if the case of doing so would not be specially tricky, just ``not very
% civil''.
%
%    \begin{macrocode}
\AddToHook { begindocument }
  {
    \@ifpackageloaded { breqn }
      {
%    \end{macrocode}
% Contrary to the practice in \pkg{amsmath}, which prints \cs{tag} even in
% unnumbered environments, the starred environments from \pkg{breqn} don't
% typeset any tag/number at all, even for a manually given \opt{number=} as an
% option.  So, even if one can actually set a label in them, it is not really
% meaningful to make a reference to them.
%    \begin{macrocode}
        \AddToHook { env / dgroup / begin }
          {
            \@@_zcsetup:x
              {
                counterresetby =
                  {
                    parentequation =
                      \@@_counter_reset_by:n { equation } ,
                    equation = parentequation ,
                  } ,
                currentcounter = parentequation ,
                countertype = { parentequation = equation } ,
              }
          }
        \clist_map_inline:nn
          {
            dmath ,
            dseries ,
            darray ,
          }
          {
            \AddToHook { env / #1 / begin }
              { \@@_zcsetup:n { currentcounter = equation } }
          }
      }
      {}
  }
%    \end{macrocode}
%
%
%
% \subsection{\pkg{listings} package}
%
%    \begin{macrocode}
\AddToHook { begindocument }
  {
    \@ifpackageloaded { listings }
      {
        \@@_zcsetup:n
          {
            countertype =
              {
                lstlisting = listing ,
                lstnumber = line ,
              } ,
            counterresetby = { lstnumber = lstlisting } ,
          }
        \lst@AddToHook { Init }
          {
%    \end{macrocode}
% Set (also) a \cs{zlabel} with the label received in the \texttt{label=}
% option from the \texttt{lstlisting} environment.
%    \begin{macrocode}
            \tl_if_empty:NF \lst@label
              { \zlabel { \lst@label } }
%    \end{macrocode}
% The correct place to set \texttt{currentcounter} to \texttt{lstnumber} is
% indeed the \texttt{Init} hook, since \pkg{listings} itself sets
% \cs{@currentlabel} to \cs{thelstnumber} in the same hook.  See section
% ``Line numbers'' of `\texttt{texdoc listings-devel}' (the \file{.dtx}), and
% search for the definition of macro \cs{c@lstnumber}.  Note that
% \pkg{listings} \emph{does use} \cs{refstepcounter}\texttt{\{lstnumber\}},
% but does so in the \texttt{EveryPar} hook, and there must be some grouping
% involved such that \cs{@currentcounter} ends up not being visible to the
% label.  Indeed, the fact that \pkg{listings} manually sets
% \cs{@currentlabel} to \cs{thelstnumber} is a signal that the work of
% \cs{refstepcounter} is being restrained somehow.
%    \begin{macrocode}
            \@@_zcsetup:n { currentcounter = lstnumber }
          }
        \msg_info:nnn { zref-clever } { compat-package } { listings }
      }
      {}
  }
%    \end{macrocode}
%
%
% \subsection{\pkg{enumitem} package}
%
% The procedure below will ``see'' any changes made to the \texttt{enumerate}
% environment (made with \pkg{enumitem}'s \cs{renewlist}) as long as it is
% done in the preamble.  Though, technically, \cs{renewlist} can be issued
% anywhere in the document, this should be more than enough for the purpose at
% hand.  Besides, trying to retrieve this information ``on the fly'' would be
% much overkill.
%
% The only real reason to ``renew'' \texttt{enumerate} itself is to change
% \marg{max-depth}.  \cs{renewlist} \emph{hard-codes} \texttt{max-depth} in
% the environment's definition (well, just as the kernel does), so we cannot
% retrieve this information from any sort of variable.  But \cs{renewlist}
% also creates any needed missing counters, so we can use their existence to
% make the appropriate settings.  In the end, the existence of the counters is
% indeed what matters from \pkg{zref-clever}'s perspective.  Since the first
% four are defined by the kernel and already setup for \pkg{zref-clever} by
% default, we start from \(5\), and stop at the first non-existent
% \cs{c@enumN} counter.
%
%    \begin{macrocode}
\AddToHook { begindocument }
  {
    \@ifpackageloaded { enumitem }
      {
        \int_set:Nn \l_tmpa_int { 5 }
        \bool_while_do:nn
          {
            \cs_if_exist_p:c
              { c@ enum \int_to_roman:n { \l_tmpa_int } }
          }
          {
            \@@_zcsetup:x
              {
                counterresetby =
                  {
                    enum \int_to_roman:n { \l_tmpa_int } =
                    enum \int_to_roman:n { \l_tmpa_int - 1 }
                  } ,
                countertype =
                  { enum \int_to_roman:n { \l_tmpa_int } = item } ,
              }
            \int_incr:N \l_tmpa_int
          }
        \int_compare:nNnT { \l_tmpa_int } > { 5 }
          { \msg_info:nnn { zref-clever } { compat-package } { enumitem } }
      }
      {}
  }
%    \end{macrocode}
%
%
%    \begin{macrocode}
%</package>
%    \end{macrocode}
%
%
%
% \section{Dictionaries}
%
% \subsection{English}
%
%    \begin{macrocode}
%<*package>
\zcDeclareLanguage { english }
\zcDeclareLanguageAlias { american   } { english }
\zcDeclareLanguageAlias { australian } { english }
\zcDeclareLanguageAlias { british    } { english }
\zcDeclareLanguageAlias { canadian   } { english }
\zcDeclareLanguageAlias { newzealand } { english }
\zcDeclareLanguageAlias { UKenglish  } { english }
\zcDeclareLanguageAlias { USenglish  } { english }
%</package>
%    \end{macrocode}
%
%    \begin{macrocode}
%<*dict-english>
%    \end{macrocode}
%
%    \begin{macrocode}
namesep   = {\nobreakspace} ,
pairsep   = {~and\nobreakspace} ,
listsep   = {,~} ,
lastsep   = {~and\nobreakspace} ,
tpairsep  = {~and\nobreakspace} ,
tlistsep  = {,~} ,
tlastsep  = {,~and\nobreakspace} ,
notesep   = {~} ,
rangesep  = {~to\nobreakspace} ,

type = part ,
  Name-sg = Part ,
  name-sg = part ,
  Name-pl = Parts ,
  name-pl = parts ,

type = chapter ,
  Name-sg = Chapter ,
  name-sg = chapter ,
  Name-pl = Chapters ,
  name-pl = chapters ,

type = section ,
  Name-sg = Section ,
  name-sg = section ,
  Name-pl = Sections ,
  name-pl = sections ,

type = paragraph ,
  Name-sg = Paragraph ,
  name-sg = paragraph ,
  Name-pl = Paragraphs ,
  name-pl = paragraphs ,
  Name-sg-ab = Par. ,
  name-sg-ab = par. ,
  Name-pl-ab = Par. ,
  name-pl-ab = par. ,

type = appendix ,
  Name-sg = Appendix ,
  name-sg = appendix ,
  Name-pl = Appendices ,
  name-pl = appendices ,

type = subappendix ,
  Name-sg = Appendix ,
  name-sg = appendix ,
  Name-pl = Appendices ,
  name-pl = appendices ,

type = page ,
  Name-sg = Page ,
  name-sg = page ,
  Name-pl = Pages ,
  name-pl = pages ,
  name-sg-ab = p. ,
  name-pl-ab = pp. ,
  rangesep = {\textendash} ,

type = line ,
  Name-sg = Line ,
  name-sg = line ,
  Name-pl = Lines ,
  name-pl = lines ,

type = figure ,
  Name-sg = Figure ,
  name-sg = figure ,
  Name-pl = Figures ,
  name-pl = figures ,
  Name-sg-ab = Fig. ,
  name-sg-ab = fig. ,
  Name-pl-ab = Figs. ,
  name-pl-ab = figs. ,

type = table ,
  Name-sg = Table ,
  name-sg = table ,
  Name-pl = Tables ,
  name-pl = tables ,

type = item ,
  Name-sg = Item ,
  name-sg = item ,
  Name-pl = Items ,
  name-pl = items ,

type = footnote ,
  Name-sg = Footnote ,
  name-sg = footnote ,
  Name-pl = Footnotes ,
  name-pl = footnotes ,

type = note ,
  Name-sg = Note ,
  name-sg = note ,
  Name-pl = Notes ,
  name-pl = notes ,

type = equation ,
  Name-sg = Equation ,
  name-sg = equation ,
  Name-pl = Equations ,
  name-pl = equations ,
  Name-sg-ab = Eq. ,
  name-sg-ab = eq. ,
  Name-pl-ab = Eqs. ,
  name-pl-ab = eqs. ,
  refpre = {(} ,
  refpos = {)} ,

type = theorem ,
  Name-sg = Theorem ,
  name-sg = theorem ,
  Name-pl = Theorems ,
  name-pl = theorems ,

type = lemma ,
  Name-sg = Lemma ,
  name-sg = lemma ,
  Name-pl = Lemmas ,
  name-pl = lemmas ,

type = corollary ,
  Name-sg = Corollary ,
  name-sg = corollary ,
  Name-pl = Corollaries ,
  name-pl = corollaries ,

type = proposition ,
  Name-sg = Proposition ,
  name-sg = proposition ,
  Name-pl = Propositions ,
  name-pl = propositions ,

type = definition ,
  Name-sg = Definition ,
  name-sg = definition ,
  Name-pl = Definitions ,
  name-pl = definitions ,

type = proof ,
  Name-sg = Proof ,
  name-sg = proof ,
  Name-pl = Proofs ,
  name-pl = proofs ,

type = result ,
  Name-sg = Result ,
  name-sg = result ,
  Name-pl = Results ,
  name-pl = results ,

type = remark ,
  Name-sg = Remark ,
  name-sg = remark ,
  Name-pl = Remarks ,
  name-pl = remarks ,

type = example ,
  Name-sg = Example ,
  name-sg = example ,
  Name-pl = Examples ,
  name-pl = examples ,

type = algorithm ,
  Name-sg = Algorithm ,
  name-sg = algorithm ,
  Name-pl = Algorithms ,
  name-pl = algorithms ,

type = listing ,
  Name-sg = Listing ,
  name-sg = listing ,
  Name-pl = Listings ,
  name-pl = listings ,

type = exercise ,
  Name-sg = Exercise ,
  name-sg = exercise ,
  Name-pl = Exercises ,
  name-pl = exercises ,

type = solution ,
  Name-sg = Solution ,
  name-sg = solution ,
  Name-pl = Solutions ,
  name-pl = solutions ,
%    \end{macrocode}
%
%    \begin{macrocode}
%</dict-english>
%    \end{macrocode}
%
%
%
% \subsection{German}
%
%    \begin{macrocode}
%<*package>
\zcDeclareLanguage
  [ declension = { N , A , D , G } , gender = { f , m , n } , allcaps ]
  { german }
\zcDeclareLanguageAlias { austrian     } { german }
\zcDeclareLanguageAlias { germanb      } { german }
\zcDeclareLanguageAlias { ngerman      } { german }
\zcDeclareLanguageAlias { naustrian    } { german }
\zcDeclareLanguageAlias { nswissgerman } { german }
\zcDeclareLanguageAlias { swissgerman  } { german }
%</package>
%    \end{macrocode}
%
%    \begin{macrocode}
%<*dict-german>
%    \end{macrocode}
%
%    \begin{macrocode}
namesep  = {\nobreakspace} ,
pairsep  = {~und\nobreakspace} ,
listsep  = {,~} ,
lastsep  = {~und\nobreakspace} ,
tpairsep = {~und\nobreakspace} ,
tlistsep = {,~} ,
tlastsep = {~und\nobreakspace} ,
notesep  = {~} ,
rangesep = {~bis\nobreakspace} ,

type = part ,
  gender = m ,
  case = N ,
    Name-sg = Teil ,
    Name-pl = Teile ,
  case = A ,
    Name-sg = Teil ,
    Name-pl = Teile ,
  case = D ,
    Name-sg = Teil ,
    Name-pl = Teilen ,
  case = G ,
    Name-sg = Teiles ,
    Name-pl = Teile ,

type = chapter ,
  gender = n ,
  case = N ,
    Name-sg = Kapitel ,
    Name-pl = Kapitel ,
  case = A ,
    Name-sg = Kapitel ,
    Name-pl = Kapitel ,
  case = D ,
    Name-sg = Kapitel ,
    Name-pl = Kapiteln ,
  case = G ,
    Name-sg = Kapitels ,
    Name-pl = Kapitel ,

type = section ,
  gender = m ,
  case = N ,
    Name-sg = Abschnitt ,
    Name-pl = Abschnitte ,
  case = A ,
    Name-sg = Abschnitt ,
    Name-pl = Abschnitte ,
  case = D ,
    Name-sg = Abschnitt ,
    Name-pl = Abschnitten ,
  case = G ,
    Name-sg = Abschnitts ,
    Name-pl = Abschnitte ,

type = paragraph ,
  gender = m ,
  case = N ,
    Name-sg = Absatz ,
    Name-pl = Absätze ,
  case = A ,
    Name-sg = Absatz ,
    Name-pl = Absätze ,
  case = D ,
    Name-sg = Absatz ,
    Name-pl = Absätzen ,
  case = G ,
    Name-sg = Absatzes ,
    Name-pl = Absätze ,

type = appendix ,
  gender = m ,
  case = N ,
    Name-sg = Anhang ,
    Name-pl = Anhänge ,
  case = A ,
    Name-sg = Anhang ,
    Name-pl = Anhänge ,
  case = D ,
    Name-sg = Anhang ,
    Name-pl = Anhängen ,
  case = G ,
    Name-sg = Anhangs ,
    Name-pl = Anhänge ,

type = subappendix ,
  gender = m ,
  case = N ,
    Name-sg = Anhang ,
    Name-pl = Anhänge ,
  case = A ,
    Name-sg = Anhang ,
    Name-pl = Anhänge ,
  case = D ,
    Name-sg = Anhang ,
    Name-pl = Anhängen ,
  case = G ,
    Name-sg = Anhangs ,
    Name-pl = Anhänge ,

type = page ,
  gender = f ,
  case = N ,
    Name-sg = Seite ,
    Name-pl = Seiten ,
  case = A ,
    Name-sg = Seite ,
    Name-pl = Seiten ,
  case = D ,
    Name-sg = Seite ,
    Name-pl = Seiten ,
  case = G ,
    Name-sg = Seite ,
    Name-pl = Seiten ,
  rangesep = {\textendash} ,

type = line ,
  gender = f ,
  case = N ,
    Name-sg = Zeile ,
    Name-pl = Zeilen ,
  case = A ,
    Name-sg = Zeile ,
    Name-pl = Zeilen ,
  case = D ,
    Name-sg = Zeile ,
    Name-pl = Zeilen ,
  case = G ,
    Name-sg = Zeile ,
    Name-pl = Zeilen ,

type = figure ,
  gender = f ,
  case = N ,
    Name-sg = Abbildung ,
    Name-pl = Abbildungen ,
    Name-sg-ab = Abb. ,
    Name-pl-ab = Abb. ,
  case = A ,
    Name-sg = Abbildung ,
    Name-pl = Abbildungen ,
    Name-sg-ab = Abb. ,
    Name-pl-ab = Abb. ,
  case = D ,
    Name-sg = Abbildung ,
    Name-pl = Abbildungen ,
    Name-sg-ab = Abb. ,
    Name-pl-ab = Abb. ,
  case = G ,
    Name-sg = Abbildung ,
    Name-pl = Abbildungen ,
    Name-sg-ab = Abb. ,
    Name-pl-ab = Abb. ,

type = table ,
  gender = f ,
  case = N ,
    Name-sg = Tabelle ,
    Name-pl = Tabellen ,
  case = A ,
    Name-sg = Tabelle ,
    Name-pl = Tabellen ,
  case = D ,
    Name-sg = Tabelle ,
    Name-pl = Tabellen ,
  case = G ,
    Name-sg = Tabelle ,
    Name-pl = Tabellen ,

type = item ,
  gender = m ,
  case = N ,
    Name-sg = Punkt ,
    Name-pl = Punkte ,
  case = A ,
    Name-sg = Punkt ,
    Name-pl = Punkte ,
  case = D ,
    Name-sg = Punkt ,
    Name-pl = Punkten ,
  case = G ,
    Name-sg = Punktes ,
    Name-pl = Punkte ,

type = footnote ,
  gender = f ,
  case = N ,
    Name-sg = Fußnote ,
    Name-pl = Fußnoten ,
  case = A ,
    Name-sg = Fußnote ,
    Name-pl = Fußnoten ,
  case = D ,
    Name-sg = Fußnote ,
    Name-pl = Fußnoten ,
  case = G ,
    Name-sg = Fußnote ,
    Name-pl = Fußnoten ,

type = note ,
  gender = f ,
  case = N ,
    Name-sg = Anmerkung ,
    Name-pl = Anmerkungen ,
  case = A ,
    Name-sg = Anmerkung ,
    Name-pl = Anmerkungen ,
  case = D ,
    Name-sg = Anmerkung ,
    Name-pl = Anmerkungen ,
  case = G ,
    Name-sg = Anmerkung ,
    Name-pl = Anmerkungen ,

type = equation ,
  gender = f ,
  case = N ,
    Name-sg = Gleichung ,
    Name-pl = Gleichungen ,
  case = A ,
    Name-sg = Gleichung ,
    Name-pl = Gleichungen ,
  case = D ,
    Name-sg = Gleichung ,
    Name-pl = Gleichungen ,
  case = G ,
    Name-sg = Gleichung ,
    Name-pl = Gleichungen ,
  refpre = {(} ,
  refpos = {)} ,

type = theorem ,
  gender = n ,
  case = N ,
    Name-sg = Theorem ,
    Name-pl = Theoreme ,
  case = A ,
    Name-sg = Theorem ,
    Name-pl = Theoreme ,
  case = D ,
    Name-sg = Theorem ,
    Name-pl = Theoremen ,
  case = G ,
    Name-sg = Theorems ,
    Name-pl = Theoreme ,

type = lemma ,
  gender = n ,
  case = N ,
    Name-sg = Lemma ,
    Name-pl = Lemmata ,
  case = A ,
    Name-sg = Lemma ,
    Name-pl = Lemmata ,
  case = D ,
    Name-sg = Lemma ,
    Name-pl = Lemmata ,
  case = G ,
    Name-sg = Lemmas ,
    Name-pl = Lemmata ,

type = corollary ,
  gender = n ,
  case = N ,
    Name-sg = Korollar ,
    Name-pl = Korollare ,
  case = A ,
    Name-sg = Korollar ,
    Name-pl = Korollare ,
  case = D ,
    Name-sg = Korollar ,
    Name-pl = Korollaren ,
  case = G ,
    Name-sg = Korollars ,
    Name-pl = Korollare ,

type = proposition ,
  gender = m ,
  case = N ,
    Name-sg = Satz ,
    Name-pl = Sätze ,
  case = A ,
    Name-sg = Satz ,
    Name-pl = Sätze ,
  case = D ,
    Name-sg = Satz ,
    Name-pl = Sätzen ,
  case = G ,
    Name-sg = Satzes ,
    Name-pl = Sätze ,

type = definition ,
  gender = f ,
  case = N ,
    Name-sg = Definition ,
    Name-pl = Definitionen ,
  case = A ,
    Name-sg = Definition ,
    Name-pl = Definitionen ,
  case = D ,
    Name-sg = Definition ,
    Name-pl = Definitionen ,
  case = G ,
    Name-sg = Definition ,
    Name-pl = Definitionen ,

type = proof ,
  gender = m ,
  case = N ,
    Name-sg = Beweis ,
    Name-pl = Beweise ,
  case = A ,
    Name-sg = Beweis ,
    Name-pl = Beweise ,
  case = D ,
    Name-sg = Beweis ,
    Name-pl = Beweisen ,
  case = G ,
    Name-sg = Beweises ,
    Name-pl = Beweise ,

type = result ,
  gender = n ,
  case = N ,
    Name-sg = Ergebnis ,
    Name-pl = Ergebnisse ,
  case = A ,
    Name-sg = Ergebnis ,
    Name-pl = Ergebnisse ,
  case = D ,
    Name-sg = Ergebnis ,
    Name-pl = Ergebnissen ,
  case = G ,
    Name-sg = Ergebnisses ,
    Name-pl = Ergebnisse ,

type = remark ,
  gender = f ,
  case = N ,
    Name-sg = Bemerkung ,
    Name-pl = Bemerkungen ,
  case = A ,
    Name-sg = Bemerkung ,
    Name-pl = Bemerkungen ,
  case = D ,
    Name-sg = Bemerkung ,
    Name-pl = Bemerkungen ,
  case = G ,
    Name-sg = Bemerkung ,
    Name-pl = Bemerkungen ,

type = example ,
  gender = n ,
  case = N ,
    Name-sg = Beispiel ,
    Name-pl = Beispiele ,
  case = A ,
    Name-sg = Beispiel ,
    Name-pl = Beispiele ,
  case = D ,
    Name-sg = Beispiel ,
    Name-pl = Beispielen ,
  case = G ,
    Name-sg = Beispiels ,
    Name-pl = Beispiele ,

type = algorithm ,
  gender = m ,
  case = N ,
    Name-sg = Algorithmus ,
    Name-pl = Algorithmen ,
  case = A ,
    Name-sg = Algorithmus ,
    Name-pl = Algorithmen ,
  case = D ,
    Name-sg = Algorithmus ,
    Name-pl = Algorithmen ,
  case = G ,
    Name-sg = Algorithmus ,
    Name-pl = Algorithmen ,

type = listing ,
  gender = n ,
  case = N ,
    Name-sg = Listing ,
    Name-pl = Listings ,
  case = A ,
    Name-sg = Listing ,
    Name-pl = Listings ,
  case = D ,
    Name-sg = Listing ,
    Name-pl = Listings ,
  case = G ,
    Name-sg = Listings ,
    Name-pl = Listings ,

type = exercise ,
  gender = f ,
  case = N ,
    Name-sg = Übungsaufgabe ,
    Name-pl = Übungsaufgaben ,
  case = A ,
    Name-sg = Übungsaufgabe ,
    Name-pl = Übungsaufgaben ,
  case = D ,
    Name-sg = Übungsaufgabe ,
    Name-pl = Übungsaufgaben ,
  case = G ,
    Name-sg = Übungsaufgabe ,
    Name-pl = Übungsaufgaben ,

type = solution ,
  gender = f ,
  case = N ,
    Name-sg = Lösung ,
    Name-pl = Lösungen ,
  case = A ,
    Name-sg = Lösung ,
    Name-pl = Lösungen ,
  case = D ,
    Name-sg = Lösung ,
    Name-pl = Lösungen ,
  case = G ,
    Name-sg = Lösung ,
    Name-pl = Lösungen ,
%    \end{macrocode}
%
%    \begin{macrocode}
%</dict-german>
%    \end{macrocode}
%
%
%
% \subsection{French}
%
%    \begin{macrocode}
%<*package>
\zcDeclareLanguage [ gender = { f , m } ] { french }
\zcDeclareLanguageAlias { acadian  } { french }
\zcDeclareLanguageAlias { canadien } { french }
\zcDeclareLanguageAlias { francais } { french }
\zcDeclareLanguageAlias { frenchb  } { french }
%</package>
%    \end{macrocode}
%
%    \begin{macrocode}
%<*dict-french>
%    \end{macrocode}
%
%    \begin{macrocode}
namesep  = {\nobreakspace} ,
pairsep  = {~et\nobreakspace} ,
listsep  = {,~} ,
lastsep  = {~et\nobreakspace} ,
tpairsep = {~et\nobreakspace} ,
tlistsep = {,~} ,
tlastsep = {~et\nobreakspace} ,
notesep  = {~} ,
rangesep = {~à\nobreakspace} ,

type = part ,
  gender = f ,
  Name-sg = Partie ,
  name-sg = partie ,
  Name-pl = Parties ,
  name-pl = parties ,

type = chapter ,
  gender = m ,
  Name-sg = Chapitre ,
  name-sg = chapitre ,
  Name-pl = Chapitres ,
  name-pl = chapitres ,

type = section ,
  gender = f ,
  Name-sg = Section ,
  name-sg = section ,
  Name-pl = Sections ,
  name-pl = sections ,

type = paragraph ,
  gender = m ,
  Name-sg = Paragraphe ,
  name-sg = paragraphe ,
  Name-pl = Paragraphes ,
  name-pl = paragraphes ,

type = appendix ,
  gender = f ,
  Name-sg = Annexe ,
  name-sg = annexe ,
  Name-pl = Annexes ,
  name-pl = annexes ,

type = subappendix ,
  gender = f ,
  Name-sg = Annexe ,
  name-sg = annexe ,
  Name-pl = Annexes ,
  name-pl = annexes ,

type = page ,
  gender = f ,
  Name-sg = Page ,
  name-sg = page ,
  Name-pl = Pages ,
  name-pl = pages ,
  rangesep = {\textendash} ,

type = line ,
  gender = f ,
  Name-sg = Ligne ,
  name-sg = ligne ,
  Name-pl = Lignes ,
  name-pl = lignes ,

type = figure ,
  gender = f ,
  Name-sg = Figure ,
  name-sg = figure ,
  Name-pl = Figures ,
  name-pl = figures ,

type = table ,
  gender = f ,
  Name-sg = Table ,
  name-sg = table ,
  Name-pl = Tables ,
  name-pl = tables ,

type = item ,
  gender = m ,
  Name-sg = Point ,
  name-sg = point ,
  Name-pl = Points ,
  name-pl = points ,

type = footnote ,
  gender = f ,
  Name-sg = Note ,
  name-sg = note ,
  Name-pl = Notes ,
  name-pl = notes ,

type = note ,
  gender = f ,
  Name-sg = Note ,
  name-sg = note ,
  Name-pl = Notes ,
  name-pl = notes ,

type = equation ,
  gender = f ,
  Name-sg = Équation ,
  name-sg = équation ,
  Name-pl = Équations ,
  name-pl = équations ,
  refpre = {(} ,
  refpos = {)} ,

type = theorem ,
  gender = m ,
  Name-sg = Théorème ,
  name-sg = théorème ,
  Name-pl = Théorèmes ,
  name-pl = théorèmes ,

type = lemma ,
  gender = m ,
  Name-sg = Lemme ,
  name-sg = lemme ,
  Name-pl = Lemmes ,
  name-pl = lemmes ,

type = corollary ,
  gender = m ,
  Name-sg = Corollaire ,
  name-sg = corollaire ,
  Name-pl = Corollaires ,
  name-pl = corollaires ,

type = proposition ,
  gender = f ,
  Name-sg = Proposition ,
  name-sg = proposition ,
  Name-pl = Propositions ,
  name-pl = propositions ,

type = definition ,
  gender = f ,
  Name-sg = Définition ,
  name-sg = définition ,
  Name-pl = Définitions ,
  name-pl = définitions ,

type = proof ,
  gender = f ,
  Name-sg = Démonstration ,
  name-sg = démonstration ,
  Name-pl = Démonstrations ,
  name-pl = démonstrations ,

type = result ,
  gender = m ,
  Name-sg = Résultat ,
  name-sg = résultat ,
  Name-pl = Résultats ,
  name-pl = résultats ,

type = remark ,
  gender = f ,
  Name-sg = Remarque ,
  name-sg = remarque ,
  Name-pl = Remarques ,
  name-pl = remarques ,

type = example ,
  gender = m ,
  Name-sg = Exemple ,
  name-sg = exemple ,
  Name-pl = Exemples ,
  name-pl = exemples ,

type = algorithm ,
  gender = m ,
  Name-sg = Algorithme ,
  name-sg = algorithme ,
  Name-pl = Algorithmes ,
  name-pl = algorithmes ,

type = listing ,
  gender = f ,
  Name-sg = Liste ,
  name-sg = liste ,
  Name-pl = Listes ,
  name-pl = listes ,

type = exercise ,
  gender = m ,
  Name-sg = Exercice ,
  name-sg = exercice ,
  Name-pl = Exercices ,
  name-pl = exercices ,

type = solution ,
  gender = f ,
  Name-sg = Solution ,
  name-sg = solution ,
  Name-pl = Solutions ,
  name-pl = solutions ,
%    \end{macrocode}
%
%    \begin{macrocode}
%</dict-french>
%    \end{macrocode}
%
%
%
% \subsection{Portuguese}
%
%    \begin{macrocode}
%<*package>
\zcDeclareLanguage [ gender = { f , m } ] { portuguese }
\zcDeclareLanguageAlias { brazilian } { portuguese }
\zcDeclareLanguageAlias { brazil    } { portuguese }
\zcDeclareLanguageAlias { portuges  } { portuguese }
%</package>
%    \end{macrocode}
%
%    \begin{macrocode}
%<*dict-portuguese>
%    \end{macrocode}
%
%    \begin{macrocode}
namesep  = {\nobreakspace} ,
pairsep  = {~e\nobreakspace} ,
listsep  = {,~} ,
lastsep  = {~e\nobreakspace} ,
tpairsep = {~e\nobreakspace} ,
tlistsep = {,~} ,
tlastsep = {~e\nobreakspace} ,
notesep  = {~} ,
rangesep = {~a\nobreakspace} ,

type = part ,
  gender = f ,
  Name-sg = Parte ,
  name-sg = parte ,
  Name-pl = Partes ,
  name-pl = partes ,

type = chapter ,
  gender = m ,
  Name-sg = Capítulo ,
  name-sg = capítulo ,
  Name-pl = Capítulos ,
  name-pl = capítulos ,

type = section ,
  gender = f ,
  Name-sg = Seção ,
  name-sg = seção ,
  Name-pl = Seções ,
  name-pl = seções ,

type = paragraph ,
  gender = m ,
  Name-sg = Parágrafo ,
  name-sg = parágrafo ,
  Name-pl = Parágrafos ,
  name-pl = parágrafos ,
  Name-sg-ab = Par. ,
  name-sg-ab = par. ,
  Name-pl-ab = Par. ,
  name-pl-ab = par. ,

type = appendix ,
  gender = m ,
  Name-sg = Apêndice ,
  name-sg = apêndice ,
  Name-pl = Apêndices ,
  name-pl = apêndices ,

type = subappendix ,
  gender = m ,
  Name-sg = Apêndice ,
  name-sg = apêndice ,
  Name-pl = Apêndices ,
  name-pl = apêndices ,

type = page ,
  gender = f ,
  Name-sg = Página ,
  name-sg = página ,
  Name-pl = Páginas ,
  name-pl = páginas ,
  name-sg-ab = p. ,
  name-pl-ab = pp. ,
  rangesep = {\textendash} ,

type = line ,
  gender = f ,
  Name-sg = Linha ,
  name-sg = linha ,
  Name-pl = Linhas ,
  name-pl = linhas ,

type = figure ,
  gender = f ,
  Name-sg = Figura ,
  name-sg = figura ,
  Name-pl = Figuras ,
  name-pl = figuras ,
  Name-sg-ab = Fig. ,
  name-sg-ab = fig. ,
  Name-pl-ab = Figs. ,
  name-pl-ab = figs. ,

type = table ,
  gender = f ,
  Name-sg = Tabela ,
  name-sg = tabela ,
  Name-pl = Tabelas ,
  name-pl = tabelas ,

type = item ,
  gender = m ,
  Name-sg = Item ,
  name-sg = item ,
  Name-pl = Itens ,
  name-pl = itens ,

type = footnote ,
  gender = f ,
  Name-sg = Nota ,
  name-sg = nota ,
  Name-pl = Notas ,
  name-pl = notas ,

type = note ,
  gender = f ,
  Name-sg = Nota ,
  name-sg = nota ,
  Name-pl = Notas ,
  name-pl = notas ,

type = equation ,
  gender = f ,
  Name-sg = Equação ,
  name-sg = equação ,
  Name-pl = Equações ,
  name-pl = equações ,
  Name-sg-ab = Eq. ,
  name-sg-ab = eq. ,
  Name-pl-ab = Eqs. ,
  name-pl-ab = eqs. ,
  refpre = {(} ,
  refpos = {)} ,

type = theorem ,
  gender = m ,
  Name-sg = Teorema ,
  name-sg = teorema ,
  Name-pl = Teoremas ,
  name-pl = teoremas ,

type = lemma ,
  gender = m ,
  Name-sg = Lema ,
  name-sg = lema ,
  Name-pl = Lemas ,
  name-pl = lemas ,

type = corollary ,
  gender = m ,
  Name-sg = Corolário ,
  name-sg = corolário ,
  Name-pl = Corolários ,
  name-pl = corolários ,

type = proposition ,
  gender = f ,
  Name-sg = Proposição ,
  name-sg = proposição ,
  Name-pl = Proposições ,
  name-pl = proposições ,

type = definition ,
  gender = f ,
  Name-sg = Definição ,
  name-sg = definição ,
  Name-pl = Definições ,
  name-pl = definições ,

type = proof ,
  gender = f ,
  Name-sg = Demonstração ,
  name-sg = demonstração ,
  Name-pl = Demonstrações ,
  name-pl = demonstrações ,

type = result ,
  gender = m ,
  Name-sg = Resultado ,
  name-sg = resultado ,
  Name-pl = Resultados ,
  name-pl = resultados ,

type = remark ,
  gender = f ,
  Name-sg = Observação ,
  name-sg = observação ,
  Name-pl = Observações ,
  name-pl = observações ,

type = example ,
  gender = m ,
  Name-sg = Exemplo ,
  name-sg = exemplo ,
  Name-pl = Exemplos ,
  name-pl = exemplos ,

type = algorithm ,
  gender = m ,
  Name-sg = Algoritmo ,
  name-sg = algoritmo ,
  Name-pl = Algoritmos ,
  name-pl = algoritmos ,

type = listing ,
  gender = f ,
  Name-sg = Listagem ,
  name-sg = listagem ,
  Name-pl = Listagens ,
  name-pl = listagens ,

type = exercise ,
  gender = m ,
  Name-sg = Exercício ,
  name-sg = exercício ,
  Name-pl = Exercícios ,
  name-pl = exercícios ,

type = solution ,
  gender = f ,
  Name-sg = Solução ,
  name-sg = solução ,
  Name-pl = Soluções ,
  name-pl = soluções ,
%    \end{macrocode}
%
%    \begin{macrocode}
%</dict-portuguese>
%    \end{macrocode}
%
%
%
% \subsection{Spanish}
%
%    \begin{macrocode}
%<*package>
\zcDeclareLanguage [ gender = { f , m } ] { spanish }
%</package>
%    \end{macrocode}
%
%    \begin{macrocode}
%<*dict-spanish>
%    \end{macrocode}
%
%    \begin{macrocode}
namesep  = {\nobreakspace} ,
pairsep  = {~y\nobreakspace} ,
listsep  = {,~} ,
lastsep  = {~y\nobreakspace} ,
tpairsep = {~y\nobreakspace} ,
tlistsep = {,~} ,
tlastsep = {~y\nobreakspace} ,
notesep  = {~} ,
rangesep = {~a\nobreakspace} ,

type = part ,
  gender = f ,
  Name-sg = Parte ,
  name-sg = parte ,
  Name-pl = Partes ,
  name-pl = partes ,

type = chapter ,
  gender = m ,
  Name-sg = Capítulo ,
  name-sg = capítulo ,
  Name-pl = Capítulos ,
  name-pl = capítulos ,

type = section ,
  gender = f ,
  Name-sg = Sección ,
  name-sg = sección ,
  Name-pl = Secciones ,
  name-pl = secciones ,

type = paragraph ,
  gender = m ,
  Name-sg = Párrafo ,
  name-sg = párrafo ,
  Name-pl = Párrafos ,
  name-pl = párrafos ,

type = appendix ,
  gender = m ,
  Name-sg = Apéndice ,
  name-sg = apéndice ,
  Name-pl = Apéndices ,
  name-pl = apéndices ,

type = subappendix ,
  gender = m ,
  Name-sg = Apéndice ,
  name-sg = apéndice ,
  Name-pl = Apéndices ,
  name-pl = apéndices ,

type = page ,
  gender = f ,
  Name-sg = Página ,
  name-sg = página ,
  Name-pl = Páginas ,
  name-pl = páginas ,
  rangesep = {\textendash} ,

type = line ,
  gender = f ,
  Name-sg = Línea ,
  name-sg = línea ,
  Name-pl = Líneas ,
  name-pl = líneas ,

type = figure ,
  gender = f ,
  Name-sg = Figura ,
  name-sg = figura ,
  Name-pl = Figuras ,
  name-pl = figuras ,

type = table ,
  gender = m ,
  Name-sg = Cuadro ,
  name-sg = cuadro ,
  Name-pl = Cuadros ,
  name-pl = cuadros ,

type = item ,
  gender = m ,
  Name-sg = Punto ,
  name-sg = punto ,
  Name-pl = Puntos ,
  name-pl = puntos ,

type = footnote ,
  gender = f ,
  Name-sg = Nota ,
  name-sg = nota ,
  Name-pl = Notas ,
  name-pl = notas ,

type = note ,
  gender = f ,
  Name-sg = Nota ,
  name-sg = nota ,
  Name-pl = Notas ,
  name-pl = notas ,

type = equation ,
  gender = f ,
  Name-sg = Ecuación ,
  name-sg = ecuación ,
  Name-pl = Ecuaciones ,
  name-pl = ecuaciones ,
  refpre = {(} ,
  refpos = {)} ,

type = theorem ,
  gender = m ,
  Name-sg = Teorema ,
  name-sg = teorema ,
  Name-pl = Teoremas ,
  name-pl = teoremas ,

type = lemma ,
  gender = m ,
  Name-sg = Lema ,
  name-sg = lema ,
  Name-pl = Lemas ,
  name-pl = lemas ,

type = corollary ,
  gender = m ,
  Name-sg = Corolario ,
  name-sg = corolario ,
  Name-pl = Corolarios ,
  name-pl = corolarios ,

type = proposition ,
  gender = f ,
  Name-sg = Proposición ,
  name-sg = proposición ,
  Name-pl = Proposiciones ,
  name-pl = proposiciones ,

type = definition ,
  gender = f ,
  Name-sg = Definición ,
  name-sg = definición ,
  Name-pl = Definiciones ,
  name-pl = definiciones ,

type = proof ,
  gender = f ,
  Name-sg = Demostración ,
  name-sg = demostración ,
  Name-pl = Demostraciones ,
  name-pl = demostraciones ,

type = result ,
  gender = m ,
  Name-sg = Resultado ,
  name-sg = resultado ,
  Name-pl = Resultados ,
  name-pl = resultados ,

type = remark ,
  gender = f ,
  Name-sg = Observación ,
  name-sg = observación ,
  Name-pl = Observaciones ,
  name-pl = observaciones ,

type = example ,
  gender = m ,
  Name-sg = Ejemplo ,
  name-sg = ejemplo ,
  Name-pl = Ejemplos ,
  name-pl = ejemplos ,

type = algorithm ,
  gender = m ,
  Name-sg = Algoritmo ,
  name-sg = algoritmo ,
  Name-pl = Algoritmos ,
  name-pl = algoritmos ,

type = listing ,
  gender = m ,
  Name-sg = Listado ,
  name-sg = listado ,
  Name-pl = Listados ,
  name-pl = listados ,

type = exercise ,
  gender = m ,
  Name-sg = Ejercicio ,
  name-sg = ejercicio ,
  Name-pl = Ejercicios ,
  name-pl = ejercicios ,

type = solution ,
  gender = f ,
  Name-sg = Solución ,
  name-sg = solución ,
  Name-pl = Soluciones ,
  name-pl = soluciones ,
%    \end{macrocode}
%
%    \begin{macrocode}
%</dict-spanish>
%    \end{macrocode}
%
%
% \PrintIndex
%
% \end{implementation}
