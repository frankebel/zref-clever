% \iffalse meta-comment
%
% File: zref-clever.dtx
%
% This file is part of the LaTeX package "zref-clever".
%
% Copyright (C) 2021  Gustavo Barros
%
% It may be distributed and/or modified under the conditions of the
% LaTeX Project Public License (LPPL), either version 1.3c of this
% license or (at your option) any later version.  The latest version
% of this license is in the file:
%
%    https://www.latex-project.org/lppl.txt
%
% and version 1.3 or later is part of all distributions of LaTeX
% version 2005/12/01 or later.
%
%
% This work is "maintained" (as per LPPL maintenance status) by
% Gustavo Barros.
%
% This work consists of the files zref-clever.dtx,
%                                 zref-clever.ins,
%                                 zref-clever.tex,
%                                 zref-clever-code.tex,
%           and the derived files zref-clever.sty and
%                                 zref-clever.pdf,
%                                 zref-clever-code.pdf.
%
% The released version of this package is available from CTAN.
%
% -----------------------------------------------------------------------
%
% The development version of the package can be found at
%
%    https://github.com/gusbrs/zref-clever
%
% for those people who are interested.
%
% -----------------------------------------------------------------------
%
% \fi
%
% \iffalse
%<*driver>
\documentclass{l3doc}
% Have \GetFileInfo pick up date and version data
\usepackage{zref-check}
\NewDocumentCommand\opt{m}{\texttt{#1}}
\MakeShortVerb{\|}
\begin{document}
  \DocInput{zref-clever.dtx}
\end{document}
%</driver>
% \fi
%
%
% \begin{documentation}
%
%
% \end{documentation}
%
%
%
% \begin{implementation}
%
% \section{Initial setup}
%
% Start the \pkg{DocStrip} guards.
%    \begin{macrocode}
%<*package>
%    \end{macrocode}
%
% Identify the internal prefix (\LaTeX3 \pkg{DocStrip} convention).
%    \begin{macrocode}
%<@@=zrefclever>
%    \end{macrocode}
%
%
%
% Identify the package.
%    \begin{macrocode}
\ProvidesExplPackage {zref-clever} {2021-08-09} {0.1.0-alpha}
  {Do-what-I-mean cross-references based on zref}
%    \end{macrocode}
%
% \section{Dependencies}
%
%    \begin{macrocode}
\RequirePackage { zref-user }
\RequirePackage { zref-counter }
% FIXME I should not load 'zref-titleref' by default, since it patches a lot
% of stuff, including 'caption', all sectioning commands etc.  I should make
% the package use it, if it is loaded.
% \RequirePackage { zref-titleref }
% CHECK 'abspage' to sort pages?  Check if really needed.
% \RequirePackage { zref-abspage }
\RequirePackage { translations }
%    \end{macrocode}
%
%
%
% \section{\pkg{zref} setup}
%
% We are interested in three basic label elements: the reference itself, the
% page, and the counter.  The `page' and `counter' are respectively handled by
% modules \pkg{zref-base} and \pkg{zref-counter}.  But the reference itself,
% stored by \pkg{zref} in the `default' field, is somewhat a disputed real
% estate.  In particular, the use of \cs{labelformat} will include there the
% reference ``prefix'' and complicate the job we are trying to do here.
% Hence, we isolate \cs{the}\meta{counter} and store it ``clean'' in
% \texttt{zc@thecounter} for reserved use.
%
% From the definition of \cs{@currentlabel} done inside \cs{refstepcounter} in
% 'texdoc source2e', section 'ltcounts.dtx'.  We just drop the \cs{p@...}
% prefix.
%    \begin{macrocode}
\zref@newprop { zc@thecounter } { \cs:w the \@currentcounter \cs_end: }
\zref@addprop \ZREF@mainlist { zc@thecounter }
%    \end{macrocode}
%
% However, the moment where the label is set is a privileged one, because at
% this point we have a lot of raw information available.  Information which
% may be difficult to retrieve later on by parsing the reference printed value
% of the counter, which we stored in \texttt{zc@thecounter} above.  Hence, we
% seize the opportunity to store some of that information in a way which eases
% significantly the task of processing the reference later on: i) the counter
% \emph{value}, as a number; ii) the counter values of the set of counters
% which may trigger a reset of the current counter.
%
% The first one is trivial, \cs{c@}\meta{counter} contains the counter's
% numerical value (see `texdoc source2e', section `ltcounts.dtx'), we just
% store it in \texttt{zc@countervalue}.
%    \begin{macrocode}
\zref@newprop { zc@countervalue } { \int_use:c { c@\@currentcounter } }
\zref@addprop \ZREF@mainlist { zc@countervalue }
%    \end{macrocode}
%
% The second one is trickier.  For starters, the counters which may reset the
% current counter are not retrievable from the counter itself, because this
% information is stored with the counter that does the resetting, not with the
% one that gets reset (the list is stored in \cs{cl@}\meta{counter} with
% format |\@elt{countera}\@elt{counterb}\@elt{counterc}|, again see section
% `ltcounts.dtx' in `source2e').  Besides, there may be a chain of resetting
% counters, which must be taken into account.  The procedure below examines a
% set of counters, those included in \cs{g_@@_reseters_seq}, and for each
% counter each of those reset, define a \texttt{tl} variable with name
% \cs{g_@@_counter_ \meta{counter} _within_tl} storing the value of its
% ``within-counter''.  These \texttt{tl} variables are then used by
% \cs{@@_get_within_counters:n} to store the desired information when the
% label is actually set.  There are two relevante caveats to this procedure:
% i) \cs{g_@@_reseters_seq} is populated by hand with the ``usual suspects'',
% there is no way (that I know of) to ensure it is exhaustive; ii)
% \cs{@@_set_within_counters:} is being run \cs{AtBeginDocument}, hence if
% counters resetting behavior gets redefined mid-document, or new ones
% created, this information will not be taken into account.  I believe these
% caveats are not particularly grave to the case at hand: i) it is not that
% difficult to create a reasonable ``usual suspects'' list which, of course,
% should include the counters for the sectioning commands, to start with; ii)
% it is easy to add more counters to this list if needed; iii) setting one's
% counters (the ones that may get reset) in the preamble is not that
% restrictive a limitation, indeed, I'd expect the ones we are interested in
% here to be set by \cs{documentclass}.
%
% \begin{variable}{\g_@@_reseters_seq}
%   Stores the list of counters which get examined \cs{AtBeginDocument} as
%   potential ``within-counters'' for other counters.
%    \begin{macrocode}
\seq_set_from_clist:Nn \g_@@_reseters_seq
  {
    part ,
    chapter ,
    section ,
    subsection ,
    subsubsection ,
    paragraph ,
    subparagraph
  }
%    \end{macrocode}
% \end{variable}
%
% \begin{macro}{\@@_set_within_counters:}
%   Map through \cs{g_@@_reseters_seq} defining \texttt{tl} variables storing,
%   for each counter that may be reset, the ``within-counter'' which may do
%   so.
%    \begin{macrocode}
\cs_new:Npn \@@_set_within_counters:
  {
    \group_begin:
      \seq_map_inline:Nn \g_@@_reseters_seq
        {
          \cs_if_exist:cT { c@ ##1 }
            {
              \cs_set:Npn \@@_elt_aux:n ####1
                {
                  \tl_if_exist:cF { g_@@_counter_ ####1 _within_tl }
                    { \tl_gset:cx { g_@@_counter_ ####1 _within_tl } {##1} }
                }
              \cs_set_eq:NN \@elt \@@_elt_aux:n
              \use:c { cl@ ##1 }
            }
        }
    \group_end:
  }
%    \end{macrocode}
% And run it \cs{AtBeginDocument}.
%    \begin{macrocode}
\AtBeginDocument { \@@_set_within_counters: }
%    \end{macrocode}
% \end{macro}
%
% \begin{macro}{\@@_get_within_counters:n}
%   Recursively generate a \emph{sequence} of ``within-counters'' values, for
%   a given \Arg{counter}.
%    \begin{macrocode}
\cs_new:Npn \@@_get_within_counters:n #1
  {
    \bool_lazy_and:nnT
      { \tl_if_exist_p:c { g_@@_counter_ \tl_use:N #1 _within_tl } }
      {
        \cs_if_exist_p:c
          { c@ \use:c { g_@@_counter_ \tl_use:N #1 _within_tl } }
      }
      {
        { \int_use:c { c@ \use:c { g_@@_counter_ \tl_use:N #1 _within_tl } } }
        \@@_get_within_counters:n
          { \use:c { g_@@_counter_ \tl_use:N #1 _within_tl } }
      }
  }
%    \end{macrocode}
% \end{macro}
%
% Finally, add \texttt{zc@within} to \pkg{zref}'s \texttt{main} property list.
%    \begin{macrocode}
\zref@newprop { zc@within }
  { \@@_get_within_counters:n { \@currentcounter } }
\zref@addprop \ZREF@mainlist { zc@within }
%    \end{macrocode}
%
%
%
%    \begin{macrocode}
%</package>
%    \end{macrocode}
%
% \PrintIndex
%
% \end{implementation}
%
